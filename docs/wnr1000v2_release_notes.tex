\documentclass[12pt]{report}
\textwidth 6.5in
\oddsidemargin 0pt
\textheight 9in
\topmargin 0pt
\headsep 0pt
\headheight 0pt
%\linespread{1.6}
\usepackage{amssymb,amsmath,epsfig,subfigure,psfrag}
\newtheorem{prop}{Proposition}
\newtheorem{Def}{Definition}
\newtheorem{Cor}{Corollary}
\newtheorem{Lem}{Lemma}
\newtheorem{Thm}{Theorem}
\newtheorem{Rem}{Remarks}
\newcommand{\fclass}[1]{{\mathcal{#1}}} % function classes
\newcommand{\msae}[1]{\bar{#1}}% Maximal Sub-Additive Embedded
\newcommand{\mlabel}[1]{
  \marginpar{#1} %to be deleted for hardcopy
  }
\newcommand{\tlabel}[1]{
  \label{#1}%
%  \marginpar{#1} %to be deleted for hardcopy
  }
\newcommand{\ispnpedf}{\mbox{ISP-NPEDF}}

\newcommand{\myvector}[1]{
  \boldsymbol{#1}%
}
\newcommand{\mathset}[1]{
  \mathbb{#1}%
}

\newcommand{\esssup}[1]{
  \mbox{\upshape ess}_{#1}\!\sup%
}

\newcommand{\abs}[1]{
  | #1 |%
}

\newcommand{\norm}[1]{
  \| #1 \|
}

\newcommand{\mybegineq}[1]{
  \marginpar{#1} %to be deleted for hardcopy
  \begin{equation}\label{#1}
}

\newcommand{\myendeq}{
  \end{equation}%
}





\newcommand{\myscriptsize}{\small}
\newcommand{\proof}{\noindent\hspace{2em}{\it Proof: }}
\newcommand{\proofof}[1]{\noindent\hspace{2em}{\it Proof of #1: }}
\newcommand{\myendproof}{\hspace*{\fill}$\blacksquare$}

\begin{document}
\title{WNR1000v2 Release notes}
% \author{\normalsize{Chia-Sheng Chang}\thanks{
%     Corresponding author: Chia-Sheng Chang, e-mail:
%     \texttt{changcs@santos.ee.ntu.edu.tw,}
%     postal address: Room 533, Electrical Engineering Building, No.\ 1, Sec.\
%     4, Roosevelt Rd., Taipei 10617, Taiwan, R.O.C.}
%   \normalsize{ and Kwang-Cheng Chen} \\
%   \texttt{\small changcs@santos.ee.ntu.edu.tw, chenkc@cc.ee.ntu.edu.tw}\\
%   {\small Institute of Communications Engineering and Department of
%     Electrical Engineering}\\ {\small National Taiwan University}
%   }

\author{\normalsize{Ronger Pan}}

\maketitle
\thispagestyle{empty}
%% \begin{abstract}
%%   \noindent Large delay schedulable regions, work-conserving property,
%%   inter-session protection property, and acceptable realization
%%   complexity are widely accepted performance metrics for
%%   comparing packet
%%   scheduling disciplines in packet-switching networks. % Under
%% %  continuous-time assumptions, it is a difficult task to design a
%% %  scheduling discipline with all these advantages.
%%   In this paper, we propose a \emph{work-conserving} scheduling
%%   discipline, called \ispnpedf, with \emph{optimal} delay schedulable
%%   regions, and inter-session protection property. In addition,
%%   the realization complexity of \ispnpedf\ is lower than that of
%%   plain EDF (Earliest-Deadline First) with traffic regulators.



%% %  \bfseries keywords: keywords

%% \end{abstract}

\psfull
\tableofcontents
\thispagestyle{empty}
\newpage
\setcounter{page}{1}

\chapter{Revision History}
\tlabel{sec:revision-history}

   \begin{tabular}{|c||c|c|} \hline
      Date & firmware version  & Revised by \\ \hline
      2008/08/25 & Release V1.0.0.0 & Ronger Pan \\ \hline
      2008/08/28 & Release V1.0.0.2 & Ronger Pan \\ \hline
      2008/09/03 & Release V1.0.0.3 & Ronger Pan \\ \hline
      2008/09/03 & Release V1.0.0.4 & Ronger Pan \\ \hline
      2008/09/09 & Release V1.0.0.5 & Ronger Pan \\ \hline
      2008/09/10 & Release V1.0.0.6 & Ronger Pan \\ \hline
      2008/09/12 & Release V1.0.0.7 & Ronger Pan \\ \hline
      2008/09/22 & Release V1.0.0.8 & Ronger Pan \\ \hline
	  2008/10/08 & Release V1.0.0.9 & Ronger Pan \\ \hline
	  2008/10/15 & Release V1.0.1.0 & Ronger Pan \\ \hline
	  2008/10/17 & Release V1.0.1.1 & Ronger Pan \\ \hline
	  2008/10/22 & Release V1.0.1.2 & Ronger Pan \\ \hline
	  2008/10/24 & Release V1.0.1.3 & Ronger Pan \\ \hline
	  2008/10/29 & Release V1.0.1.4 & Ronger Pan \\ \hline
	  2008/11/05 & Release V1.0.1.5 & Ronger Pan \\ \hline
	  2008/11/07 & Release V1.0.1.6 & Ronger Pan \\ \hline
	  2008/11/14 & Release V1.0.1.7 & Ronger Pan \\ \hline
	  2008/11/19 & Release V1.0.1.8 & Ronger Pan \\ \hline
	  2008/11/26 & Release V1.0.1.9 & Ronger Pan \\ \hline
	  2008/12/02 & Release V1.0.2.0 & Ronger Pan \\ \hline
	  2008/12/09 & Release V1.0.2.1 & Ronger Pan \\ \hline
	  2008/12/12 & Release V1.0.2.2 & Ronger Pan \\ \hline
	  2008/12/17 & Release V1.0.2.3 & Ronger Pan \\ \hline
	  2008/12/19 & Release V1.0.2.4 & Ronger Pan \\ \hline
	  2008/12/24 & Release V1.0.2.5 & Ronger Pan \\ \hline
	  2008/12/26 & Release V1.0.2.6 & Ronger Pan \\ \hline
	  2009/01/06 & Release V1.0.2.7 & Ronger Pan \\ \hline
	  2009/01/14 & Release V1.0.2.8 & Ronger Pan \\ \hline
	  2009/01/21 & Release V1.0.2.9 & Ronger Pan \\ \hline
	  2009/02/05 & Release V1.0.3.0 & Ronger Pan \\ \hline
	  2009/02/09 & Release V1.0.3.1 & Ronger Pan \\ \hline
	  2009/02/13 & Release V1.0.3.2 & Ronger Pan \\ \hline
	  2009/02/18 & Release V1.0.3.3 & Ronger Pan \\ \hline
	  2009/02/23 & Release V1.0.3.4 & Ronger Pan \\ \hline
	  2009/02/27 & Release V1.0.3.5 & Ronger Pan \\ \hline
	  2009/03/03 & Release V1.0.3.6 & Ronger Pan \\ \hline
	  2009/03/06 & Release V1.0.3.7 & Ronger Pan \\ \hline
	  2009/03/06 & Release V1.0.3.8 & Ronger Pan \\ \hline
	  2009/03/11 & Release V1.0.3.9 & Ronger Pan \\ \hline
	  2009/03/13 & Release V1.0.3.10 & Ronger Pan \\ \hline
   \end{tabular}

\section{Firmware V1.0.0.0}
\tlabel{sec:1-0-0}

\subsection{What's new}
\begin{itemize}
\item GIT Repository  itgserver/pub/scm/openwrt/ronger/openwrt.git
\begin{itemize}
    \item This firmware is based on Atheros LSDK 7.2.0.138.
    \item Branch: \texttt{sh\_sw\_one\_br}
    \item Tag: \texttt{WNR1000v2-V-1-0-0-0} fixed the following bugs:
\end{itemize}
\begin{enumerate}
\item Function
        \begin{itemize}
                \item \texttt{Ported Realtek 8366SR driver to linux and boot loader.}
        \end{itemize}
        \begin{itemize}
                \item \texttt{added netfilter dos and block sites patches to linux kernel}
        \end{itemize}
         \begin{itemize}
                \item \texttt{added GUI web pages for 802.11a/n}
        \end{itemize}
          \begin{itemize}
                \item \texttt{added multi language for WNR1000v2}
        \end{itemize}
           \begin{itemize}
                \item \texttt{ported new http server uhttp and its 
			      configuration management tool datalib to wnr1000v2}
        \end{itemize}
            \begin{itemize}
                \item \texttt{Ported firewall net-wall and the related 
				netfilter patches based on wnr2000}
        \end{itemize}
            \begin{itemize}
                \item \texttt{Ported firewall net-wall and the related 
				netfilter patches based on wnr2000}
        \end{itemize}
            \begin{itemize}
                \item \texttt{Ported pppoe and pptp of wnr2000 to wnr1000v2}
        \end{itemize}
            \begin{itemize}
                \item \texttt{Added dns hijack to dnsmasq}
        \end{itemize}
            \begin{itemize}
                \item \texttt{Fixed WAN/LAN conflict}
        \end{itemize}
\end{enumerate}

\subsection{Known issues}
        \begin{enumerate}
          \item \texttt{ART calibration failed on this version}
        \end{enumerate}


\section{Firmware V1.0.0.2}

\tlabel{sec:1-0-1}
\subsection{What's new}
\begin{itemize}
\item GIT Repository  itgserver/pub/scm/openwrt/ronger/openwrt.git
\begin{itemize}
    \item Branch: \texttt{sh\_sw\_one\_br}
    \item Tag: \texttt{WNR1000v2-V-1-0-0-2} fixed the following bugs:
\end{itemize}

\begin{enumerate}
\item GUI Display
        \begin{itemize}
                \item \texttt{User login and password maintain support}
        \end{itemize}
        \begin{itemize}
                \item \texttt{Modified Remote Management page according to netgear GUI spec 2.1}
        \end{itemize}
        \begin{itemize}
                \item \texttt{Modified Attached Device page according to netgear GUI spec 2.1}
        \end{itemize}
        \begin{itemize}
                \item \texttt{Modified Back up Settgings page according to netgear GUI spec 2.1}
        \end{itemize}
        \begin{itemize}
                \item \texttt{Modified Set Password page according to netgear GUI spec 2.1}
        \end{itemize}
\item Wireless
        \begin{itemize}
                \item \texttt{WDS support added.}
        \end{itemize}
	\begin{itemize}
                \item \texttt{Updated to ART tool v0\_7\_b13.1ar928xALL}
        \end{itemize}
\end{enumerate}
\end{itemize}

\subsection{Known issues}
        \begin{enumerate}
                \item \texttt{This version has been updated to Atheros LSDK 7.2.0.143.}
        \end{enumerate}

\section{Firmware V1.0.0.3}

\tlabel{sec:1-0-1}
\subsection{What's new}
\begin{itemize}
\item GIT Repository  itgserver/pub/scm/openwrt/ronger/openwrt.git
\begin{itemize}
    \item Branch: \texttt{sh\_sw\_one\_br}
    \item Tag: \texttt{WNR1000v2-V-1-0-0-3} fixed the following bugs:
\end{itemize}

\begin{enumerate}
\item GUI Display
        \begin{itemize}
                \item \texttt{Added WPA enterprise and multiple BSSID web pages.}
        \end{itemize}
\item Wireless
        \begin{itemize}
                \item \texttt{WDS support added.}
        \end{itemize}
\item Linux
	\begin{itemize}
	\item \texttt{Added ntfs support and removed nfs in kernel configuration.}
	\item \texttt{hardcoding board-specific bootargs into WNR1000v2's Linux config.}
	\item \texttt{Set argc = 0 in prom\_init() to ignore bootargs passed from u-boot.}
	\item \texttt{Enable HOTPLUG support in linux kernel.}
	 \end{itemize}
\item Application
        \begin{itemize}
                \item \texttt{Changed the default hostname to WNR1000v2 in ppp, datalib, miniupnpd,net-dump,smtpclient.}
		\item \texttt{Fix DHCP can not obtain IP address.}
		\item \texttt{Changed the MAC LOCATION according to our own flash partition table.}
		\item \texttt{Restart crond if configuring email and block sites/service.}
		\item \texttt{ntpclinet initial script.}
		\item \texttt{Smart Wizard 3.0(SOAP) supported.}
		 \item \texttt{Enable attached device function}
        \end{itemize}
        \begin{itemize}
                \item \texttt{Updated to ART tool v0\_7\_b13.1ar928xALL}
        \end{itemize}
\end{enumerate}
\end{itemize}

\subsection{Known issues}
        \begin{enumerate}
                \item \texttt{This version has been updated to Atheros LSDK 7.2.0.143.}
        \end{enumerate}

\section{Firmware V1.0.0.4}

\tlabel{sec:1-0-1}
\subsection{What's new}
\begin{itemize}
\item GIT Repository  itgserver/pub/scm/openwrt/ronger/openwrt.git
\begin{itemize}
    \item Branch: \texttt{sh\_sw\_one\_br}
    \item Tag: \texttt{WNR1000v2-V-1-0-0-4} fixed the following bugs:
\end{itemize}

\begin{enumerate}
\item Driver
        \begin{itemize}
                \item \texttt{Added code segments to reset rtl8366sr switch.}
		\item \texttt{Forbid asserting AR7161's ETH0\_RESET\_L and ETH1\_RESET\_L pins when initializing ag7100}
        \end{itemize}
\end{enumerate}
\end{itemize}

\subsection{Steps to burn boot loader and firmware}
        \begin{enumerate}
		\item \texttt{Please burn u-boot-2000V0.3.bin}
		\item \texttt{Set up a tftp server on your PC, its ip address is 192.168.1.12.}
                \item \texttt{Entering into boot loader}
		\item \texttt{ag7100> set serverip 192.168.1.12}
		\item \texttt{ag7100> tftp 0x80010000 u-boot-1000V0.3.bin;erase 0xbf000000 +0x50000;cp.b 0x80010000 0xbf000000 0x50000}
		\item \texttt{ag7100> set bootcmd 'fsload 80800000 image/uImage;bootm 80800000'}
		\item \texttt{ag7100> saveenv}
		\item \texttt{ag7100> reset}
		\item \texttt{Entering into boot loader again}
		\item \texttt{ag7100>bootm}
		\item \texttt{Then the device should be in tftp recovery mode. Please run the coammdn "tftp -i 192.168.1.1 put WNR1000v2-V1.0.0.4.img" on MS-DOS of your PC.} 
        \end{enumerate}

\subsection{Known issues}
        \begin{enumerate}
                \item \texttt{This version has been updated to Atheros LSDK 7.2.0.143.}
        \end{enumerate}

\section{Firmware V1.0.0.5}

\tlabel{sec:1-0-1}
\subsection{Repository}
\begin{itemize}
\item GIT Repository  itgserver/pub/scm/openwrt/ronger/openwrt.git
\begin{itemize}
    \item Branch: \texttt{sh\_sw\_one\_br}
    \item Tag: \texttt{WNR1000v2-V-1-0-0-5}
\end{itemize}
\end{itemize}

\subsection{New Features}
\begin{itemize}
\item As below:
\begin{enumerate}
	\item FIREWALL
        \begin{itemize}
                \item \texttt{11397 ALG suport}
                \item \texttt{11398 Port forwarding and port triggering}
                \item \texttt{11399 Adding DMZ to the firewall}
                \item \texttt{11400 NAPT Conflicts Resolution}
                \item \texttt{11402 SPI firewall/DoS Protection}
                \item \texttt{11403 Block Sites supporting}
                \item \texttt{11404 Block Service/IP Filtering}
                \item \texttt{11405 Dulplicate Blocking}
                \item \texttt{11406 Filtering schedule}
                \item \texttt{11425 Remote Web Access}
        \end{itemize}
	\item APPLICATION
	\begin{itemize}
		\item \texttt{11408 Basic routing support}
                \item \texttt{11409 Static routing support}
                \item \texttt{11410 Adding DNS client, DNS server configurations DNS relay\/Proxy.}
                \item \texttt{11412 DHCP server supporting, LAN IP assign, MAC spoofing, WAN\/LAN ip confilct}
                \item \texttt{11419 email report}
                \item \texttt{11427 NTP client porting}
                \item \texttt{11436 QoS should be implemented.}
	\end{itemize}
\end{enumerate}
\end{itemize}

\subsection{Fixed Bugs}
\begin{itemize}
\item As below:
\begin{enumerate}
	\item GUI
        \begin{itemize}
                \item \texttt{11274 Adding web pages for USB Samba according to netgear's USB storage spec}
		\item \texttt{11658 [DHCP-client]Default value of Account Name still shows WNR2000}
		\item \texttt{11683 [DNI-connection-info]there is no DHCP Server}
		\item \texttt{11746 [Block services]When you edit block services, IP address can not be edited unless you click the radio button again}
		\item \texttt{11747 [Block services]If there are spaces in Service Type/User Defined, it only saves characters forward spaces}
		\item \texttt{11752 [DHCP-client/PPPoE/PPTP/BPA]After click "Test" button,a blank testpage pop up}
		\item \texttt{11796 [BPA]Connect" and "Disconnect" button can not work in connection status page}
		\item \texttt{11798 [PPTP]"Connect" and "Disconnect" button can not work in connection status page}
		\item \texttt{11803 [PPTP]Gateway IP Address can not be saved config}
		\item \texttt{11804 [PPTP]If user leave blank for the field of My IP Address, this Gateway IP Address field SHOULD be grayed out.}
		\item \texttt{11824 [Setup wizard-Static]After detect WAN as Static IP (Fixed) Addresses,please check IP Address,Subnet mask,Gateway IP address,DNS address.}
		\item \texttt{11825 [Setup wizard-PPPoE]You select "Get Dynamically From ISP" in WIZ\_pppoe.htm,but it shows "Use Static IP Address" in Basic setting page}
		\item \texttt{11827 [GUI]QOS\_main.html "wmm" and "turn on qos" value can not be set}
		\item \texttt{11837 [Setup wizard-PPTP]Idle Timeout (In minutes),Connection ID/Name and My IP Address can not be saved from WIZ\_pptp.htm}
		\item \texttt{11838 [GUI]QOS\_main.html "wmm" and "turn on qos" value can not be set}
		\item \texttt{11839 [Setup wizard-BPA]We can not dial on BPA form setup wizard because idle time and Authentication Server can not be saved config in WIZ\_bpa.htm}
		\item \texttt{11841 [Setup wizard-DHCP]Account name and domain name can not be saved config in WIZ\_dhcp.htm page}
		\item \texttt{11843 [Setup wizard-Static]After detect WAN as Static IP (Fixed) Addresses,static IP address can not be saved cofing.}
		\item \texttt{11845 [Seup wizard-PPPoE]You select "Get Dynamically From ISP" in WIZ\_pppoe.htm,but static IP address can not be saved config}
		\item \texttt{11929 Two "RADIUS server Port" in WPA-enterprise}
		\item \texttt{11848 [Static route]When you add static routes,they do not show in routing table}
		\item \texttt{11390 wireless settings with WPA enterprise and Multiple BSSID supported}
		\item \texttt{11317 Modify web pages for netgear's auto test according to netgear GUI spec 2.1}
		\item \texttt{11467 Wireless Advanced Setting Enable SSID Broadcast can't work}
		\item \texttt{11542 Modified "Router Status" page according to netgear GUI spec 2.1}
		\item \texttt{11771 [DHCP-client]When I paste strange stirng as Account name and Domain name,the webpage shows error.}
		\item \texttt{11772 [DHCP-client]Account name and domain name should not accept 2 bytes characters.}
		\item \texttt{11774 [DHCP-client]Default MAC should not be change.}
		\item \texttt{11776 [DHCP-client]Please remove the redundant right brackets in the webpage.}
		\item \texttt{11789 [BPA]Default MAC should not be change.}
		\item \texttt{11799 [PPTP]Default MAC should not be change.}
		\item \texttt{11801 [PPTP]Click the link of "Gateway IP Address" in the webpage will call the help frame become blank.}
		\item \texttt{11809 [Setup wizard]When Detecting Internet connection,there is a warning message pop up every time.}
	\end{itemize}
	\item APPLICATION
	\begin{itemize}
		\item \texttt{11284 Porting ART tools to the buildroot.}
		\item \texttt{11769 [DHCP-client]Whenever IP assignment, router should automatically flush the static route (should only flush the WAN routing)}
        \end{itemize}
        \item LINUX KERNEL
        \begin{itemize}
		\item \texttt{11886 [SIP]Contact URIs show in SIP server are error}
                \item \texttt{11887 [SIP]LAN PCs can not talk to each other for a long time.}
		\item \texttt{11392 Behavioral Requirements for Unicast UDP according to the chapter 1.1 in netgear spec 1.6}
                \item \texttt{11394 Behavorial Requirements for TCP and ICMP}
        \end{itemize}
\end{enumerate}
\end{itemize}

\subsection{Steps to burn boot loader and firmware}
\begin{itemize}
\item As below:
        \begin{enumerate}
                \item \texttt{Please burn u-boot-1000V0.3.bin}
                \item \texttt{Set up a tftp server on your PC, its ip address is 192.168.1.12.}
                \item \texttt{Entering into boot loader}
                \item \texttt{ag7100> set serverip 192.168.1.12}
                \item \texttt{ag7100> tftp 0x80010000 u-boot-1000V0.3.bin;erase 0xbf000000 +0x50000;cp.b 0x80010000 0xbf000000 0x50000}
                \item \texttt{ag7100> set bootcmd 'fsload 80800000 image/uImage;bootm 80800000'}
                \item \texttt{ag7100> saveenv}
                \item \texttt{ag7100> reset}
                \item \texttt{Entering into boot loader again}
                \item \texttt{ag7100>bootm}
                \item \texttt{Then the device should be in tftp recovery mode. Please run the coammdn "tftp -i 192.168.1.1 put WNR1000v2-V1.0.0.4.img" on MS-DOS of your PC.}
        \end{enumerate}
\end{itemize}

\subsection{Known issues}
        \begin{enumerate}
                \item \texttt{This version has been updated to Atheros LSDK 7.2.0.143.}
        \end{enumerate}

\section{Firmware V1.0.0.6}

\tlabel{sec:1-0-1}
\subsection{Repository}
\begin{itemize}
\item GIT Repository  itgserver/pub/scm/openwrt/ronger/openwrt.git
\begin{itemize}
    \item Branch: \texttt{sh\_sw\_one\_br}
    \item Tag: \texttt{WNR1000v2-V-1-0-0-6}
\end{itemize}
\end{itemize}

\subsection{New Features}
\begin{itemize}
\item As below:
\begin{enumerate}
	\item APPLICATION
	\begin{itemize}
		
                \item \texttt{11430  POT support}
	\end{itemize}
\end{enumerate}
\end{itemize}

\subsection{Fixed Bugs}
\begin{itemize}
\item As below:
\begin{enumerate}
	\item GUI
        \begin{itemize}
		\item \texttt{11428  Router status staistics  }
		\item \texttt{11753  [DHCP-client]"Release" and "Renew" button can not work in connection status page  }
		\item \texttt{11971  channel has some problems in b/g mode and a mode}  
		\item \texttt{11972  wireless network mode is not match the spec 1.6  }
		\item \texttt{11862  [RIP]RIP does not work  }
		\item \texttt{11968  [SOAP]Run CD every time,the DUT reboot and soap test can not go through.  }
		\item \texttt{11773  [DHCP-client]Please check MAC address when you select "Use This MAC Address"}  
		\item \texttt{11775  [DHCP-client]Please check IP address,Subnet mask,Gateway IP address,DNS IP address when you select Use Static IP Address and Use these DNS servers  }

		\item \texttt{11805  [PPTP]According Spec1.6,Additional Help description.  }
		\item \texttt{11835  [Setup wizard-PPTP]Please remove the Superfluous ">" in WIZ\_pptp.htm page  }
		\item \texttt{11836  [Setup wizard-PPTP]Please check My IP address,idle time in WIZ\_pptp.htm page}  
		\item \texttt{11840  [Setup wizard-bpa]Please check idle time in WIZ\_bpa.htm  }
		\item \texttt{11842  [Setup wizard-DHCP]Please check Account name and domain name in WIZ\_dhcp.htm page  }
		\item \texttt{11872  [Port forwarding]Port forwarding page shows error when you input some special characters as server name  }
	\end{itemize}
	\item APPLICATION
	\begin{itemize}
		\item \texttt{11871  [LAN Setup]LAN IP can not be changed unless you reboot the DUT  }
        \end{itemize}
\end{enumerate}
\end{itemize}

\subsection{Steps to burn boot loader and firmware}
\begin{itemize}
\item As below:
        \begin{enumerate}
                \item \texttt{Please burn u-boot-1000V0.3.bin}
                \item \texttt{Set up a tftp server on your PC, its ip address is 192.168.1.12.}
                \item \texttt{Entering into boot loader}
                \item \texttt{ag7100> set serverip 192.168.1.12}
                \item \texttt{ag7100> tftp 0x80010000 u-boot-1000V0.3.bin;erase 0xbf000000 +0x50000;cp.b 0x80010000 0xbf000000 0x50000}
                \item \texttt{ag7100> set bootcmd 'fsload 80800000 image/uImage;bootm 80800000'}
                \item \texttt{ag7100> saveenv}
                \item \texttt{ag7100> reset}
                \item \texttt{Entering into boot loader again}
                \item \texttt{ag7100>bootm}
                \item \texttt{Then the device should be in tftp recovery mode. Please run the coammdn "tftp -i 192.168.1.1 put WNR1000v2-V1.0.0.4.img" on MS-DOS of your PC.}
        \end{enumerate}
\end{itemize}

\subsection{Known issues}
        \begin{enumerate}
                \item \texttt{This version has been updated to Atheros LSDK 7.2.0.143.}
				\item \texttt{This release has not been tested by ST.}
        \end{enumerate}

		
\section{Firmware V1.0.0.7}

\tlabel{sec:1-0-1}
\subsection{Repository}
\begin{itemize}
\item GIT Repository  itgserver/pub/scm/openwrt/ronger/openwrt.git
\begin{itemize}
    \item Branch: \texttt{sh\_sw\_one\_br}
    \item Tag: \texttt{WNR1000v2-V-1-0-0-7}
\end{itemize}
\end{itemize}

\subsection{New Features}
\begin{itemize}
\item As below:
\begin{enumerate}
	\item APPLICATION
	\begin{itemize}
	\item \texttt{11862 [RIP]RIP does not work}
	\item \texttt{11431 Power LED indication support}
	\end{itemize}
\end{enumerate}
\end{itemize}

\subsection{Steps to burn boot loader and firmware}
\begin{itemize}
\item As below:
        \begin{enumerate}
                \item \texttt{Please burn u-boot-1000V0.4.bin}
                \item \texttt{Set up a tftp server on your PC, its ip address is 192.168.1.12.}
                \item \texttt{Entering into boot loader}
                \item \texttt{ag7100> set serverip 192.168.1.12}
                \item \texttt{ag7100> tftp 0x80010000 u-boot-1000V0.3.bin}
				\item \texttt{ag7100> erase 0xbf000000 +0x50000}
				\item \texttt{ag7100> cp.b 0x80010000 0xbf000000 0x50000}
                \item \texttt{ag7100> set bootcmd 'fsload 80800000 image/uImage;bootm 80800000'}
                \item \texttt{ag7100> saveenv}
                \item \texttt{ag7100> reset}
                \item \texttt{Entering into boot loader again}
                \item \texttt{ag7100>bootm}
                \item \texttt{Then the device should be in tftp recovery mode. Please run the coammand "tftp -i 192.168.1.1 put WNR1000v2-V1.0.0.7.img" on MS-DOS of your PC.}
        \end{enumerate}
\end{itemize}

\subsection{Known issues}
        \begin{enumerate}
                \item \texttt{This version has been updated to Atheros LSDK 7.2.0.143.}
        \end{enumerate}


\section{Firmware V1.0.0.8}

\tlabel{sec:1-0-1}
\subsection{Repository}
\begin{itemize}
\item GIT Repository  itgserver/pub/scm/openwrt/ronger/openwrt.git
\begin{itemize}
    \item Branch: \texttt{sh\_sw\_one\_br}
    \item Tag: \texttt{WNR1000v2-V-1-0-0-8}
\end{itemize}
\end{itemize}

\subsection{New Features}
\begin{itemize}
\item As below:
\begin{enumerate}
	\item APPLICATION
	\begin{itemize}
                \item \texttt{11441 Intel ViiV should be supported}
				\item \texttt{11433 	 	Internet connection LED indication}
				\item \texttt{11434 	 	LED of Reset to facotry default}
				\item \texttt{11435 	 	Firmware upgrading LED}
				\item \texttt{11934 	 	[DHCP-Server]Address Reservation can not work}
				\item \texttt{12012 	 	[Log]The button "send log" can not work}
				\item \texttt{12047 	 	Reset button cannot work}
				\item \texttt{12049 	 	SOAP should be supported for 5G wireless}
				\item \texttt{11889 	 	Ethernet port should be supported in QoS}
				\item \texttt{12045 	 	Should add green packet feature in switch driver}
	\end{itemize}
\end{enumerate}
\begin{enumerate}
	\item GUI
	\begin{itemize}
				\item \texttt{12007 	 	Should add a web page for a/n in advanced wireless settings}
				\item \texttt{12171 	 	According to netgear spec2.1, add hotkeys in Wireless Settings GUI page}
	\end{itemize}
\end{enumerate}
\end{itemize}

\subsection{Fixed Bugs}
\begin{itemize}
\item As below:
\begin{enumerate}
	\item GUI
        \begin{itemize}
		\item \texttt{11753 	 	[DHCP-client]"Release" and "Renew" button can not work in connection status page}
		\item \texttt{11826 	 	[PPPoE]Connection Mode of PPPoE can not be changed to Always on or Manually connect.}
		\item \texttt{11393 	 	NAT filtering GUI according to the chapter 1.2.1 in netgear spec 1.6}
		\item \texttt{12117 	 	[Router status]Remove the Superfluous ) on router status page}
		\item \texttt{11574 	 	GUI should support WPA enterprise and multiple BSSID}
		\item \texttt{11747 	 	[Block services]If there are spaces in Service Type/User Defined, it only saves characters forward spaces}
		\item \texttt{11752 	 	[DHCP-client/PPPoE/PPTP/BPA]After click "Test" button,a blank testpage pop up}
		\item \texttt{11754 	 	[DHCP-client]"Lease Obtained" and "Lease Expires" do not show in conncection status page}
		\item \texttt{11755 	 	[DHCP-client]DHCP Server does not shows in connection status page}
		\item \texttt{11758 	 	[DHCP-client]For the cases of Ethernet cable plug-off,IP Address,Subnet mask,Gateway IP address,DNS IP address and Domain Name Server in Basic settin Router status page should become 0.0.0.0g page and}
		\item \texttt{11811 	 	[PPPoE]"Connect" and "Disconnect" button can not work in connection status page}
		\item \texttt{11827 	 	[GUI]QOS\_main.html "wmm" and "turn on qos" value can not be set}
		\item \texttt{11837 	 	[Setup wizard-PPTP]Idle Timeout (In minutes),Connection ID/Name and My IP Address can not be saved from WIZ\_pptp.htm}
		\item \texttt{11838 	 	[GUI]QOS\_main.html "wmm" and "turn on qos" value can not be set}
		\item \texttt{11839 	 	[Setup wizard-BPA]We can not dial on BPA form setup wizard because idle time and Authentication Server can not be saved config in WIZ\_bpa.htm}
		\item \texttt{11841 	 	[Setup wizard-DHCP]Account name and domain name can not be saved config in WIZ\_dhcp.htm page}
		\item \texttt{11843 	 	[Setup wizard-Static]After detect WAN as Static IP (Fixed) Addresses,static IP address can not be saved cofing.}
		\item \texttt{11845 	 	[Seup wizard-PPPoE]You select "Get Dynamically From ISP" in WIZ\_pppoe.htm,but static IP address can not be saved config}
		\item \texttt{11929 	 	Two "RADIUS server Port" in WPA-enterprise}
		\item \texttt{11935 	 	[DHCP-Server]DHCP Server can not be disabled}
		\item \texttt{11937 	 	[WAN Setup]If you change WAN MTU and click apply,the webpage go back to Basic settings page}
		\item \texttt{11942 	 	[Backup Settings]After restore config file,the webpage can not go back because the time in webpage is not enough.}
		\item \texttt{11945 	 	[Router Status]Hardware version does not show on router status page}
		\item \texttt{11946 	 	[Statistic]"Show statistic" page does not work}
		\item \texttt{11947 	 	[Intel VIIV]"http://www.routerlogin.com/router-info.htm" page is blank}
		\item \texttt{11971 	 	channel has some problems in b/g mode and a mode}
		\item \texttt{11996 	 	[Email]Your Outgoing Mail Server should not allow 2 bytes character}
		\item \texttt{11998 	 	[Schedule]Please check Start Blocking and End Blocking}
		\item \texttt{12008 	 	[DNI-wireless]wireless a/n and wireless b/g/n should support "WPA/WPA2 Enterprise"}
		\item \texttt{12011 	 	[Schedule]We should not select "All Day" when "Days to Block" is not select}
		\item \texttt{12033 	 	The wireless GUI setting operation time doesn't syn. up}
		\item \texttt{12067 		[Basic Wireless settings][a/n][b/g/n]According to Netgear spec1.6,the DUT should allowed any to be a SSID if it support WPS}
		\item \texttt{12073 	 	[Basic Wireless settings][a/n][b/g/n]When SSID is </script> , Basic wireless settings page and router status page shows error}
		\item \texttt{12095 	 	[Advanced Wireless settings][a/n]WPS Settings in Advanced Wireless Settings a/n page is not grayed out when Security is Auto/Shared WEP.}
		\item \texttt{12102 	 	[Basic Wireless settings][a/n][b/g/n][Guest Network][WPA/WPA2 Enterprise]RADIUS server Port should only allow number.}
		\item \texttt{12103 	 	[Basic Wireless settings][a/n][b/g/n][Guest Network][WPA/WPA2 Enterprise]Warning message for RADIUS server Port is error}
		\item \texttt{12104 	 	[Basic Wireless settings][a/n][b/g/n][Guest Network][WPA/WPA2 Enterprise]RADIUS server Shared Secret should not allow 2 bytes characters}
		\item \texttt{12110 	 	[Basic Wireless settings][a/n][b/g/n]Webpage return before wireless function is OK.}
		\item \texttt{12142 	 	[Guest Network][b/g/n]If you input many spaces in Guest Wireless Network Name(SSID),it shows only one space in Network Profiles}
		\item \texttt{12143 	 	[Guest Network][b/g/n]The webpage shows error when you input some special characters as SSID}
		\item \texttt{12034 	 	Wrong command in the apup}
		\item \texttt{11444 	 	Modified "WAN Setup" page according to netgear GUI spec 2.1}
		\item \texttt{11445 	 	Modified "UPnP" page according to netgear GUI spec 2.1}
		\item \texttt{11446 	 	Modified "LAN Setup" page according to netgear GUI spec 2.1}
		\item \texttt{11447 	 	Modified "Port Forwarding / Port Triggering" page according to netgear GUI spec 2.1}
		\item \texttt{11452 	 	Modified "Dynamic DNS" page according to netgear GUI spec 2.1}
		\item \texttt{11453 	 	Modified "Static Routes" page according to netgear GUI spec 2.1}
		\item \texttt{11455 	 	Modified "Wireless Repeating Function" page according to netgear GUI spec 2.1}
		\item \texttt{11467 	 	Wireless Advanced Setting Enable SSID Broadcast can't work}
		\item \texttt{11541 	 	Modified "Router Upgrade" page according to netgear GUI spec 2.1}
		\item \texttt{11542 	 	Modified "Router Status" page according to netgear GUI spec 2.1}
		\item \texttt{11546 	 	Modified "Attached Devices" page according to netgear GUI spec 2.1}
		\item \texttt{11547 	 	Modified "Backup Settings" page according to netgear GUI spec 2.1}
		\item \texttt{11548 	 	Modified "Set Password" page according to netgear GUI spec 2.1}
		\item \texttt{11790 	 	[BPA]Please check MAC address when you select "Use This MAC Address"}
		\item \texttt{11793 	 	[BPA]Please check DNS IP address when you select Use these DNS servers}
		\item \texttt{11801 	 	[PPTP]Click the link of "Gateway IP Address" in the webpage will call the help frame become blank}
		\item \texttt{11836 	 	[Setup wizard-PPTP]Please check My IP address,idle time in WIZ\_pptp.htm page}
		\item \texttt{11840 	 	[Setup wizard-bpa]Please check idle time in WIZ\_bpa.htm}
		\item \texttt{11842 	 	[Setup wizard-DHCP]Please check Account name and domain name in WIZ\_dhcp.htm page}
		\item \texttt{11872 	 	[Port forwarding]Port forwarding page shows error when you input some special characters as server name}
		\item \texttt{11921 	 	[Wiress settings a/n]Channels shows error in Wiress settings a/n page}
		\item \texttt{12197 	 	[SQA Taipei-17][LAN Setup]The default device name on LAN setup page is NULL.}
	\end{itemize}
	\item APPLICATION
	\begin{itemize}
		\item \texttt{11750 	 	[DHCP-client]There DUT should use ARP to detect conflict with the assigned IP address after DHCPACK is received.}
		\item \texttt{11761 	 	[DHCP-client]On T1=50\% and T2=87.5\% leased time, should send DHCP\_REQUEST to reusing the current IP address.}
		\item \texttt{11763 	 	[DHCP-client]If the whole DHCP request procedure is failed, router should restart the procedure every 5 minutes.}
		\item \texttt{11795 	 	[BPA]Password can not be changed in /tmp/bpalogin.conf}
		\item \texttt{11810 	 	[Block sites/Block services]When I select "Per Schedule",Block sites and Block services can not work}
		\item \texttt{11823 	 	[Setup wizard]According to WPN824v3,WNR2000,after DHCP, you need detect PPTP again}
		\item \texttt{11949 	 	[NAT][WAN/LAN Conflict]Static WAN Subnet is a subnet of LAN subnet,WAN/LAN conflict fail.}
		\item \texttt{11950 	 	[NAT][WAN/LAN Conflict]LAN subnet is a subnet of Static WAN Subnet,WAN/LAN conflict fail.}
		\item \texttt{11951 	 	[NAT][WAN/LAN Conflict]LAN subnet is as same as Static WAN Subnet,WAN/LAN conflict fail.}
		\item \texttt{11955 	 	[NAT][WAN/LAN Conflict]When WAN/LAN conflict,LAN PC should auto get new IP address.}
		\item \texttt{11956 	 	[NAT][WAN/LAN Conflict]There is no conflict detection on static PPPoE mode}
		\item \texttt{11957 	 	[NAT][WAN/LAN conflict]There is no conflict detection on static PPTP mode}
		\item \texttt{11958 	 	[NAT][WAN/LAN conflict]There is no conflict detection on static BPA mode}
		\item \texttt{11886 	 	[SIP]Contact URIs show in SIP server are error}
		\item \texttt{12188 	 	[qos] when reboot the board, qos does not takce effect}
		\item \texttt{11644 	 	DHCP can't obtain IP address when booting the DUT first time}
		\item \texttt{11461 	 	Ethernet PHY port should support pull-low and pull-high action}
		\item \texttt{11941 	 	[Logs]If the DUT can not get Current NTP time,there is no log in logs page}
		\item \texttt{12187 	 	Power LED changes to amber when erasing MTD}
		\item \texttt{11417 	 	User login and password maintain support}
		\item \texttt{11764 	 	MAC address of wifi0 and wifi1 always are 00:00:00:00:00:00}
		\item \texttt{11973 	 	country setting ioctl can't work }
        \end{itemize}
\end{enumerate}
\end{itemize}

\subsection{Steps to burn boot loader and firmware}
\begin{itemize}
\item As below:
        \begin{enumerate}
                \item \texttt{Please burn u-boot-1000V0.4.bin}
                \item \texttt{Set up a tftp server on your PC, its ip address is 192.168.1.12.}
                \item \texttt{Entering into boot loader}
                \item \texttt{ag7100> set serverip 192.168.1.12}
                \item \texttt{ag7100> tftp 0x80010000 u-boot-1000V0.4.bin}
		\item \texttt{ag7100> erase 0xbf000000 +0x50000}
		\item \texttt{ag7100> cp.b 0x80010000 0xbf000000 0x50000}
                \item \texttt{ag7100> set bootcmd 'fsload 80800000 image/uImage;bootm 80800000'}
                \item \texttt{ag7100> saveenv}
                \item \texttt{ag7100> reset}
                \item \texttt{Entering into boot loader again}
                \item \texttt{ag7100>bootm}
                \item \texttt{Then the device should be in tftp recovery mode. Please run the coammdn "tftp -i 192.168.1.1 put WNR1000v2-V1.0.0.8.img" on MS-DOS of your PC.}
        \end{enumerate}
\end{itemize}

\subsection{Known issues}
        \begin{enumerate}
                \item \texttt{This version has been updated to Atheros LSDK 7.2.0.143.}
        \end{enumerate}	
		
\section{Firmware V1.0.0.9}

\tlabel{sec:1-0-1}
\subsection{Repository}
\begin{itemize}
\item GIT Repository  itgserver/pub/scm/openwrt/ronger/openwrt.git
\begin{itemize}
    \item Branch: \texttt{sh\_sw\_one\_br}
    \item Tag: \texttt{WNR1000v2-V-1-0-0-9}
\end{itemize}
\end{itemize}

\subsection{Fixed Bugs}
\begin{itemize}
\item As below:
\begin{enumerate}
\item Bug list:
\begin{itemize}
	\item \texttt{11441 Intel ViiV should be supported}
	\item \texttt{11747 [Block services]If there are spaces in Service Type/User Defined, it only saves characters forward spaces}
	\item \texttt{11753 [DHCP-client]"Release" and "Renew" button can not work in connection status page}
	\item \texttt{11758 [DHCP-client]For the cases of Ethernet cable plug-off,IP Address,Subnet mask,Gateway IP address,DNS IP address and Domain Name Server in Basic setting page and Router status page should become 0.0.0.0}
	\item \texttt{11811 [PPPoE]"Connect" and "Disconnect" button can not work in connection status page}
	\item \texttt{11826 [PPPoE]Connection Mode of PPPoE can not be changed to Always on or Manually connect.}
	\item \texttt{11873 [Top]Please change the model name to WNR1000v2 in top frame's picture}
	\item \texttt{11935 [DHCP-Server]DHCP Server can not be disabled}
	\item \texttt{11937 [WAN Setup]If you change WAN MTU and click apply,the webpage go back to Basic settings page}
	\item \texttt{11942 [Backup Settings]After restore config file,the webpage can not go back because the time in webpage is not enough.}
	\item \texttt{11947 [Intel VIIV]"http://www.routerlogin.com/router-info.htm" page is blank}
	\item \texttt{11972 wireless network mode is not match the spec 1.6}
	\item \texttt{11998 [Schedule]Please check Start Blocking and End Blocking}
	\item \texttt{12086 [Basic wireless settings][a/n]When "Up to 300Mbps" is selected, router status does not show double channels}
	\item \texttt{12098 [Basic Wireless settings][a/n][b/g/n][Guest Network][WPA-PSK][WPA2-PSK][WPA/WPA2-PSK]If there is a space in WPA-PSK PassPhrase, the webpage shows error.}
	\item \texttt{12101 [Basic Wireless settings][a/n][b/g/n][Guest Network][WPA-PSK]If I set WPA-PSK PassPhrase as </script>,the webpage shows error}
	\item \texttt{12136 [Guest Network]Please add a Guest Network webpage for a/n}
	\item \texttt{12179 Merge 2 band into one page for basic wireless setting and advanced wireless setting}
	\item \texttt{12195 [SQA Taipei-15][Multi-language]Login to DUT always must select language.}
	\item \texttt{12209 [SQA Taipei-30]The default region is " Select Region" on both wireless setting a/n and wireless setting b/g/n pages, but the status page shows "Europe".}
	\item \texttt{12261 [Storage Administration]When you edit "guest",Group Member shows error in UI.}
	\item \texttt{12282 wireless setting a/n page should show correct channel list}
	\item \texttt{12299 [Disk Management--Shared Folder]There is a javascript error when you click ">>" or "<<" button and no one in "Available Groups" is selected}
	\item \texttt{12300 [Disk Management--Shared Folder]Location shows error in edit mode}
	\item \texttt{12308 [Disk Management--Shared Folder]Shared Folder shows error if there is a warning messages before it saves config.}
	\item \texttt{12311 [Disk Management--Shared Folder]Shared Folder should not begin with //}
	\item \texttt{12317 [Disk Management--Shared Folder]After you edit the Shared Folder list, shared folder become blank in the table}
	\item \texttt{12321 [Disk Management]Disk Management page shows error if you input some special characters as Display name}
	\item \texttt{12322 [Disk Management--Shared Folder]Assigned to Folder field should not be blank if you want to add a folder to shared folder list.}
	\item \texttt{12324 [Disk Management]There should be only one : after "Shared Storage IP Address"}
	\item \texttt{12326 [Disk Management]There should be a : after "Disk"}
	\item \texttt{12338 [Storage Administration]When I click "delete" button in User Accounts list,there is a warning pop up}
	\item \texttt{12340 [Storage Administration-User Accounts]User name must be unique}
	\item \texttt{12344 [Storage Administration-Group Account]If there is a space in description,the webpage shows error.}
	\item \texttt{12345 [Storage Administration-Group Account]Group account and Description should allow special characters}
	\item \texttt{12347 [Storage Administration-User Account]If there is a space in full name or Description,the webpage shows error.}
	\item \texttt{12348 [Storage Administration-Group Account]User name,full name and Description should allow special characters}
	\item \texttt{12349 [Storage Administration-Group Account]Group account and Description should not allow 2bytes characters}
	\item \texttt{12368 There is no "select region" in the region list, the default value of region is NOT "select region". Please refer to router spec.}
	\item \texttt{12422 After doing reset to default, it can't open "routerlogin" page automatically.}
	\item \texttt{12475 Display correct sharing folder in the windows and save sharing folder records.}
	\item \texttt{12434 USB bug--After you add second or more shared folders, then the shared folder page in the "Disk management" page should be refreshed automatically.}
	\item \texttt{12476 The Location of shared folder dialog in USB disk management should display currnent folder location not nothing.}
	\item \texttt{11435 Firmware upgrading LED}
	\item \texttt{11737 [Block sites]NNTP can not be blocked when they contain the keyword.}
	\item \texttt{11750 [DHCP-client]There DUT should use ARP to detect conflict with the assigned IP address after DHCPACK is received.}
	\item \texttt{11767 [DHCP-client]For the cases of Ethernet cable plug-off and plug-in, client should send DHCP\_REQUEST and include current IP address as preferred client IP address.}
	\item \texttt{11769 [DHCP-client]Whenever IP assignment, router should automatically flush the static route (should only flush the WAN routing)}
	\item \texttt{11795 [BPA]Password can not be changed in /tmp/bpalogin.conf}
	\item \texttt{11823 [Setup wizard]According to WPN824v3,WNR2000,after DHCP, you need detect PPTP again}
	\item \texttt{11880 [NAT]NAT Mapping Timer test fail}
	\item \texttt{11881 [NAT]"Outbound UDP Packets with Destination Port �R 1024" test fail}
	\item \texttt{11934 [DHCP-Server]Address Reservation can not work}
	\item \texttt{11941 [Logs]If the DUT can not get Current NTP time,there is no log in logs page}
	\item \texttt{11940 [NTP]NTP time can not get the Current time unless you reboot the DUT if you do not connect the DUT to internet first.}
	\item \texttt{11949 [NAT][WAN/LAN Conflict]Static WAN Subnet is a subnet of LAN subnet,WAN/LAN conflict fail.}
	\item \texttt{11950 [NAT][WAN/LAN Conflict]LAN subnet is a subnet of Static WAN Subnet,WAN/LAN conflict fail.}
	\item \texttt{11951 [NAT][WAN/LAN Conflict]LAN subnet is as same as Static WAN Subnet,WAN/LAN conflict fail.}
	\item \texttt{11953 [NAT][WAN/LAN conflict]When the LAN-WAN IP address conflict, the web page of DUT should give a tip of IP address conflict}
	\item \texttt{11955 [NAT][WAN/LAN Conflict]When WAN/LAN conflict,LAN PC should auto get new IP address.}
	\item \texttt{11956 [NAT][WAN/LAN Conflict]There is no conflict detection on static PPPoE mode}
	\item \texttt{11957 [NAT][WAN/LAN conflict]There is no conflict detection on static PPTP mode}
	\item \texttt{11958 [NAT][WAN/LAN conflict]There is no conflict detection on static BPA mode}
	\item \texttt{11968 [SOAP]Run CD every time,the DUT reboot and soap test can not go through.}
	\item \texttt{12012 [Log]The button "send log" can not work}
	\item \texttt{12015 [NAT]4.5. Concurrent TCP Outbound Packets (Skype 3.2.9) fail}
	\item \texttt{12019 [DHCP-client]According to Netgear Spec 1.6: It is recommended to send a DHCP RELEASE message to the server under the case software shutdown or reboot.}
	\item \texttt{12122 Before sending discover packet, should check whether WAN cable is plugged.}
	\item \texttt{12152 [Port Triggering]The port trigger could not use the same outgoing port to trigger different port rules.}
	\item \texttt{12158 [SQA Taipei-7][Setup wizard]Wizard cannot detect bigpond mode.}
	\item \texttt{12191 [SQA Taipei-10][LED]When WAN connection is DHCP mode, the LED is amber.}
	\item \texttt{12182 [SQA Taipei-2][LED]Power LED is amber, it should be green.}
	\item \texttt{12183 [SQA Taipei-3][LED]LAN link to 100Mbps, the amber and green LEDs turns on at the same time.}
	\item \texttt{12184 [SQA Taipei-4][Setup wizard]The account name is NULL when using wizard to configure DHCP mode.}
	\item \texttt{12185 [SQA Taipei-5][PPTP]PPTP idle time function is fail.}
	\item \texttt{12186 [SQA Taipei-6][Setup wizard]If WAN port cable not plugged in and we use wizard to detect WAN port type,the detected is "Static IP". It should display failure page.}
	\item \texttt{12187 Power LED changes to amber when erasing MTD}
	\item \texttt{12192 [SQA Taipei-12][LED]Bigpond mode not connected to internet, the WAN LED is green.}
	\item \texttt{12194 [SQA Taipei-14][LED]When WAN port is static IP the WAN LED is amber.}
	\item \texttt{12196 [SQA Taipei-16][LED]When WAN port IP is released, the WAN LED is still green. It should be amber}
	\item \texttt{12204 [SQA Taipei-24]There is no reset button on the PCB. (Hardware version 2976326600(B).)}
	\item \texttt{12206 [SQA Taipei-26][CDRouter]CDRouter port trigger function test fail.}
	\item \texttt{12207 [SQA Taipei-27][CDRouter]CDRouter PPTP pass-through test fail.}
	\item \texttt{12214 According to WNR1000, Newly added port triggering rule should be enabled by default}
	\item \texttt{12252 [Setup wizard]Sometimes setup wizard can not go through,it always refresh the page "Detecting Connection Type on Internet Port"}
	\item \texttt{12366 Before sending NTP packet, the WAN connection must be connected.}
	\item \texttt{12405 [SQA Taipei-35]Unable to open the "Wireless a/n" / "Wireless b/g/n" / "Guest Network" /"schedule" / "port forwarding" pages when we login DUT remotely.}
	\item \texttt{12068 [Basic Wireless settings][a/n][b/g/n]If there is a space in SSID,the SSID shows OK in UI but error in iwconfig}
	\item \texttt{12097 [Basic Wireless settings][a/n][b/g/n][Guest Network][WPA-PSK][WPA2-PSK][WPA/WPA2-PSK]Wireless client WNDA3100 can not connect to the DUT when security is WPA-PSK,WPA2-PSK,WPA/WPA2-PSK}
	\item \texttt{12099 [Basic Wireless settings][a/n][b/g/n][Guest Network][WPA-PSK][WPA2-PSK][WPA/WPA2-PSK]If I set WPA-PSK PassPhrase as 64 hexdigits, it save as no security in function}
	\item \texttt{12109 [Basic Wireless settings][a/n][b/g/n][Guest Network][WPA/WPA2 Enterprise]WPA/WPA2 Enterprise can not be tested because wireless client can not connect to the DUT when security is WPA-PSK,WPA2-PSK,WPA/WPA2-PSK.}
	\item \texttt{12112 [Advanced Wireless settings][a/n][b/g/n]"Enable Wireless Router Radio" in Advanced wireless settings page should not turn on and off both a/n and b/g/n.}
	\item \texttt{12114 [Advanced Wireless settings][a/n][b/g/n]"Disable SSID Broadcast" does not work}
	\item \texttt{12137 [Guest Network][b/g/n]"Disable SSID Broadcast" does not work on Guest Network page}
	\item \texttt{12157 [Advanced Wireless settings][a/n][b/g/n]Fragmentation does not work.}
	\item \texttt{12236 Should adjust the channel list under 300M mode of a band}
	\item \texttt{12265 WPS LED should be controlled by software}
	\item \texttt{12385 [WPS]WPS icon under Vista still shows WNR2000}
	\item \texttt{11317 Modify web pages for netgear's auto test according to netgear GUI spec 2.1}
	\item \texttt{11444 Modified "WAN Setup" page according to netgear GUI spec 2.1}
	\item \texttt{11446 Modified "LAN Setup" page according to netgear GUI spec 2.1}
	\item \texttt{11447 Modified "Port Forwarding / Port Triggering" page according to netgear GUI spec 2.1}
	\item \texttt{11451 Modified "QoS Setup" page according to netgear GUI spec 2.1}
	\item \texttt{11452 Modified "Dynamic DNS" page according to netgear GUI spec 2.1}
	\item \texttt{11454 Modified "Remote Management" page according to netgear GUI spec 2.1}
	\item \texttt{11456 Modified "Wireless Settings" page according to netgear GUI spec 2.1}
	\item \texttt{11541 Modified "Router Upgrade" page according to netgear GUI spec 2.1}
	\item \texttt{11542 Modified "Router Status" page according to netgear GUI spec 2.1}
	\item \texttt{11547 Modified "Backup Settings" page according to netgear GUI spec 2.1}
	\item \texttt{11790 [BPA]Please check MAC address when you select "Use This MAC Address"}
	\item \texttt{11793 [BPA]Please check DNS IP address when you select Use these DNS servers}
	\item \texttt{11801 [PPTP]Click the link of "Gateway IP Address" in the webpage will call the help frame become blank}
	\item \texttt{11840 [Setup wizard-bpa]Please check idle time in WIZ\_bpa.htm}
	\item \texttt{11872 [Port forwarding]Port forwarding page shows error when you input some special characters as server name}
	\item \texttt{11918 [Port triggering]Port Triggering Timeout(in minutes) should only accept numbers}
	\item \texttt{11919 [Port triggering]Port triggering page shows error when you input some special characters as server name}
	\item \texttt{11920 [Port tringgering]Although the warning messages says the port must be in the range [1-65535] excluding 80,I can input 80 port}
	\item \texttt{11927 [DMZ]If LAN IP address changes,DMZ IP address shows error}
	\item \texttt{11932 [NAPT]The rule should be added when there exists a temporal rule using the same public port and mapping to the same LAN IP and private/internal port}
	\item \texttt{11936 [DHCP Server]LAN Setup page shows error when you input some special characters as device name}
	\item \texttt{12056 Add auto radio test button for wireless settings}
	\item \texttt{12061 [QoS]QoS Setup should gray-out when set WDS Repeater mode.}
	\item \texttt{12065 [QoS]When set MAC Address priority ,something is wrong.}
	\item \texttt{12124 [Wireless Card Access List][a/n][b/g/n]Wireless Card Access List page shows error when you input some special characters as device name}
	\item \texttt{12171 According to netgear spec2.1, add hotkeys in Wireless Settings GUI page}
	\item \texttt{12180 Merge 2 band into one page for advanced wireless setting}
	\item \texttt{12198 [SQA Taipei-18][RIP]The default value of RIP direction is 'Both' . Default must is 'None'.}
	\item \texttt{12251 [PPPoE]Static IP address in PPPoE page saves config error in both setup wizard and basic settings page.}
	\item \texttt{12351 Please add GMT+5:30 Chennai, Kolkata, Mumbai, New Delhi in the Time Zone.}
	\item \texttt{12374 [WPS]WPS is configured but "Keep Existing Wireless Settings" on Advanced wireless settings a/n page does not check}
	\item \texttt{12456 The Show Status never show Dynamic DNS service is not enabled. , even if I disable the DDNS function.}
	\item \texttt{11432 Wireless LED indication}
	\item \texttt{11887 [SIP]LAN PCs can not talk to each other for a long time.}
	\item \texttt{11921 [Wiress settings a/n]Channels shows error in Wiress settings a/n page}
	\item \texttt{12057 ethernet LAN/WAN throughput has only 170M}
	\item \texttt{12058 Wireless LED can not be on/off}
	\item \texttt{12100 Sometimes strange infomation in console}
	\item \texttt{12156 Wireless on/off push button should can be controlled}
	\item \texttt{12163 The command to call wsccmd should be changed.}
	\item \texttt{12165 The LOGO of wnr1000v2 should be fetched from NETGEAR.}
	\item \texttt{12166 Replace all the strings "WNR1000v2" to "WNR1000v2"}
	\item \texttt{12181 [SQA Taipei-1]The product name should be "WNR1000v2", but GUI displays "WNR2000".}
	\item \texttt{12203 [SQA Taipei-23]LAN and WAN port MAC is FF:FF:FF:FF:FF:FF.}
	\item \texttt{12230 Switch HW or Switch driver issue}
	\item \texttt{12270 [DNI-Storage] user's group name should be corresponding to group name}
	\item \texttt{12294 several giga bit Ethernet PCs cant be detected as 1Gbps}
	\item \texttt{11431 Power LED indication support}
	\item \texttt{11686 [DNI-statistics] there is no info for wan and lan status}
	\item \texttt{11889 Ethernet port should be supported in QoS}
	\item \texttt{12044 Should add power save feature in switch driver}
	\item \texttt{12045 Should add green packet feature in switch driver}
	\item \texttt{12062 [QoS]The traffic ratio is wrong.}
	\item \texttt{12155 Wireless LED should be controlled by software}
	\item \texttt{12193 [SQA Taipei-13][Qos]The QoS function does not be enabled immediately until we restart DUT.}
	\item \texttt{12211 [u-boot] add wpspinset command to write wps pin to flash}
	\item \texttt{12287 HW design spec have ask SW to set LED mode}
	\item \texttt{12367 The LAN LED is not lightened up if we connect to 10Mbps Ethernet adapter.}
\end{itemize}
\end{enumerate}
\end{itemize}

\subsection{Steps to burn boot loader and firmware}
\begin{itemize}
\item As below:
        \begin{enumerate}
                \item \texttt{Please burn u-boot-1000V0.5.bin}
                \item \texttt{Set up a tftp server on your PC, its ip address is 192.168.1.12.}
                \item \texttt{Entering into boot loader}
                \item \texttt{ag7100> set serverip 192.168.1.12}
                \item \texttt{ag7100> tftp 0x80010000 u-boot-1000V0.5.bin}
		\item \texttt{ag7100> erase 0xbf000000 +0x50000}
		\item \texttt{ag7100> cp.b 0x80010000 0xbf000000 0x50000}
                \item \texttt{ag7100> set bootcmd 'fsload 80800000 image/uImage;bootm 80800000'}
                \item \texttt{ag7100> saveenv}
                \item \texttt{ag7100> reset}
                \item \texttt{Entering into boot loader again}
                \item \texttt{ag7100>bootm}
                \item \texttt{Then the device should be in tftp recovery mode. Please run the coammdn "tftp -i 192.168.1.1 put WNR1000v2-V1.0.0.9.img" on MS-DOS of your PC.}
        \end{enumerate}
\end{itemize}

\subsection{Known issues}
        \begin{enumerate}
                \item \texttt{This version has been updated to Atheros LSDK 7.2.0.143.}
        \end{enumerate}		
		
\section{Firmware V1.0.1.0}

\tlabel{sec:1-0-1}
\subsection{Repository}
\begin{itemize}
\item GIT Repository  itgserver/pub/scm/openwrt/ronger/openwrt.git
\begin{itemize}
    \item Branch: \texttt{sh\_sw\_one\_br}
    \item Tag: \texttt{WNR1000v2-V-1-0-1-0}
\end{itemize}
\end{itemize}

\subsection{New Features}
\begin{itemize}
\item As below:
\begin{enumerate}
	\item APPLICATION
	\begin{itemize}
		\item 	\texttt{11437 Router debugging mode should be supported}
		\item 	\texttt{11416 Wi-Fi Protected Setup(WPS) support}
	\end{itemize}
\end{enumerate}

\subsection{Fixed Bugs}
\begin{itemize}
\item As below:
\begin{enumerate}
	\item APPLICATION
	\begin{itemize}
		\item 	\texttt{12475 Display correct sharing folder in the windows and save sharing folder records.}
		\item 	\texttt{12176 Disk management should be supported}
		\item 	\texttt{12327 [Disk Management]The DUT should support more than one partitions}
		\item 	\texttt{12336 [Disk Management]When I visit "192.168.1.1",it shows all the \/tmp in the DUT}
		\item 	\texttt{12431 USB bug--location should display currnent folder location}
		\item 	\texttt{12432 USB bug--Should return to the upper of the root folder, so you can share the whole HDD.}
		\item 	\texttt{12434 USB bug--After you add second or more shared folders, then the shared folder page in the "Disk management" page should be refreshed automatically.}
		\item 	\texttt{12522 [USB storage]The shared folder can not be visited.}
		\item 	\texttt{12523 [USB storage]After reboot,usb storage can not work until you apply again}
		\item 	\texttt{12551 [SAMBA]wrong share info file name}
		\item 	\texttt{11737 [Block sites]NNTP can not be blocked when they contain the keyword.}
		\item 	\texttt{11767 [DHCP-client]For the cases of Ethernet cable plug-off and plug-in, \\
			client should send DHCP_REQUEST and include current IP address as preferred \\
			$Client IP.}
		\item 	\texttt{11769 [DHCP-client]Whenever IP assignment, router should automatically flush the static route (should only flush the WAN routing)}
		\item 	\texttt{11786 [BPA]Connection should disconnect if no traffic through after idle timeout}
		\item 	\texttt{11788 [BPA]The session needs to be restarted if detected by not seeing incoming heartbeats for a timeout period of 15 minutes then client should restart the authentication process.}
		\item 	\texttt{11880 [NAT]NAT Mapping Timer test fail}
		\item 	\texttt{11881 [NAT]"Outbound UDP Packets with Destination Port . 1024" test fail}
		\item 	\texttt{11940 [NTP]NTP time can not get the Current time unless you reboot the DUT if you do not connect the DUT to internet first.}
		\item 	\texttt{12122 Before sending discover packet, should check whether WAN cable is plugged.}
		\item 	\texttt{12152 [Port Triggering]The port trigger could not use the same outgoing port to trigger different port rules.}
		\item 	\texttt{12185 [SQA Taipei-5][PPTP]PPTP idle time function is fail.}
		\item 	\texttt{12186 [SQA Taipei-6][Setup wizard]If WAN port cable not plugged in and we use wizard to detect WAN port type,the detected is "Static IP". It should display failure page.}
		\item 	\texttt{12206 [SQA Taipei-26][CDRouter]CDRouter port trigger function test fail.}
		\item 	\texttt{12252 [Setup wizard]Sometimes setup wizard can not go through,it always refresh the page "Detecting Connection Type on Internet Port"}
		\item 	\texttt{12330 [Disk Management]After working in Disk Management page for a few time,there a too many "uhttpd"}
		\item 	\texttt{12366 Before sending NTP packet, the WAN connection must be connected.}
		\item 	\texttt{12405 [SQA Taipei-35]Unable to open the "Wireless a\/n" \/ "Wireless b\/g\/n" \/ "Guest Network" \/"schedule" \/ "port forwarding" pages when we login DUT remotely.}
		\item 	\texttt{12571 Remove the line "WAN cable is not plugged, NOT send discover!!!" in dhcpc}
		\item 	\texttt{12178 WPS Push button behavior definition}
		\item 	\texttt{12232 [Basic wireless settings][b/g/n]No matter what I set, PassPhrase always shows NETGEAR in WPA-PSK,WPA2-PSK,WPA/WPA2-PSK in UI and shows no security in iwconfig}
		\item 	\texttt{12385 [WPS]WPS icon under Vista still shows WNR2000}
		\item 	\texttt{11927 [DMZ]If LAN IP address changes,DMZ IP address shows error}
		\item 	\texttt{12124 [Wireless Card Access List][a/n][b/g/n]Wireless Card Access List page shows error when you input some special characters as device name}
		\item 	\texttt{12433 USB bug--After you assigned one available group to "assigned to folder", then that group should be removed from "available group".}
		\item 	\texttt{12163 The command to call wsccmd should be changed.}
		\item 	\texttt{12363 [WPS]Configure APUT using PIN method through Vista, if I set security mode as WEP,the DUT should select Open WEP}
		\item 	\texttt{12373 [WPS]Configure APUT using PIN method through Vista, if I set security mode as WPA2-PSK,it shows WPA-PSK in both console and UI.}
		\item 	\texttt{12380 [WPS]The DUT can not add WPS client through Software Push button}
		\item 	\texttt{11686 [DNI-statistics] there is no info for wan and lan status}
		\item 	\texttt{12062 [QoS]The traffic ratio is wrong.}
		\item 	\texttt{12147 [SIP]Imcoming call pop up constantly when call from one PC to another in LAN}
		\item 	\texttt{12188 [qos] when reboot the board, qos does not takce effect}
		\item 	\texttt{11732 There is Call Trace info in console when running the DUT for two days without doing anything}
		\item 	\texttt{12260 [Disk Management]"Safely Remove Disk" button can not work.}
		\item 	\texttt{12497 Sometimes Hotplug created an useless folder}
	\end{itemize}
	
\end{enumerate}
\begin{enumerate}
	\item GUI
	\begin{itemize}
		\item 	\texttt{11873 [Top]Please change the model name to WNR1000v2 in top frame's picture}
		\item 	\texttt{11972 wireless network mode is not match the spec 1.6}
		\item 	\texttt{11998 [Schedule]Please check Start Blocking and End Blocking}
		\item 	\texttt{12086 [Basic wireless settings][a/n]When "Up to 300Mbps" is selected, router status does not show double channels}
		\item 	\texttt{12098 [Basic Wireless settings][a/n][b/g/n][Guest Network][WPA-PSK][WPA2-PSK][WPA/WPA2-PSK]If there is a space in WPA-PSK PassPhrase, the webpage shows error.}
		\item 	\texttt{12101 [Basic Wireless settings][a/n][b/g/n][Guest Network][WPA-PSK]If I set WPA-PSK PassPhrase as </script>,the webpage shows error}
		\item 	\texttt{12136 [Guest Network]Please add a Guest Network webpage for a/n}
		\item 	\texttt{12179 Merge 2 band into one page for basic wireless setting and advanced wireless setting}
		\item 	\texttt{12209 [SQA Taipei-30]The default region is " Select Region" on both wireless setting a/n and wireless setting b/g/n pages, but the status page shows "Europe".}
		\item 	\texttt{12261 [Storage Administration]When you edit "guest",Group Member shows error in UI.}
		\item 	\texttt{12282 wireless setting a/n page should show correct channel list}
		\item 	\texttt{12299 [Disk Management--Shared Folder]There is a javascript error when you click ">>" or "<<" button and no one in "Available Groups" is selected}
		\item 	\texttt{12300 [Disk Management--Shared Folder]Location shows error in edit mode}
		\item 	\texttt{12301 [Disk Management--Shared Folder]There is a javascript error if no folder is selected in folder list and click "Save" button}
		\item 	\texttt{12308 [Disk Management--Shared Folder]Shared Folder shows error if there is a warning messages before it saves config.}
		\item 	\texttt{12311 [Disk Management--Shared Folder]Shared Folder should not begin with \/\/}
		\item 	\texttt{12316 [Disk Management--Shared Folder]After add a Shared Folder, Disk Management page can not return to itself.}
		\item 	\texttt{12317 [Disk Management--Shared Folder]After you edit the Shared Folder list, shared folder become blank in the table}
		\item 	\texttt{12321 [Disk Management]Disk Management page shows error if you input some special characters as Display name}
		\item 	\texttt{12324 [Disk Management]There should be only one : after "Shared Storage IP Address"}
		\item 	\texttt{12326 [Disk Management]There should be a : after "Disk"}
		\item 	\texttt{12332 [Disk Management]"Groups with Access" shows error with Sub-folder.}
		\item 	\texttt{12338 [Storage Administration]When I click "delete" button in User Accounts list,there is a warning pop up}
		\item 	\texttt{12339 [Storage Administration]The webpage can not return after you delete a group in Group Management list and User Account list}
		\item 	\texttt{12340 [Storage Administration-User Accounts]User name must be unique}
		\item 	\texttt{12344 [Storage Administration-Group Account]If there is a space in description,the webpage shows error.}
		\item 	\texttt{12345 [Storage Administration-Group Account]Group account and Description should allow special characters}
		\item 	\texttt{12347 [Storage Administration-User Account]If there is a space in full name or Description,the webpage shows error.}
		\item 	\texttt{12349 [Storage Administration-Group Account]Group account and Description should not allow 2bytes characters}
		\item 	\texttt{12368 There is no "select region" in the region list, the default value of region is NOT "select region". Please refer to router spec.}
		\item 	\texttt{12422 After doing reset to default, it can't open "routerlogin" page automatically.}
		\item 	\texttt{12498 [DISK-Share Flolder] if group is special char, Available Groups will show wrong}
		\item 	\texttt{12500 [Guest network]Menu should allow guest network for a/n}
		\item 	\texttt{12304 [Disk Management--Shared Folder]Folder list shows error if there is a folder which folder name contents 128 characters}
		\item 	\texttt{12305 [Disk Management--Shared Folder]Folder name should allow 128 characters}
		\item 	\texttt{12322 [Disk Management--Shared Folder]Assigned to Folder field should not be blank if you want to add a folder to shared folder list.}
		\item 	\texttt{12334 [Disk Management]The same shared folder could show in the Shared Folder list with different display name.}
		\item 	\texttt{12348 [Storage Administration-Group Account]User name,full name and Description should allow special characters}
		\item 	\texttt{12476 The Location of shared folder dialog in USB disk management display error.}
		\item 	\texttt{11953 [NAT][WAN/LAN conflict]When the LAN-WAN IP address conflict, the web page of DUT should give a tip of IP address conflict}
		\item 	\texttt{12184 [SQA Taipei-4][Setup wizard]The account name is NULL when using wizard to configure DHCP mode.}
		\item 	\texttt{12214 According to WNR1000, Newly added port triggering rule should be enabled by default}
		\item 	\texttt{12399 [SQA Taipei-32]Click the "Clear Log" button on log page, GUI sometime cannot open log page correctly}
		\item 	\texttt{12407 [SQA Taipei-36]GUI displays the remote login port is being used when we modify "allow remote access by" method from one to another.}
		\item 	\texttt{12034 Wrong command in the apup}
		\item 	\texttt{11317 Modify web pages for netgear's auto test according to netgear GUI spec 2.1}
		\item 	\texttt{11451 Modified "QoS Setup" page according to netgear GUI spec 2.1}
		\item 	\texttt{11454 Modified "Remote Management" page according to netgear GUI spec 2.1}
		\item 	\texttt{11456 Modified "Wireless Settings" page according to netgear GUI spec 2.1}
		\item 	\texttt{11541 Modified "Router Upgrade" page according to netgear GUI spec 2.1}
		\item 	\texttt{11919 [Port triggering]Port triggering page shows error when you input some special characters as server name}
		\item 	\texttt{11920 [Port tringgering]Although the warning messages says the port must be in the range [1-65535] excluding 80,I can input 80 port}
		\item 	\texttt{11932 [NAPT]The rule should be added when there exists a temporal rule using the same public port and mapping to the same LAN IP and private/internal port}
		\item 	\texttt{11936 [DHCP Server]LAN Setup page shows error when you input some special characters as device name}
		\item 	\texttt{12056 Add auto radio test button for wireless settings}
		\item 	\texttt{12061 [QoS]QoS Setup should gray-out when set WDS Repeater mode.}
		\item 	\texttt{12065 [QoS]When set MAC Address priority ,something is wrong.}
		\item 	\texttt{12153 [QoS]The Uplink bandwidth should more than 100Mbps.}
		\item 	\texttt{12171 According to netgear spec2.1, add hotkeys in Wireless Settings GUI page}
		\item 	\texttt{12180 Merge 2 band into one page for advanced wireless setting}
		\item 	\texttt{12198 [SQA Taipei-18][RIP]The default value of RIP direction is 'Both' . Default must is 'None'.}
		\item 	\texttt{12250 [Connection status][BPA]Disconnect and connect to BPA,the webpage shows Logged out}
		\item 	\texttt{12251 [PPPoE]Static IP address in PPPoE page saves config error in both setup wizard and basic settings page.}
		\item 	\texttt{12259 [Storage Administration]After add a account,webpage shows strange.}
		\item 	\texttt{12319 [Disk Management--Shared Folder]If I input a error Display name,the warning message is error}
		\item 	\texttt{12341 [Storage Administration-Group Account]Warning message is error}
		\item 	\texttt{12351 Please add GMT+5:30 Chennai, Kolkata, Mumbai, New Delhi in the Time Zone.}
		\item 	\texttt{12374 [WPS]WPS is configured but "Keep Existing Wireless Settings" on Advanced wireless settings a/n page does not check}
		\item 	\texttt{12382 [WPS]When you add WPS client successful with PIN mode,the webpages shows timeout}
		\item 	\texttt{12456 The Show Status never show Dynamic DNS service is not enabled. , even if I disable the DDNS function.}
		\item 	\texttt{12513 [Storage Administration-User Accounts]If you add more than one user in User Accont page ,there is a javascript error}
		\item 	\texttt{12165 The LOGO of wnr1000v2 should be fetched from NETGEAR.}
		\item 	\texttt{12548 The LAN and WAN status of statistics page always displays "Link Down"}
		\item 	\texttt{11400 NAPT Conflicts Resolution}
		\item 	\texttt{11448 Samba and hotplug porting}
		\item 	\texttt{11764 MAC address of wifi0 and wifi1 always are 00:00:00:00:00:00}
		\item 	\texttt{11973 country setting ioctl can't work}
		\item 	\texttt{12003 hostapd should be included in configuration file}
		\item 	\texttt{12068 [Basic Wireless settings][a/n][b/g/n]If there is a space in SSID,the SSID shows OK in UI but error in iwconfig}
		\item 	\texttt{12099 [Basic Wireless settings][a/n][b/g/n][Guest Network][WPA-PSK][WPA2-PSK][WPA/WPA2-PSK]If I set WPA-PSK PassPhrase as 64 hexdigits, it save as no security in function}
		\item 	\texttt{12112 [Advanced Wireless settings][a/n][b/g/n]"Enable Wireless Router Radio" in Advanced wireless settings page should not turn on and off both a/n and b/g/n.}
		\item 	\texttt{12114 [Advanced Wireless settings][a/n][b/g/n]"Disable SSID Broadcast" does not work}
		\item 	\texttt{12131 [Wireless Card Access List][a/n]Access Control does not work.}
		\item 	\texttt{12137 [Guest Network][b/g/n]"Disable SSID Broadcast" does not work on Guest Network page}
		\item 	\texttt{12236 Should adjust the channel list under 300M mode of a band}
		\item 	\texttt{12383 [Basic Wireless settings][b/g/n]When I set open wep,it works error in both UI and console.}
		\item 	\texttt{12507 [Guest Network]If there is a space in SSID,the SSID shows OK in UI but error in iwconfig}
		\item 	\texttt{12509 [Basic wireless settings]Wireless settings shows different between UI and console}
		\item 	\texttt{12526 [Advance wireless setting]If "Enable Wireless Router Radio" is uncheck and reboot the DUT,wireless will not work.}
		\item 	\texttt{12083 [Basic Wireless settings][a/n]Wireless client WNDA3100 can scan and connnect to the DUT only when the DUT works in channel 36~48,149~165}
		\item 	\texttt{12135 [Wireless Card Access List][b/g/n]Access Control on Wireless Card Access List page control both a/n and b/g/n mode}
		\item 	\texttt{12210 [SQA Taipei-31]Wireless clients(Intel? PRO/Wireless 4965AGN, Intel? PRO/Wireless 3965AG, Atheros AR5008, Netgear WNDA3100) cannot associate with DUT.}

	\end{itemize}
	\begin{enumerate}
	\item WIRELESS
	\begin{itemize}
		\item 	\texttt{12113 Control wireless on/off behavior according to wireless push button}
		\item 	\texttt{12115 still saw "Not associated" on 5GHz interface}
		\item 	\texttt{12229 RF and Wireless SW issue}	
	\end{itemize}
	\end{enumerate}
\end{enumerate}
\end{itemize}

\subsection{Steps to burn boot loader and firmware}
\begin{itemize}
\item As below:
        \begin{enumerate}
                \item \texttt{Please burn u-boot-1000V0.5.bin}
                \item \texttt{Set up a tftp server on your PC, its ip address is 192.168.1.12.}
                \item \texttt{Entering into boot loader}
                \item \texttt{ag7100> set serverip 192.168.1.12}
                \item \texttt{ag7100> tftp 0x80010000 u-boot-1000V0.5.bin}
		\item \texttt{ag7100> erase 0xbf000000 +0x50000}
		\item \texttt{ag7100> cp.b 0x80010000 0xbf000000 0x50000}
                \item \texttt{ag7100> set bootcmd 'fsload 80800000 image/uImage;bootm 80800000'}
                \item \texttt{ag7100> saveenv}
                \item \texttt{ag7100> reset}
                \item \texttt{Entering into boot loader again}
                \item \texttt{ag7100>bootm}
                \item \texttt{Then the device should be in tftp recovery mode. Please run the coammdn "tftp -i 192.168.1.1 put WNR1000v2-V1.0.1.0.img" on MS-DOS of your PC.}
        \end{enumerate}
\end{itemize}

\subsection{Known issues}
        \begin{enumerate}
                \item \texttt{This version has been updated to Atheros LSDK 7.2.0.143.}
        \end{enumerate}		

\section{Firmware V1.0.1.1}

\tlabel{sec:1-0-1}
\subsection{Repository}
\begin{itemize}
\item GIT Repository  itgserver/pub/scm/openwrt/ronger/openwrt.git
\begin{itemize}
    \item Branch: \texttt{sh\_sw\_one\_br}
    \item Tag: \texttt{WNR1000v2-V-1-0-1-1}
\end{itemize}
\end{itemize}

\subsection{Fixed Bugs}
\begin{itemize}
\item As below:
\begin{enumerate}
	\item APPLICATION
	\begin{itemize}
		\item 	\texttt{12596 The read&write changes to read only in samba read&write group.}
		\item 	\texttt{12355 [Storage Administration]User Account "Admin" and "Guest" have the same access}
		\item 	\texttt{12595 Samba should be called or killed in hotplug routine.}
		\item 	\texttt{12613 hotplug2 warning message is dumped after system boot}
		\item 	\texttt{12631 More than one init process after system boot up}
		\item 	\texttt{12566 [QoS]When delete some default QoS Policy ,the new QoS Policy can not be saved.}
	\end{itemize}
\end{enumerate}
\begin{enumerate}
	\item GUI
	\begin{itemize}
		\item 	\texttt{12177 [Port Triggering]Port Triggering Timeout(in minutes) is wrong.}
		\item 	\texttt{12306 [Disk Management--Shared Folder]Folder name should allow special characters}
		\item 	\texttt{12302 [Disk Management--Shared Folder]When you creat new folder, space should be allowed}
		\item 	\texttt{12153 [QoS]The Uplink bandwidth should more than 100Mbps.}
	\end{itemize}
\end{enumerate}
\begin{enumerate}
	\item WIRELESS
	\begin{itemize}
		\item 	\texttt{12526 [Advance wireless setting]If "Enable Wireless Router Radio" \\
						is uncheck and reboot the DUT,wireless will not work.}
		\item 	\texttt{12609 [Wireless][b/g/n]When security mode is WPA-PSK [TKIP] + \\
						WPA2-PSK [AES],it shows no security in iwconfig}
		\item 	\texttt{12084 [Basic wireless settings][a/n]According to Netgear spec1.6,\\
						the default mode for 11N 5GHz is 40 MHz.}
		\item 	\texttt{12362 [WPS]Configure APUT using PIN method through Vista or Intel PROSet, \\
						it shows OK in console,but error in UI}
		\item 	\texttt{12365 [WPS]Configure APUT using PIN method through Vista or Intel PROSet,\\
						if there is a space in SSID, it shows OK in console,but error in UI}
		\item 	\texttt{12384 [WPS]"Disable Router's PIN" can not work.}
		\item 	\texttt{12020 sometimes the device will crash if set 300M+wep or 300M+wpa}
		\item 	\texttt{12568 system crash when g/n interface is configured as wpa-psk(tkip) and radio disabled}
	\end{itemize}
\end{enumerate}
\begin{enumerate}
	\item ETHERNET
	\begin{itemize}
		\item 	\texttt{11807 [PPTP]The DUT has been crash when I set PPTP}
		\item 	\texttt{12222 [Ethernet]LAN/WAN transmit timed out when connected with ADSL modem}
	\end{itemize}
\end{enumerate}
\begin{enumerate}
	\item KERNEL
	\begin{itemize}
		\item 	\texttt{11939 [Logs]Some logs shows not Complete}
		\item 	\texttt{12147 [SIP]Imcoming call pop up constantly when call from one PC to another in LAN}
	\end{itemize}
\end{enumerate}
\end{itemize}

\subsection{Steps to burn boot loader and firmware}
\begin{itemize}
\item As below:
        \begin{enumerate}
                \item \texttt{Please burn u-boot-1000V0.5.bin}
                \item \texttt{Set up a tftp server on your PC, its ip address is 192.168.1.12.}
                \item \texttt{Entering into boot loader}
                \item \texttt{ag7100> set serverip 192.168.1.12}
                \item \texttt{ag7100> tftp 0x80010000 u-boot-1000V0.5.bin}
		\item \texttt{ag7100> erase 0xbf000000 +0x50000}
		\item \texttt{ag7100> cp.b 0x80010000 0xbf000000 0x50000}
                \item \texttt{ag7100> set bootcmd 'fsload 80800000 image/uImage;bootm 80800000'}
                \item \texttt{ag7100> saveenv}
                \item \texttt{ag7100> reset}
                \item \texttt{Entering into boot loader again}
                \item \texttt{ag7100>bootm}
                \item \texttt{Then the device should be in tftp recovery mode. Please run the coammdn "tftp -i 192.168.1.1 put WNR1000v2-V1.0.1.1.img" on MS-DOS of your PC.}
        \end{enumerate}
\end{itemize}

\subsection{Known issues}
        \begin{enumerate}
                \item \texttt{This version has been updated to Atheros LSDK 7.2.0.143.}
        \end{enumerate}	

\section{Firmware V1.0.1.2}

\tlabel{sec:1-0-1}
\subsection{Repository}
\begin{itemize}
\item GIT Repository  itgserver/pub/scm/openwrt/ronger/openwrt.git
\begin{itemize}
    \item Branch: \texttt{sh\_sw\_one\_br}
    \item Tag: \texttt{WNR1000v2-V-1-0-1-2}
\end{itemize}
\end{itemize}

\subsection{Fixed Bugs}
\begin{itemize}
\item As below:
\begin{enumerate}
	\item APPLICATION
	\begin{itemize}
		\item 	\texttt{12668 Opened files not closed after use }
		\item 	\texttt{12602 [Disk Management --Shared folder]When you add a shared folder,\\
						Partition shows in Shared Folder list is error }
		\item 	\texttt{12537 WPS run twice under unconfigured mode, and run one time under configured mode. }
		\item \texttt{12363 [WPS]Configure APUT using PIN method through Vista, if I set security \\
						mode as WEP,the DUT should select Open WEP.}
		\item \texttt{12662 [QoS]QoS can not work in V1.0.1.0}
		\item \texttt{12634 [USB storage]User Accounts "Admin" and "Guest" , \\
						Groups "Admin" and "Guest" can not visit the shared folder}
		\item \texttt{12334 [Disk Management]The same shared folder could show in \\
						the Shared Folder list with different display name.}
	\end{itemize}
\end{enumerate}
\begin{enumerate}
	\item GUI
	\begin{itemize}
		\item 	\texttt{12370 After doing reset to default, it can't open "routerlogin" page automatically. }
		\item 	\texttt{12130 [Wireless Card Access List][a/n]The Available Wireless Cards \\
						list should display any available wireless PCs and their MAC addresses.}
		\item \texttt{12598 [Disk Management--Shared folder]Please define the maxlength of folder name in UI }
		\item \texttt{12350 [Storage Administration-Group Account]User name,full name and \\
						Description should not allow 2 bytes characters }
		\item \texttt{12603 [Storage Administration -- User Account]There is a javascript \\
						error when you edit "User Account" page.}
		\item \texttt{12604 [Disk Management--Shared folder]After you click "Cancel" \\
						button,Edit "Shared Folder" page acts as add a shared folder }
		\item \texttt{12657 [Disk Management--Shared folder]"Cancel" button in edit \\
						"Shared folder" make partition and location NULL on the webpage} 
		\item \texttt{12665 [Router status]WLAN status always shows 54M on Statistics page}
		\item \texttt{12306 [Disk Management--Shared Folder]Folder name should allow special characters}
		\item \texttt{12382 [WPS]When you add WPS client successful with PIN mode,the webpages shows timeout}
	\end{itemize}
\end{enumerate}
\begin{enumerate}
	\item WIRELESS
	\begin{itemize}
		\item 	\texttt{12528 [Wireless]Sometimes restore to factory default, ath0 does not shows in ifconfig }
	\end{itemize}
\end{enumerate}
\begin{enumerate}
	\item ETHERNET
	\begin{itemize}
		\item 	\texttt{12144 [Guest Network][b\/g\/n]After apply this page,webpage can not return.}
	\end{itemize}
\end{enumerate}
\begin{enumerate}
	\item KERNEL
	\begin{itemize}
		\item 	\texttt{12231 ntfs should support write}
	\end{itemize}
\end{enumerate}
\end{itemize}

\subsection{Steps to burn boot loader and firmware}
\begin{itemize}
\item As below:
        \begin{enumerate}
                \item \texttt{Please burn u-boot-1000V0.5.bin}
                \item \texttt{Set up a tftp server on your PC, its ip address is 192.168.1.12.}
                \item \texttt{Entering into boot loader}
                \item \texttt{ag7100> set serverip 192.168.1.12}
                \item \texttt{ag7100> tftp 0x80010000 u-boot-1000V0.5.bin}
		\item \texttt{ag7100> erase 0xbf000000 +0x50000}
		\item \texttt{ag7100> cp.b 0x80010000 0xbf000000 0x50000}
                \item \texttt{ag7100> set bootcmd 'fsload 80800000 image/uImage;bootm 80800000'}
                \item \texttt{ag7100> saveenv}
                \item \texttt{ag7100> reset}
                \item \texttt{Entering into boot loader again}
                \item \texttt{ag7100>bootm}
                \item \texttt{Then the device should be in tftp recovery mode. Please run the coammdn "tftp -i 192.168.1.1 put WNR1000v2-V1.0.1.2.img" on MS-DOS of your PC.}
        \end{enumerate}
\end{itemize}

\subsection{Known issues}
        \begin{enumerate}
                \item \texttt{This version has been updated to Atheros LSDK 7.2.0.143.}
        \end{enumerate}	

\section{Firmware V1.0.1.3}

\tlabel{sec:1-0-1}
\subsection{Repository}
\begin{itemize}
\item GIT Repository  itgserver/pub/scm/openwrt/ronger/openwrt.git
\begin{itemize}
    \item Branch: \texttt{sh\_sw\_one\_br}
    \item Tag: \texttt{WNR1000v2-V-1-0-1-3}
\end{itemize}
\end{itemize}

\subsection{Fixed Bugs}
\begin{itemize}
\item As below:
\begin{enumerate}
	\item APPLICATION
	\begin{itemize}
		\item 	\texttt{12637[Disk Management --Shared folder]When you edit the shared folder, the group shows error}
		\item 	\texttt{12666 the Location of Partition in sharedfolder list in disk management web show error .}
		\item 	\texttt{12602[Disk Management --Shared folder]When you add a shared folder,Partition \\
						shows in Shared Folder list is error }
		\item \texttt{12760 [disk management]when push "Return to Upper Folder" button ,can not return top folder .}
		\item \texttt{12690 [Disk Management --Shared folder]When you remove the disk,the infomation in the shared folder list should be remove }
		\item \texttt{12700 [USB storage]After restore to factory default,the shared folder \\
						list is still exist but the group info and account info have gone}
		\item \texttt{12774 [wnr1000v2][disk management && user management]so many print infomation \\
						on the termination ,badly slow down the speed of update of webpage. }
		\item \texttt{12746    	Need to reload samba service if LAN IP changed }
		\item \texttt{12706    	[WPS]According to Netgear's spec, Auto WEP should support WPS }
		\item \texttt{12683    	[WPS][Advance wireless settings]Uncheck "Keep Existing \\
						wireless settings" and apply will reset all wireless settings to default value }
		\item \texttt{12692    	Many potential issues inside, some settings in \\
						AP94 but you didn��t put it in 1000 script.}
		\item \texttt{11965    	[Multiple BSSID]Modify bridge to fit netgear's spec}
		\item \texttt{12736    	Maybe we should replace nmbd with our netbios name service }
		\item \texttt{12611    	nmbd sends 'br0' DNS query }
		\item \texttt{12759    	Router can't response NBNS query with resource type 20 }
		\item \texttt{12798    	update_smb is too slow}
	\end{itemize}
\end{enumerate}
\begin{enumerate}
	\item GUI
	\begin{itemize}
		\item 	\texttt{12660    	[Router status]When you select the regions that do not support \\
						a band, you should not shows the mode in router status page. }
		\item 	\texttt{12697    	[Router status]Double channels shows error in router status page }
		\item \texttt{12708    	[Basic wireless settings][a/n][b/g/n]When "Up to 300M" mode, it should show double channels}
		\item \texttt{12765    	[WDS]The router should not input itself MAC Address.}
		\item \texttt{12766    	webpage should let WPS run in WEP OPEN/AUTO mode }
		\item \texttt{12330    	[Disk Management]After working in Disk Management page for a few time,there a too many "uhttpd"}
		\item \texttt{12738    	[WPS]When use WLAN external Registrar to config the key \\
						with spaces,the spaces can not display in UI.} 
	\end{itemize}
\end{enumerate}
\end{itemize}

\subsection{Steps to burn boot loader and firmware}
\begin{itemize}
\item As below:
        \begin{enumerate}
                \item \texttt{Please burn u-boot-1000V0.5.bin}
                \item \texttt{Set up a tftp server on your PC, its ip address is 192.168.1.12.}
                \item \texttt{Entering into boot loader}
                \item \texttt{ag7100> set serverip 192.168.1.12}
                \item \texttt{ag7100> tftp 0x80010000 u-boot-1000V0.5.bin}
		\item \texttt{ag7100> erase 0xbf000000 +0x50000}
		\item \texttt{ag7100> cp.b 0x80010000 0xbf000000 0x50000}
                \item \texttt{ag7100> set bootcmd 'fsload 80800000 image/uImage;bootm 80800000'}
                \item \texttt{ag7100> saveenv}
                \item \texttt{ag7100> reset}
                \item \texttt{Entering into boot loader again}
                \item \texttt{ag7100>bootm}
                \item \texttt{Then the device should be in tftp recovery mode. Please run the coammdn "tftp -i 192.168.1.1 put WNR1000v2-V1.0.1.3.img" on MS-DOS of your PC.}
        \end{enumerate}
\end{itemize}

\subsection{Known issues}
        \begin{enumerate}
                \item \texttt{This version has been updated to Atheros LSDK 7.2.0.143.}
        \end{enumerate}	

		
\section{Firmware V1.0.1.4}

\tlabel{sec:1-0-1}
\subsection{Repository}
\begin{itemize}
\item GIT Repository  itgserver/pub/scm/openwrt/ronger/openwrt.git
\begin{itemize}
    \item Branch: \texttt{sh\_sw\_one\_br}
    \item Tag: \texttt{WNR1000v2-V-1-0-1-4}
\end{itemize}
\end{itemize}

\subsection{Fixed Bugs}
\begin{itemize}
\item As below:
\begin{enumerate}
	\item APPLICATION
	\begin{itemize}
		\item 	\texttt{12806 [Disk Management]When return to the top folder,\\
						the same shared folder could show in the Shared Folder \\
						list with different display name }
		\item 	\texttt{12858 [Disk Management --Shared folder]When you edit the \\
						top folder, it does not show the correct selection. }
		\item 	\texttt{12191 [SQA Taipei-10][LED]When WAN connection is DHCP mode, the LED is amber.}
		\item \texttt{12192[SQA Taipei-12][LED]Bigpond mode not connected to internet, the WAN LED is green. }
		\item \texttt{12194 [SQA Taipei-14][LED]When WAN port is static IP the WAN LED is amber.}
		\item \texttt{12196 [SQA Taipei-16][LED]When WAN port IP is released, the WAN LED is still green. \\
						It should be amber }
		\item \texttt{12810 [Vista Premium]Enable 6to4 tunelling support in kernel configuration file }
		\item \texttt{12782 [LED]Click "cancel" when add WPS client through PIN or Push button,\\
						the WPS LED will always light on until reboot the DUT }
		\item \texttt{12738 [WPS]When use WLAN external Registrar to config the key with spaces,\\
						the spaces can not display in UI. }
		\item \texttt{12807 [USB storage]There are too many update_smb }
		\item \texttt{12846 [Storage Administration]If Description of Group Account is NULL,\\
						we can not visit the shared folder. }
		\item \texttt{12862 [Disk Management --Shared folder]When you remove the disk unsafely, \\
						sometimes the record still in Disk Detail list }
	\end{itemize}
\end{enumerate}
\begin{enumerate}
	\item GUI
	\begin{itemize}
		\item \texttt{12765 [WDS]The router should not input itself MAC Address. }
		\item \texttt{12728 [Router status]WLAN status should show both a\/n and b\/g\/n on Statistics page }
		\item \texttt{12730 [Top]Top frame shows error when the screen is not 1024*768 }
		\item \texttt{12270 [DNI-Storage] user's group name should be corresponding to group name }
	\end{itemize}
\end{enumerate}
\end{itemize}

\subsection{Steps to burn boot loader and firmware}
\begin{itemize}
\item As below:
        \begin{enumerate}
                \item \texttt{Please burn u-boot-1000V0.6.bin}
                \item \texttt{Set up a tftp server on your PC, its ip address is 192.168.1.12.}
                \item \texttt{Entering into boot loader}
                \item \texttt{ag7100> set serverip 192.168.1.12}
                \item \texttt{ag7100> tftp 0x80010000 u-boot-1000V0.6.bin}
		\item \texttt{ag7100> erase 0xbf000000 +0x50000}
		\item \texttt{ag7100> cp.b 0x80010000 0xbf000000 0x50000}
                \item \texttt{ag7100> set bootcmd 'fsload 80800000 image/uImage;bootm 80800000'}
                \item \texttt{ag7100> saveenv}
                \item \texttt{ag7100> reset}
                \item \texttt{Entering into boot loader again}
                \item \texttt{ag7100>bootm}
                \item \texttt{Then the device should be in tftp recovery mode. Please run the coammdn "tftp -i 192.168.1.1 put WNR1000v2-V1.0.1.4.img" on MS-DOS of your PC.}
        \end{enumerate}
\end{itemize}

\subsection{Known issues}
        \begin{enumerate}
                \item \texttt{This version has been updated to Atheros LSDK 7.2.0.143.}
        \end{enumerate}	

\section{Firmware V1.0.1.5}

\tlabel{sec:1-0-1}
\subsection{Repository}
\begin{itemize}
\item GIT Repository  itgserver/pub/scm/openwrt/ronger/openwrt.git
\begin{itemize}
    \item Branch: \texttt{sh\_sw\_one\_br}
    \item Tag: \texttt{WNR1000v2-V-1-0-1-5}
\end{itemize}
\end{itemize}

\subsection{Fixed Bugs}
\begin{itemize}
\item As below:
\begin{enumerate}
	\item APPLICATION
	\begin{itemize}
		\item 	\texttt{12850 [Storage Administration]If Full Name or Description of User Account \\
						is NULL,we can not visit the shared folder. }
		\item 	\texttt{12861 [Logs]Each time WAN disconnects, the log entry should \\
						be[Internet disconnected],but it display [Internet idle-timeout]. }
		\item 	\texttt{12808 [Wireless][b/g/n]When I set security as open \\
						WEP,it always shows Auto WEP in UI and console }
		\item \texttt{12853 [WATCHDOG] HW watchdog should be supported. }
		\item \texttt{12770 [WDS]When set Wireless Repeater,the Access Point display :Not-Associated.}
		\item \texttt{12661 [QoS]When QoS is turnned on,WAN-LAN throughput is low with giga bit Ethernet PCs}
		\item \texttt{12773 [LED]One of the "Enable Wireless Router Radio" is unchecked, \\
						another is checked,both Wireless LEDs light on.}
		\item \texttt{12856 The power LED stays AMBER during booting up and turns on GREEN after boot up finished}
		\item \texttt{12866  [USB storage]Vista can not visit the USB storage.}
		\item \texttt{12948 Can��t find 1000 in the workgroup through windows explore. }
	\end{itemize}
\end{enumerate}
\begin{enumerate}
	\item GUI
	\begin{itemize}
		\item \texttt{12819 The web page of advanced wireless settings should follow netgear's \\
						requirement. }
		\item \texttt{12302 [Disk Management--Shared Folder]When you creat new folder, space \\
						should be allowed }
		\item \texttt{12857 please keep it the same (I mean SSID, channel, and mode layout \\
						of 5GHz must be the same as 2.4GHz). }
		\item \texttt{12837 [QoS]After warning message,the wrong setting can be saved config \\
						and added to QoS list}
		\item \texttt{12818 [QoS]When "Turn Internet Access QoS On" is checked and \\
						"Turn Bandwidth Control On" is unchecked,maximum of Uplink \\
						 bandwidth become 0}
		\item \texttt{12840 [QoS]QoS list page shows error when you input some special \\
						characters as the name of QoS Policy}
		\item \texttt{12841 [QoS]On QoS setup page,after warning message for error Uplink \\
						bandwidth value,the webpage should not submit}
		\item \texttt{12852 [Advanced wireless settings]Please remove "Enable WMM" item on \\
						advanced wireless settings page because it is on Qos page already.}
		\item \texttt{12625 [QoS]When edit the QoS -Priority rules ,can not change the mac address.} 
	\end{itemize}
\end{enumerate}
\end{itemize}

\subsection{Steps to burn boot loader and firmware}
\begin{itemize}
\item As below:
        \begin{enumerate}
                \item \texttt{Please burn u-boot-1000V0.7.bin}
                \item \texttt{Set up a tftp server on your PC, its ip address is 192.168.1.12.}
                \item \texttt{Entering into boot loader}
                \item \texttt{ag7100> set serverip 192.168.1.12}
                \item \texttt{ag7100> tftp 0x80010000 u-boot-1000V0.7.bin}
		\item \texttt{ag7100> erase 0xbf000000 +0x50000}
		\item \texttt{ag7100> cp.b 0x80010000 0xbf000000 0x50000}
                \item \texttt{ag7100> set bootcmd 'fsload 80800000 image/uImage;bootm 80800000'}
                \item \texttt{ag7100> saveenv}
                \item \texttt{ag7100> reset}
                \item \texttt{Entering into boot loader again}
                \item \texttt{ag7100>bootm}
                \item \texttt{Then the device should be in tftp recovery mode. Please run the coammdn "tftp -i 192.168.1.1 put WNR1000v2-V1.0.1.5.img" on MS-DOS of your PC.}
        \end{enumerate}
\end{itemize}

\subsection{Known issues}
        \begin{enumerate}
                \item \texttt{This version has been updated to Atheros LSDK 7.2.0.143.}
        \end{enumerate}	

\section{Firmware V1.0.1.6}

\tlabel{sec:1-0-1}
\subsection{Repository}
\begin{itemize}
\item GIT Repository  itgserver/pub/scm/openwrt/ronger/openwrt.git
\begin{itemize}
    \item Branch: \texttt{sh\_sw\_one\_br}
    \item Tag: \texttt{WNR1000v2-V-1-0-1-6}
\end{itemize}
\end{itemize}

\subsection{Fixed Bugs}
\begin{itemize}
\item As below:
\begin{enumerate}
	\item APPLICATION
	\begin{itemize}
		\item 	\texttt{12894 samba should support to share the whole partition}
		\item 	\texttt{12912 [USB Storage]Folder name displays uncorrectly }
		\item 	\texttt{12959 [Basic settings]When Connection Mode is "Manually", \\
						after reboot the DUT, we should not auto dial up to the server }
		\item \texttt{12712 [Vista Premium] PnP-X metadata population }
		\item \texttt{12881 WMM on QoS page does not work. }
	\end{itemize}
\end{enumerate}
\begin{enumerate}
	\item GUI
	\begin{itemize}
		\item \texttt{12950 channel a\/n can not be AUTO if WDS is enable }
		\item \texttt{12957 channel a\/n is always AUTO }
		\item \texttt{12979 some settings and GUI are incorrect and you didn��t follow what netgear provided }
		\item \texttt{12815 [QoS]when add/edit MAC Address ,if the Device Name include space,\\
						the "Device Name"/"MAC Address"display wrong.}
		\item \texttt{12834 [QoS]According to Netgear UI spec 2.1,please add \\
						"Netgear EVA(Highest)" to Application list. }
		\item \texttt{12830 [QoS]I can add two Priorities to the same Lan port }
		\item \texttt{12351 Please add ��GMT+5:30 Chennai, Kolkata, Mumbai, New Delhi�� in the Time Zone. }
		\item \texttt{12771 [WDS]When set Wireless Repeater,the "Wireless MAC of this router"display wrong.}
		\item \texttt{12813 [WPS]When you add a client with WPS,router's PIN in advanced \\
						wireless settings page shows error } 
		\item \texttt{12995 [Wireless Settings]The a/n Channel display blank.}
		\item \texttt{12829 [Router status]Please get correct LAN port value on "show statistic" page}
	\end{itemize}
\end{enumerate}
\end{itemize}

\subsection{Steps to burn boot loader and firmware}
\begin{itemize}
\item As below:
        \begin{enumerate}
                \item \texttt{Please burn u-boot-1000V0.7.bin}
                \item \texttt{Set up a tftp server on your PC, its ip address is 192.168.1.12.}
                \item \texttt{Entering into boot loader}
                \item \texttt{ag7100> set serverip 192.168.1.12}
                \item \texttt{ag7100> tftp 0x80010000 u-boot-1000V0.7.bin}
		\item \texttt{ag7100> erase 0xbf000000 +0x50000}
		\item \texttt{ag7100> cp.b 0x80010000 0xbf000000 0x50000}
                \item \texttt{ag7100> set bootcmd 'fsload 80800000 image/uImage;bootm 80800000'}
                \item \texttt{ag7100> saveenv}
                \item \texttt{ag7100> reset}
                \item \texttt{Entering into boot loader again}
                \item \texttt{ag7100>bootm}
                \item \texttt{Then the device should be in tftp recovery mode. Please run the coammdn "tftp -i 192.168.1.1 put WNR1000v2-V1.0.1.6.img" on MS-DOS of your PC.}
        \end{enumerate}
\end{itemize}

\subsection{Known issues}
        \begin{enumerate}
                \item \texttt{This version has been updated to Atheros LSDK 7.2.0.143.}
        \end{enumerate}	

\section{Firmware V1.0.1.7}

\tlabel{sec:1-0-1}
\subsection{Repository}
\begin{itemize}
\item GIT Repository  itgserver/pub/scm/openwrt/ronger/openwrt.git
\begin{itemize}
    \item Branch: \texttt{sh\_sw\_one\_br}
    \item Tag: \texttt{WNR1000v2-V-1-0-1-7}
\end{itemize}
\end{itemize}

\subsection{Fixed Bugs}
\begin{itemize}
\item As below:
\begin{enumerate}
	\item APPLICATION
	\begin{itemize}
		\item 	\texttt{12973 [USB storage] I can\��t share any other folder with \\
						��space or other symbol in the foldername��.}
		\item 	\texttt{12700 [USB storage]After restore to factory default,the \\
						shared folder list is still exist but the group info and \\
						account info have gone }
		\item 	\texttt{13020 [Disk Management]Current folder shows error with many spaces }
		\item \texttt{12401 [SQA Taipei-33]Use "Ping of Death" to attack DUT, the log \\
						records "user.info kernel: oversized IP packet from \\
						xxx.xxx.xxx.xxx(IP) }
		\item \texttt{13069 [NETGEAR-CEC -- 731]DUT can\'t block TCP SYN scan packets correctly }
		\item \texttt{12866 [USB storage]Vista can not visit the USB storage. }
		\item \texttt{12910 there is error message when starting samba }
		\item \texttt{13077 [samba]file or filename with chinese characters can't \\
						show correctly in samba }
		\item \texttt{13126 Samba can\'t show large files }
	\end{itemize}
\end{enumerate}
\begin{enumerate}
	\item GUI
	\begin{itemize}
		\item \texttt{12765 [WDS]The router should not input itself MAC Address.}
		\item \texttt{12913 [QoS]MAC Device List should not allow 2 bytes characters \\
						for QoS Policy and device name}
		\item \texttt{12979 some settings and GUI are incorrect and you didn��t follow \\
						what netgear provided}
		\item \texttt{12938 [Guest Network]Guest Network SSID should not be the same with \\
						SSID on Basic wireless settings page}
		\item \texttt{12946 [Password] If set password to some special characters, \\
						user can\'t log in to the DUT .}
		\item \texttt{13073 When you enable Wireless Repeating function, basic wireless \\
						setting page will show error.}
		\item \texttt{13074 When you input wrong mac address, the page will show a javascript.}
		\item \texttt{12991 [wds]Should add the ":"when read the MAC Address.}
		\item \texttt{12819 The web page of advanced wireless settings should follow netgear\'s requirement.}
		\item \texttt{12955 web page doesn't reboot wireless when disable WDS}
		\item \texttt{13054 [QoS]According to Netgear UI spec 2.1,please remove \\
						"Diablo II","Half Life" and "Wolfenstein" in "on line gaming" list}
		\item \texttt{13061 [Wireless Settings]In WPA/WPA2 Enterprise Security mode, \\
						please change WPA Mode:"WPA[AES]"to "WPA2[AES]" }
	\end{itemize}
\end{enumerate}
\end{itemize}

\subsection{Steps to burn boot loader and firmware}
\begin{itemize}
\item As below:
        \begin{enumerate}
                \item \texttt{Please burn u-boot-1000V0.7.bin}
                \item \texttt{Set up a tftp server on your PC, its ip address is 192.168.1.12.}
                \item \texttt{Entering into boot loader}
                \item \texttt{ag7100> set serverip 192.168.1.12}
                \item \texttt{ag7100> tftp 0x80010000 u-boot-1000V0.7.bin}
		\item \texttt{ag7100> erase 0xbf000000 +0x50000}
		\item \texttt{ag7100> cp.b 0x80010000 0xbf000000 0x50000}
                \item \texttt{ag7100> set bootcmd 'fsload 80800000 image/uImage;bootm 80800000'}
                \item \texttt{ag7100> saveenv}
                \item \texttt{ag7100> reset}
                \item \texttt{Entering into boot loader again}
                \item \texttt{ag7100>bootm}
                \item \texttt{Then the device should be in tftp recovery mode. Please run the coammdn "tftp -i 192.168.1.1 put WNR1000v2-V1.0.1.7.img" on MS-DOS of your PC.}
        \end{enumerate}
\end{itemize}

\subsection{Known issues}
        \begin{enumerate}
                \item \texttt{This version has been updated to Atheros LSDK 7.3.0.195.}
        \end{enumerate}

\section{Firmware V1.0.1.8}

\tlabel{sec:1-0-1}
\subsection{Repository}
\begin{itemize}
\item GIT Repository  itgserver/pub/scm/openwrt/ronger/openwrt.git
\begin{itemize}
    \item Branch: \texttt{sh\_sw\_one\_br}
    \item Tag: \texttt{WNR1000v2-V-1-0-1-8}
\end{itemize}
\end{itemize}

\subsection{Fixed Bugs}
\begin{itemize}
\item As below:
\begin{enumerate}
	\item APPLICATION
	\begin{itemize}
		\item   \texttt{12744 [Disk Management--Shared Folder]We can not create folder with special folder name \\
						in USB storage}
		\item   \texttt{13022 [Disk Management]I can create new folder in \\
						\/tmp\/mnt\/ of the DUT}
		\item   \texttt{12598 [Disk Management--Shared folder]Please define the maxlength \\
						of folder name in UI}
		\item   \texttt{13190 Should replace the command find with other command when scan \\
						disk information file.}
	\end{itemize}
\end{enumerate}
\begin{enumerate}
	\item GUI
	\begin{itemize}
		\item   \texttt{12937 Web pages of ftp server should be added}
		\item   \texttt{12936 web pages of UPnp media server should be added}
		\item   \texttt{13052 [WDS]Please remove the more \) on router status \\
						page when Wireless Repeater mode}
		\item   \texttt{13110 [WDS]When WDS is enabled,"Delete" button on LAN \\
						Setup page does not gray out}
		\item   \texttt{12789 [QoS]Please show MAC Device List according to Netgear's spec.}
		\item   \texttt{12938 [Guest Network]Guest Network SSID should not be the same \\
						with SSID on Basic wireless settings page}
		\item   \texttt{13023 [Advanced wireless settings]Please remove "Enable WMM" item on \\
						the help page of advanced wireless settings}
		\item   \texttt{13046 [WDS][802.11a\/n]I can apply WDS when security is WPA-PSK or \\
						WPA2-PSK in a\/n mode}
		\item   \texttt{13049 [WDS]When WDS is enabled,menu shows error}
		\item   \texttt{13050 [WDS]When WDS is enabled,Basic settings still work}
		\item   \texttt{13047 [WDS]I can apply WDS when security is WPA\/WPA2 Enterprise}
		\item   \texttt{12819 The web page of advanced wireless settings should follow netgear\'s requirement.}
		\item   \texttt{12382 [WPS]When you add WPS client successful with PIN mode,the webpages shows timeout}
		\item   \texttt{12381 [WPS]When you add WPS client,the webpages should according to Netgear's router spec1.6}
	\end{itemize}
\end{enumerate}
\end{itemize}

\subsection{Steps to burn boot loader and firmware}
\begin{itemize}
\item As below:
        \begin{enumerate}
                \item \texttt{Please burn u-boot-1000V0.7.bin}
                \item \texttt{Set up a tftp server on your PC, its ip address is 192.168.1.12.}
                \item \texttt{Entering into boot loader}
                \item \texttt{ag7100> set serverip 192.168.1.12}
                \item \texttt{ag7100> tftp 0x80010000 u-boot-1000V0.7.bin}
		\item \texttt{ag7100> erase 0xbf000000 +0x50000}
		\item \texttt{ag7100> cp.b 0x80010000 0xbf000000 0x50000}
                \item \texttt{ag7100> set bootcmd 'fsload 80800000 image/uImage;bootm 80800000'}
                \item \texttt{ag7100> saveenv}
                \item \texttt{ag7100> reset}
                \item \texttt{Entering into boot loader again}
                \item \texttt{ag7100>bootm}
                \item \texttt{Then the device should be in tftp recovery mode. Please run the coammdn "tftp -i 192.168.1.1 put WNR1000v2-V1.0.1.8.img" on MS-DOS of your PC.}
        \end{enumerate}
\end{itemize}

\subsection{Known issues}
        \begin{enumerate}
                \item \texttt{This version has been updated to Atheros LSDK 7.3.0.235.}
        \end{enumerate}

\section{Firmware V1.0.1.9}

\tlabel{sec:1-0-1}
\subsection{Repository}
\begin{itemize}
\item GIT Repository  itgserver/pub/scm/openwrt/ronger/openwrt.git
\begin{itemize}
    \item Branch: \texttt{sh\_sw\_one\_br}
    \item Tag: \texttt{WNR1000v2-V-1-0-1-9}
\end{itemize}
\end{itemize}

\subsection{Fixed Bugs}
\begin{itemize}
\item As below:
\begin{enumerate}
	\item APPLICATION
	\begin{itemize}
		\item 	\texttt{12864 ftp should be supported}
		\item 	\texttt{13105 [SOAP]The PPTP mode should check the ip conflict.}
		\item 	\texttt{13129 [SOAP]If wireless is disable, after running SOAP, it should enable wireless}
		\item 	\texttt{13157 [USB storage]If you want to save the file you edit in shared folder, \\
						a warning message "Not enough storage" pop up and the file can not be saved.}
		\item 	\texttt{13327 [DHCP server]If subnet is changed and the subnet range is from big to small, \\
						router SHOULD flush DMZ and Block sites Trusted IP Address}
		\item 	\texttt{12851 [USB storage]Remove the disk without using "safely remove disk" and then insert it, \\
				sometimes it can not mount all the disks}
	\end{itemize}
\end{enumerate}
\begin{enumerate}
	\item WIRELESS
	\begin{itemize}
		\item 	\texttt{13012 [Wireless Settings]In Wireless Network a\/n mode ,can not set the wep Security Option .}
		\item 	\texttt{13203 [Wireless settings][a\/n][b\/g\/n]If there are spaces in WPA2-PSK PassPhrase, \\
						it shows OK in UI, but "invalid WPA passphrase length" in console}
		\item 	\texttt{13202 [Wireless settings][a\/n][b\/g\/n]If there are spaces in WPA-PSK PassPhrase, \\
						it shows OK in UI, but "invalid WPA passphrase length" in console}
		\item 	\texttt{13204 [Wireless settings][a\/n][b\/g\/n]If there are spaces in WPA/WPA2-PSK PassPhrase, \\
						it shows OK in UI, but "invalid WPA passphrase length" in console}
		\item 	\texttt{13170 [Wireless settings][a\/n][b\/g\/n]If there are spaces in SSID, the characters \\
						after spaces disappear in iwconfig}
		\item 	\texttt{13152 [Advance wireless settings]Uncheck "Enable SSID Broadcast" in a\/n mode does not work.}
		\item 	\texttt{12083 [Basic Wireless settings][a\/n]Wireless client WNDA3100 can scan and connnect \\
						to the DUT only when the DUT works in channel 36~48,149~165}
		\item 	\texttt{13230 [Guest Network][a\/n][b\/g\/n]If there are spaces in SSID, the characters after \\
						spaces disappear in iwconfig}
		\item 	\texttt{13164 [Guest Network & Wireless ACL]Wireless ACL can not work with gust network}
		\item 	\texttt{12092 [Basic Wireless settings][a\/n][b\/g\/n]40MHz can only works in no security mode.}
		\item 	\texttt{13014 [Wireless Settings]In Wireless Network a\/n mode ,the "Mode" display wrong in \\
						the wireless card sometimes.}
		\item 	\texttt{12706 [WPS]According to Netgear's spec, Auto WEP should support WPS}
		\item 	\texttt{12373 [WPS]Configure APUT using PIN method through Vista, if I set security mode as WPA2-PSK, \\
						it shows WPA-PSK in both console and UI.}
		\item 	\texttt{13331 [wps]The Name (SSID) /Security Options display wrong in UI.}
		\item 	\texttt{13224 [LED][Wireless]Wireless LED is always light off.}
	\end{itemize}
\end{enumerate}
\begin{enumerate}
	\item GUI
	\begin{itemize}
		\item   \texttt{12937 Web pages of ftp server should be added}
		\item   \texttt{12936 web pages of UPnp media server should be added}
		\item   \texttt{13052 [WDS]Please remove the more \) on router status \\
						page when Wireless Repeater mode}
		\item   \texttt{13110 [WDS]When WDS is enabled,"Delete" button on LAN \\
						Setup page does not gray out}
		\item   \texttt{12789 [QoS]Please show MAC Device List according to Netgear's spec.}
		\item   \texttt{12938 [Guest Network]Guest Network SSID should not be the same \\
						with SSID on Basic wireless settings page}
		\item   \texttt{13023 [Advanced wireless settings]Please remove "Enable WMM" item on \\
						the help page of advanced wireless settings}
		\item   \texttt{13046 [WDS][802.11a\/n]I can apply WDS when security is WPA-PSK or \\
						WPA2-PSK in a\/n mode}
		\item   \texttt{13049 [WDS]When WDS is enabled,menu shows error}
		\item   \texttt{13050 [WDS]When WDS is enabled,Basic settings still work}
		\item   \texttt{13047 [WDS]I can apply WDS when security is WPA\/WPA2 Enterprise}
		\item   \texttt{12819 The web page of advanced wireless settings should follow netgear\'s requirement.}
		\item   \texttt{12382 [WPS]When you add WPS client successful with PIN mode,the webpages shows timeout}
		\item   \texttt{12381 [WPS]When you add WPS client,the webpages should according to Netgear's router spec1.6}
	\end{itemize}
\end{enumerate}
\end{itemize}

\subsection{Steps to burn boot loader and firmware}
\begin{itemize}
\item As below:
        \begin{enumerate}
                \item \texttt{Please burn u-boot-1000V0.7.bin}
                \item \texttt{Set up a tftp server on your PC, its ip address is 192.168.1.12.}
                \item \texttt{Entering into boot loader}
                \item \texttt{ag7100> set serverip 192.168.1.12}
                \item \texttt{ag7100> tftp 0x80010000 u-boot-1000V0.7.bin}
		\item \texttt{ag7100> erase 0xbf000000 +0x50000}
		\item \texttt{ag7100> cp.b 0x80010000 0xbf000000 0x50000}
                \item \texttt{ag7100> set bootcmd 'fsload 80800000 image/uImage;bootm 80800000'}
                \item \texttt{ag7100> saveenv}
                \item \texttt{ag7100> reset}
                \item \texttt{Entering into boot loader again}
                \item \texttt{ag7100>bootm}
                \item \texttt{Then the device should be in tftp recovery mode. Please run the command "tftp -i 192.168.1.1 put WNR1000v2-V1.0.1.9.img" on MS-DOS of your PC.}
        \end{enumerate}
\end{itemize}

\subsection{Known issues}
        \begin{enumerate}
                \item \texttt{This version has been updated to Atheros LSDK 7.3.0.235.}
        \end{enumerate}

\section{Firmware V1.0.2.0}

\tlabel{sec:1-0-1}
\subsection{Repository}
\begin{itemize}
\item GIT Repository  itgserver/pub/scm/openwrt/ronger/openwrt.git
\begin{itemize}
    \item Branch: \texttt{sh\_sw\_one\_br}
    \item Tag: \texttt{WNR1000v2-V-1-0-2-0}
\end{itemize}
\end{itemize}

\subsection{Fixed Bugs}
\begin{itemize}
\item As below:
\begin{enumerate}
	\item BOOTLOADER
	\begin{itemize}
		\item   \texttt{13407 Added new option in the command macset and macshow to configure 5GHz MAC address} 
	\end{itemize}
\end{enumerate}
\begin{enumerate}
	\item APPLICATION
	\begin{itemize}
		\item   \texttt{13218 The disk re-mouted info is not correct in "df" table \\
						if remove HDD not correctly.}
		\item   \texttt{13366 could not display all the partitions in the df table , \\
						but actually there are some other partitions should be displayed.}
		\item   \texttt{13141 2.4GHz wireless broadcast could not be verified with \\
						consistent results for the current device firmware.}
		\item   \texttt{13176 Sometimes wireless arguments cannot be set through smart wizard}
		\item   \texttt{13413 List folders through http server.}
		\item   \texttt{13336 [WAN Setup]the WAN MTU can't work correctly!}
		\item   \texttt{13309 [Advance wireless settings][a\/n]Short Preamble does \\
						not work with an mode}
		\item   \texttt{12883 [Guest Network][b\/g\/n]When Enable Guest Network \\
						b\/g\/n and auto channel, 40MHz does not work .}
		\item \texttt{13064 [QoS]QoS can not work under PPPoE}
	\end{itemize}
\end{enumerate}
\begin{enumerate}
	\item WIRELESS
	\begin{itemize}
		\item   \texttt{13339 [wps]PIN Number function fail.}
		\item   \texttt{13328 [wps]Fail to Configure APUT using PIN method through a \\
						WLAN external Registrar}
		\item   \texttt{13332 [wps]Both the a/n and b/g/n should be configed accross \\
						the wired external registrar}
		\item   \texttt{13245 [LED]When use WPS ,the LED in the rear panel can't flash !}
		\item   \texttt{13231 [Guest Network][a\/n][b\/g\/n]If I set SSID as some special \\
						characters. It shows no wireless in iwconfig}
		\item   \texttt{13348 [wireless repeating function]when you set the DUT as \\
						"repeater" mode,it will crash!}
		\item   \texttt{13405 Use fixed MAC of LAN and 11g as br0 MAC}
		\item   \texttt{13235 [Guest Network][b\/g\/n][a\/n]When you set Open WEP or \\
						Auto WEP, it works as no security}
		\item   \texttt{13103 br0's MAC changed after wireless restart sometimes}
	\end{itemize}
\end{enumerate}
\begin{enumerate}
	\item GUI
	\begin{itemize}
		\item   \texttt{13421 Disable UPNP media server and FTP server on GUI}
		\item   \texttt{13261 [Static Route]When the DUT use "Use Static IP Address" as \\
						WAN IP address and there is not cable connect to the DUT's WAN. \\
						We can not add static route in WAN subnet.}
		\item   \texttt{13359 [wireless settings]After apply, "RADIUS server Shared Secret" \\
						on the webpage always shows NULL}
		\item   \texttt{13360 IP address should not be the Broadcast address or the subnet.}
		\item   \texttt{13403 WPA/WPA2 enterprise security can not be used with WPS}
		\item   \texttt{13238 [Guest Network][b\/g\/n][a\/n]When you select "Shared key WEP" \\
						and apply, "Add WPS Client" link in the menu can still click.}
		\item   \texttt{13239 [Block Services ]Change the Service Type ,can put many Services \\
						with the same Port/IP}
		\item   \texttt{13414 Multi-language should be disabled}
		\item   \texttt{13244 [QoS - Priority rules]Please change "Yahoo messanger" to \\
						"Yahoo messenger"}
		\item   \texttt{13257 After restore to default, the web page cannot be displayed}
		\item   \texttt{13317 if the WDS is enabled, then the Guest Network must be disable}
		\item   \texttt{13292 [QoS]When edit the qos policy the starting port and the ending \\
						port will display blank}
		\item   \texttt{13242 [Guest Network][b\/g\/n][a\/n]When you select "WPA-PSK [TKIP]", \\
						"WPA2-PSK [AES]", "WPA-PSK [TKIP] + WPA2-PSK [AES]", it can not be saved config. \\
						So it can not work.}
		\item   \texttt{13222 [Port Forwarding ]The Protocol sequence display wrong in Ports - \\
						Custom Services page}
		\item   \texttt{13427 [CA]There is a javascript error in BSW\_pppoe.htm page}
		\item   \texttt{13428 [CA]When you apply in BSW\_pppoe.htm page, the password can not be saved config.}
		\item   \texttt{13430 [CA]The top icon of CA still shows WNR2000}
		\item   \texttt{13431 [CA]The left menu shows wrong in firefox in CA}
		\item   \texttt{13432 [CA]Please check all the value in BSW\_pppoe.htm page}
		\item   \texttt{13436 [CA]When it detect as BPA, the webpage shows error}
		\item   \texttt{13437 [CA]When you click the link "Back to internet settings page",it shows blank.}
	\end{itemize}
\end{enumerate}
\end{itemize}

\subsection{Steps to burn boot loader and firmware}
\begin{itemize}
\item As below:
        \begin{enumerate}
                \item \texttt{Please burn u-boot-1000V0.8.bin}
                \item \texttt{Set up a tftp server on your PC, its ip address is 192.168.1.12.}
                \item \texttt{Entering into boot loader}
                \item \texttt{ag7100> set serverip 192.168.1.12}
                \item \texttt{ag7100> tftp 0x80010000 u-boot-1000V0.8.bin}
		\item \texttt{ag7100> erase 0xbf000000 +0x50000}
		\item \texttt{ag7100> cp.b 0x80010000 0xbf000000 0x50000}
                \item \texttt{ag7100> set bootcmd 'fsload 80800000 image/uImage;bootm 80800000'}
                \item \texttt{ag7100> saveenv}
                \item \texttt{ag7100> reset}
                \item \texttt{Entering into boot loader again}
                \item \texttt{ag7100>bootm}
                \item \texttt{Then the device should be in tftp recovery mode. Please run the command "tftp -i 192.168.1.1 put WNR1000v2-V1.0.2.0.img" on MS-DOS of your PC.}
        \end{enumerate}
\end{itemize}

\subsection{Known issues}
        \begin{enumerate}
                \item \texttt{This version has been updated to Atheros LSDK 7.3.0.235.}
        \end{enumerate}

\section{Firmware V1.0.2.1}

\tlabel{sec:1-0-1}
\subsection{Repository}
\begin{itemize}
\item GIT Repository  itgserver/pub/scm/openwrt/ronger/openwrt.git
\begin{itemize}
    \item Branch: \texttt{sh\_sw\_one\_br}
    \item Tag: \texttt{WNR1000v2-V-1-0-2-1}
\end{itemize}
\end{itemize}

\subsection{Fixed Bugs}
\begin{itemize}
\item As below:
\begin{enumerate}
	\item APPLICATION
	\begin{itemize}
		\item 	\texttt{13465 SIP failed with nokia e70}
		\item 	\texttt{13365 [Firmware Upgrade]Too large corrupted image can make router out of memory}
		\item 	\texttt{13244 [QoS - Priority rules]Please change "Yahoo messanger" to "Yahoo messenger"}
	\end{itemize}
\end{enumerate}
\begin{enumerate}
	\item WIRELESS
	\begin{itemize}
		\item 	\texttt{13426 [Wireless]54M in an mode can not work.}
		\item 	\texttt{13474 [Advanced Wireless Settings]Should add the pin number}
		\item 	\texttt{13206 we need to change the command that GUI uses to activate WPS.}
		\item 	\texttt{12567 one WPS daemon which supports two configuration files for only one hostapd}
		\item 	\texttt{13184 [Wireless settings][a/n]Because driver support an with Israel and Middle East, \\
						please enable an mode in this two countries in UI}
		\item 	\texttt{13232 [Guest Network][a/n][b/g/n]When I set SSID as "any", it shows NULL in iwconfig}
		\item 	\texttt{13338 [wps]Push Button function fail.}
		\item 	\texttt{13208 [wireless repeating function]when you set the DUT as "base station" mode,it can't work correctly!}
		\item 	\texttt{13210 [Wireless settings][a/n][b/g/n]WPA-PSK, WPA2-PSK, WPA/WPA2-PSK PassPhrase with \\
						64 hexdigits can not work}
	\end{itemize}
\end{enumerate}
\begin{enumerate}
	\item GUI
	\begin{itemize}
		\item 	\texttt{13330 [Address Reservation]If DHCP range has changed, you should flush the reservation \\
						rules that are out of the range.}
		\item 	\texttt{13508 Knowledg base and Domentation URL should be added}
		\item 	\texttt{13490 [Address reservation]After you add a address reservation,there is something wrong \\
						with the webpage}
		\item 	\texttt{13495 Default wireless mode and channel should be changed}
		\item 	\texttt{13483 [Remote management]IP address on Remote management page should not be in the \\
						same subnet with LAN IP address.}
		\item 	\texttt{13453 [Static route]The gateway in static route should be in the same subnet with \\
						DUT's LAN or WAN, not the Destination IP Address}
		\item 	\texttt{13320 [Backup Settings]when i use firefox to restore the DUT,it can't work!}
		\item 	\texttt{13317 if the WDS is enabled, then the Guest Network must be disable}
	\end{itemize}
\end{enumerate}
\end{itemize}

\subsection{Steps to burn boot loader and firmware}
\begin{itemize}
\item As below:
        \begin{enumerate}
                \item \texttt{Please burn u-boot-1000V0.9.bin}
                \item \texttt{Set up a tftp server on your PC, its ip address is 192.168.1.12.}
                \item \texttt{Entering into boot loader}
                \item \texttt{ag7100> set serverip 192.168.1.12}
                \item \texttt{ag7100> tftp 0x80010000 u-boot-1000V0.9.bin}
				\item \texttt{ag7100> erase 0xbf000000 +0x50000}
				\item \texttt{ag7100> cp.b 0x80010000 0xbf000000 0x50000}
                \item \texttt{ag7100> set bootcmd 'fsload 80800000 image/uImage;bootm 80800000'}
                \item \texttt{ag7100> saveenv}
                \item \texttt{ag7100> reset}
                \item \texttt{Entering into boot loader again}
                \item \texttt{ag7100>bootm}
                \item \texttt{Then the device should be in tftp recovery mode. Please run the command "tftp -i 192.168.1.1 put WNR1000v2-V1.0.2.1.img" on MS-DOS of your PC.}
        \end{enumerate}
\end{itemize}

\subsection{Known issues}
        \begin{enumerate}
                \item \texttt{This version has been updated to Atheros LSDK 7.3.0.235.}
        \end{enumerate}

\section{Firmware V1.0.2.2}

\tlabel{sec:1-0-1}
\subsection{Repository}
\begin{itemize}
\item GIT Repository  itgserver/pub/scm/openwrt/ronger/openwrt.git
\begin{itemize}
    \item Branch: \texttt{sh\_sw\_one\_br}
    \item Tag: \texttt{WNR1000v2-V-1-0-2-2}
\end{itemize}
\end{itemize}

\subsection{Fixed Bugs}
\begin{itemize}
\item As below:
\begin{enumerate}
	\item APPLICATION
	\begin{itemize}
		\item 	\texttt{13558 [USB storage]We can access http:\/\/192.168.1.1\/shares via http \
						even HTTP is disabled.}
		\item 	\texttt{13567 [USB storage]Capacity on "USB Drive Approved Devices" page is small \
						than Total Space on "USB NETStorage (Basic Settings)" page}
		\item 	\texttt{13559 [USB storage]The account "admin" should use the same password as the \
						admin GUI password at all time}
		\item 	\texttt{13561 [USB storage]When Volume Name is empty, it shows "No Volume" on the \
						webpage.}
		\item 	\texttt{13631 [USB storage]Get Volume name with wrong order}
		\item 	\texttt{13636 [USB storage]Can not get the Volume Name of ntfs}
	\end{itemize}
\end{enumerate}
\begin{enumerate}
	\item WIRELESS
	\begin{itemize}
		\item 	\texttt{13389 [WMM]WMM can not work.}	
		\item 	\texttt{13332 [wps]Both the a\/n and b\/g\/n should be configed accross the wired \
						external registrar}
	\end{itemize}
\end{enumerate}
\begin{enumerate}
	\item GUI
	\begin{itemize}
		\item 	\texttt{13566 [USB storage]Shared folder can not be deleted.}
		\item 	\texttt{13579 [USB storage]"Create Network Folder" button should takes the \
						user to the USB NetStorage - Advanced Settings - Create Network Folder}
		\item 	\texttt{13418 USB Storage should be following new Spec}
		\item 	\texttt{13596 [USB storage]The Button's value on "USB settings" page should be \
						"Approved Devices" not "Browse"}
		\item 	\texttt{13576 [USB storage]The link color on should be the same with spec.}
		\item 	\texttt{13591 [USB storage]The button value on "Browse for Folder" page should \
						be "OK" not "Apply"}
		\item 	\texttt{13592 [USB Storage]The topic "Browse for Folder" is missing on the \
						webpage USB\_browse.htm}
		\item 	\texttt{13593 [USB storage]On USB\_browse.htm, the top branch should be "My Router" not Device Name}

	\end{itemize}
\end{enumerate}
\end{itemize}

\subsection{Steps to burn boot loader and firmware}
\begin{itemize}
\item As below:
        \begin{enumerate}
                \item \texttt{Please burn u-boot-1000V0.9.bin}
                \item \texttt{Set up a tftp server on your PC, its ip address is 192.168.1.12.}
                \item \texttt{Entering into boot loader}
                \item \texttt{ag7100> set serverip 192.168.1.12}
                \item \texttt{ag7100> tftp 0x80010000 u-boot-1000V0.9.bin}
				\item \texttt{ag7100> erase 0xbf000000 +0x50000}
				\item \texttt{ag7100> cp.b 0x80010000 0xbf000000 0x50000}
                \item \texttt{ag7100> set bootcmd 'fsload 80800000 image/uImage;bootm 80800000'}
                \item \texttt{ag7100> saveenv}
                \item \texttt{ag7100> reset}
                \item \texttt{Entering into boot loader again}
                \item \texttt{ag7100>bootm}
                \item \texttt{Then the device should be in tftp recovery mode. Please run the command "tftp -i 192.168.1.1 put WNR1000v2-V1.0.2.2.img" on MS-DOS of your PC.}
        \end{enumerate}
\end{itemize}

\subsection{Known issues}
        \begin{enumerate}
                \item \texttt{This version has been updated to Atheros LSDK 7.3.0.235.}
        \end{enumerate}

		
\section{Firmware V1.0.2.3}

\tlabel{sec:1-0-1}
\subsection{Repository}
\begin{itemize}
\item GIT Repository  itgserver/pub/scm/openwrt/ronger/openwrt.git
\begin{itemize}
    \item Branch: \texttt{sh\_sw\_one\_br}
    \item Tag: \texttt{WNR1000v2-V-1-0-2-3}
\end{itemize}
\end{itemize}

\subsection{Fixed Bugs}
\begin{itemize}
\item As below:
\begin{enumerate}
	\item APPLICATION
	\begin{itemize}
		\item 	\texttt{13621 [USB storage]No matter what Network\/Device Name on \
						USB NETStorage (Advanced Settings) page is ,it always work as WNR1000v2}
		\item 	\texttt{12929 [Router Upgrade]Check for New Version from the Internet \
						can not work.}
		\item 	\texttt{13708 [USB storage]Large file can not be visited from HTTP}
		\item 	\texttt{13721 [USB storage]Total disk space is more larger than WinXP shows, \
						and the free space is almost the same with total.}
	\end{itemize}
\end{enumerate}

\begin{enumerate}
	\item GUI
	\begin{itemize}
		\item 	\texttt{13592 [USB Storage]The topic "Browse for Folder" is \
						missing on the webpage USB\_browse.htm}
		\item 	\texttt{13588 [USB storage][USB storage]Read Access "admin" \
						and Write Access "All - no password" should not be allowed at the same time}
		\item 	\texttt{13573 [USB storage]Even USB Device, Folder and Share Name on \
						"Create Network Folder " or "Edit Network Folder" page are NULL, \
						I can apply the webpage.}
		\item 	\texttt{13624 [USB storage]Share name on USB NETStorage (Advanced Settings) \
						page shows different from USB NETStorage (Basic Settings) page}
		\item 	\texttt{13682 [USB storage]When you edit the share folder, the USB Device and \
						file system should not be edited on "Edit Network Folder" page}
		\item 	\texttt{13667 [USB storage]I can add different folders with the same share name}
		\item 	\texttt{13655 [USB storage]If the Share Name includes special chars, \
						USB NETStorage (Advanced Settings) page shows error}
		\item 	\texttt{13662 [USB storage]When device in Folder changes, USB device does not change}
		\item 	\texttt{13683 [USB storage]When click "Add" or "Delete" button on "USB Drive \
						Approved Devices" page, please check whether there is a device to add or delete}
		\item 	\texttt{13626 [USB storage]Please check the port value for HTTP(via internet)}
		\item 	\texttt{13582 [USB storage]The default value of internet HTTP port is NULL}
		\item 	\texttt{13482 If LAN Subnet is change from 192.168.1.17\/255.255.255.240 to \
						192.168.1.1\/255.255.255.240,we should flush some IP addresses at LAN site.}
		\item 	\texttt{13452 [Wireless ACL]When a wireless client connects to the DUT with guest \
						Network SSID, it should shows in Available Wireless Cards table}
		\item 	\texttt{13714 [mercury-mms] DUT will not reject some multicast MAC address in \
						basic settings page.}
		\item 	\texttt{13716 [mercury-mms] SSID can't contain character "&".}
		\item 	\texttt{13717 [mercury-mms] WPA-PSK/WPA2-PSK passphrase can't contain "&".}
		\item 	\texttt{13710 [mercury-mms]PPPoE cannot get connected if the password is null.}
	\end{itemize}
\end{enumerate}
\end{itemize}

\subsection{Steps to burn boot loader and firmware}
\begin{itemize}
\item As below:
        \begin{enumerate}
                \item \texttt{Please burn u-boot-1000V0.9.bin}
                \item \texttt{Set up a tftp server on your PC, its ip address is 192.168.1.12.}
                \item \texttt{Entering into boot loader}
                \item \texttt{ag7100> set serverip 192.168.1.12}
                \item \texttt{ag7100> tftp 0x80010000 u-boot-1000V0.9.bin}
				\item \texttt{ag7100> erase 0xbf000000 +0x50000}
				\item \texttt{ag7100> cp.b 0x80010000 0xbf000000 0x50000}
                \item \texttt{ag7100> set bootcmd 'fsload 80800000 image/uImage;bootm 80800000'}
                \item \texttt{ag7100> saveenv}
                \item \texttt{ag7100> reset}
                \item \texttt{Entering into boot loader again}
                \item \texttt{ag7100>bootm}
                \item \texttt{Then the device should be in tftp recovery mode. Please run the command "tftp -i 192.168.1.1 put WNR1000v2-V1.0.2.3.img" on MS-DOS of your PC.}
        \end{enumerate}
\end{itemize}

\subsection{Known issues}
        \begin{enumerate}
                \item \texttt{This version has been updated to Atheros LSDK 7.3.0.235.}
        \end{enumerate}

\section{Firmware V1.0.2.4}

\tlabel{sec:1-0-1}
\subsection{Repository}
\begin{itemize}
\item GIT Repository  itgserver/pub/scm/openwrt/ronger/openwrt.git
\begin{itemize}
    \item Branch: \texttt{sh\_sw\_one\_br}
    \item Tag: \texttt{WNR1000v2-V-1-0-2-4}
\end{itemize}
\end{itemize}

\subsection{Fixed Bugs}
\begin{itemize}
\item As below:
\begin{enumerate}
	\item APPLICATION
	\begin{itemize}
		\item 	\texttt{13641 [USB storage]Write access does not work}
		\item 	\texttt{13557 [USB storage]According to New spec,when a \
						user accesses a Network Folder which is assigned to the \
						All no password the user should NOT be prompted by a message \
						asking them to enter any password}
	\end{itemize}
\end{enumerate}

\begin{enumerate}
	\item GUI
	\begin{itemize}
		\item 	\texttt{13657 [USB storage]If there are many spaces in share name, \
						it only shows one in Avaiable Network Folders list}
		\item 	\texttt{13689 [USB storage]Please define appropriate max length for share name.}
		\item 	\texttt{13658 [USB storage]Webpage shows error when share name are some special characters}
		\item 	\texttt{13688 [USB storage]If I set share name as  it shows different in USB storage}
		\item 	\texttt{13668 [USB storage]I can add the same folder with the same access to differnet share names}
		\item 	\texttt{13671 [USB storage]After you edit the share folder in Avaiable Network Folders list, \
						"Edit Network Folder" page changes to "Create Network Folder" page}
		\item 	\texttt{13664 [USB storage]I can create a share name without a real folder}
		\item 	\texttt{13715 [mercury-mms] If disable both 2.4G and 5G wireless radio, in router status \
						page the channel info of the 2 bands are not consistent.}
		\item 	\texttt{13482 If LAN Subnet is change from 192.168.1.17\/255.255.255.240 to \
						192.168.1.1\/255.255.255.240,we should flush some IP addresses at LAN site.}
		\item 	\texttt{13258 [Router Upgrade]i can't locate the upgrade file by manual with FireFox!}
		\item 	\texttt{13738 [mercury-mms] DUT will access WNR2000 page when click "Test" button on "Basic Settings" page.}
	\end{itemize}
\end{enumerate}
\end{itemize}

\subsection{Steps to burn boot loader and firmware}
\begin{itemize}
\item As below:
        \begin{enumerate}
                \item \texttt{Please burn u-boot-1000V0.9.bin}
                \item \texttt{Set up a tftp server on your PC, its ip address is 192.168.1.12.}
                \item \texttt{Entering into boot loader}
                \item \texttt{ag7100> set serverip 192.168.1.12}
                \item \texttt{ag7100> tftp 0x80010000 u-boot-1000V0.9.bin}
				\item \texttt{ag7100> erase 0xbf000000 +0x50000}
				\item \texttt{ag7100> cp.b 0x80010000 0xbf000000 0x50000}
                \item \texttt{ag7100> set bootcmd 'fsload 80800000 image/uImage;bootm 80800000'}
                \item \texttt{ag7100> saveenv}
                \item \texttt{ag7100> reset}
                \item \texttt{Entering into boot loader again}
                \item \texttt{ag7100>bootm}
                \item \texttt{Then the device should be in tftp recovery mode. Please run the command "tftp -i 192.168.1.1 put WNR1000v2-V1.0.2.4.img" on MS-DOS of your PC.}
        \end{enumerate}
\end{itemize}

\subsection{Known issues}
        \begin{enumerate}
                \item \texttt{This version has been updated to Atheros LSDK 7.3.0.235.}
        \end{enumerate}

\section{Firmware V1.0.2.5}

\tlabel{sec:1-0-1}
\subsection{Repository}
\begin{itemize}
\item GIT Repository  itgserver/pub/scm/openwrt/ronger/openwrt.git
\begin{itemize}
    \item Branch: \texttt{sh\_sw\_one\_br}
    \item Tag: \texttt{WNR1000v2-V-1-0-2-5}
\end{itemize}
\end{itemize}

\subsection{Fixed Bugs}
\begin{itemize}
\item As below:
\begin{enumerate}
	\item APPLICATION
	\begin{itemize}
		\item 	\texttt{13581 [USB storage]HTTP via internet can not work.}
		\item 	\texttt{13806 [Netgear-Lancelot]If the share name is ?U drive?, then it will only display \
						\\\\wnr1000v2 on the authentication pop-up dialog not \\\\wnr1000v2 drive}
		\item 	\texttt{13614 [USB storage]Disable Network Connection does not work.}
		\item 	\texttt{13756 [NETGEAR-CEC-1061--POT] Can't access the POT web page \
						http:\/\/www.routerlogin.com/POT.htm}
		\item 	\texttt{13739 [mercury-mms] There are abmormal items in system logs.}
		\item 	\texttt{13713 [mercury-mms] The system logs will be flushed under some \
						circumstances that are not expected to do so.}
		\item 	\texttt{12929 [Router Upgrade]Check for New Version from the Internet can not work.}
		\item 	\texttt{13710 [mercury-mms]PPPoE cannot get connected if the password is null.}
		\item 	\texttt{13712 [mercury-mms] Set as PPPoE with "Dial on demand" mode DUT will always \
						automatically connect to server after apply in basic settings even \
						if there's no Internet traffic from LAN.}
		\item 	\texttt{13823 [USB storage]Change Network\/Device Name on USB NETStorage page \
						without delete the old connection, it will call some problems}
		\item 	\texttt{13708 [USB storage]Large file can not be visited from HTTP}
		\item 	\texttt{13711 [mercury-mms] PPPoE cannot get connected if the user name \
						contains special character.}
	\end{itemize}
\end{enumerate}

\begin{enumerate}
	\item GUI
	\begin{itemize}
		\item 	\texttt{13640 [USB storage]Folder name with many spaces shows only one on \
						"Browse for Folder" page}
		\item 	\texttt{13669 [USB storage]After you edit the share folder in Avaiable Network \
						Folders list,"USB NETStorage (Advanced Settings)" page does not refresh itself}
		\item 	\texttt{13673 [USB storage]In USB spec, "Apply" and "Cancel" buttons do not \
						exist on "USB Drive Approved Devices" page}
		\item 	\texttt{13626 [USB storage]Please check the port value for HTTP(via internet)}
		\item 	\texttt{13578 [USB storage]"Safely Remove USB Device" button now removes all the devices.}
		\item 	\texttt{13687 [USB storage]When you add the second device to Approved USB Devices \
						list fail, Approved USB Devices list become NULL,but the first device is still exist.}
		\item 	\texttt{13805 [NETGEAR-Lancelot] ?Edit? should be grayed out if there is not network folder.}
		\item 	\texttt{13655 [USB storage]If the Share Name is special characters, USB NETStorage \
						(Advanced Settings) page shows error}
		\item 	\texttt{13841 In USB NETStorage (Advanced Settings), the blank Device Name will be forbidden.}
		\item 	\texttt{13570 [USB storage]Even there is no folder is selected, I can apply on USB\_browse.htm}
		\item 	\texttt{13763 [WPS & Guest Network]Since WPS should not work on guest network interface, \
						please remove all the changes in UI.}
		\item 	\texttt{13347 [wps]Push Button display wrong in Firefox\/2.0.0.2}
		\item 	\texttt{13795 [mercury-mms] 5G MAC is the same with 2.4G MAC in WDS setting GUI.}
		\item 	\texttt{13800 [mercury-mms] In the RIP direction list "None" option is no longer needed.}
		\item 	\texttt{13803 [Wireless ACL]When you refresh the WLG\_acl\_add.htm page ,Available \
						Wireless Cards list shows error and become longer.}
		\item 	\texttt{13709 [mercury-mms] Firefox,Netscape and Safari DHCP lease time GUI has issue.}
	\end{itemize}
\end{enumerate}

\begin{enumerate}
	\item WIRELESS
	\begin{itemize}
		\item 	\texttt{13778 [mercury-mms] With some config the 5G and 2.4G \
						radio will be incorrectly turned off.}
		\item 	\texttt{13770 WNR1000v2 5G band WDS doesn't work}
	\end{itemize}
\end{enumerate}
\end{itemize}

\subsection{Steps to burn boot loader and firmware}
\begin{itemize}
\item As below:
        \begin{enumerate}
                \item \texttt{Please burn u-boot-1000V0.9.bin}
                \item \texttt{Set up a tftp server on your PC, its ip address is 192.168.1.12.}
                \item \texttt{Entering into boot loader}
                \item \texttt{ag7100> set serverip 192.168.1.12}
                \item \texttt{ag7100> tftp 0x80010000 u-boot-1000V0.9.bin}
				\item \texttt{ag7100> erase 0xbf000000 +0x50000}
				\item \texttt{ag7100> cp.b 0x80010000 0xbf000000 0x50000}
                \item \texttt{ag7100> set bootcmd 'fsload 80800000 image/uImage;bootm 80800000'}
                \item \texttt{ag7100> saveenv}
                \item \texttt{ag7100> reset}
                \item \texttt{Entering into boot loader again}
                \item \texttt{ag7100>bootm}
                \item \texttt{Then the device should be in tftp recovery mode. Please run the command "tftp -i 192.168.1.1 put WNR1000v2-V1.0.2.5.img" on MS-DOS of your PC.}
        \end{enumerate}
\end{itemize}

\subsection{Known issues}
        \begin{enumerate}
                \item \texttt{This version has been updated to Atheros LSDK 7.3.0.235.}
        \end{enumerate}

\section{Firmware V1.0.2.6}

\tlabel{sec:1-0-1}
\subsection{Repository}
\begin{itemize}
\item GIT Repository  itgserver/pub/scm/openwrt/ronger/openwrt.git
\begin{itemize}
    \item Branch: \texttt{sh\_sw\_one\_br}
    \item Tag: \texttt{WNR1000v2-V-1-0-2-6}
\end{itemize}
\end{itemize}

\subsection{Fixed Bugs}
\begin{itemize}
\item As below:
\begin{enumerate}
	\item APPLICATION
	\begin{itemize}
		\item 	\texttt{13843 [USB storage]If you share a folder with chmod drwxr-xr-x \
						and chown root, write access does not work.}
		\item 	\texttt{13581 [USB storage]HTTP via internet can not work.}
		\item 	\texttt{13069 [NETGEAR-CEC -- 731]DUT can't block TCP SYN scan packets correctly}
		\item 	\texttt{13757 [mercury-mms] DUT will forward all echo\/chargen packets to LAN \
						when config port forwarding}
		\item 	\texttt{13708 [USB storage]Large file can not be visited from HTTP}
	\end{itemize}
\end{enumerate}

\begin{enumerate}
	\item GUI
	\begin{itemize}
		\item 	\texttt{13570 [USB storage]Even there is no folder is selected, \
						I can apply on USB\_browse.htm}
		\item 	\texttt{13886 set the workgroup name, device name and share name length to 20 chars}
	\end{itemize}
\end{enumerate}

\begin{enumerate}
	\item WIRELESS
	\begin{itemize}
		\item 	\texttt{13785 [mercury-mms] When 2.4G WDS is successful, both base station \
						and repeater GUI show that status is disconnection.}
		\item 	\texttt{13788 [mercury-mms] WNR1000v2 2.4G WDS cannot work with itself \
						if security is WEP}
		\item 	\texttt{13761 [mercury-mms] 5G cannot be scanned if enable 5G guest network.}
	\end{itemize}
\end{enumerate}
\end{itemize}

\subsection{Steps to burn boot loader and firmware}
\begin{itemize}
\item As below:
        \begin{enumerate}
                \item \texttt{Please burn u-boot-1000V0.9.bin}
                \item \texttt{Set up a tftp server on your PC, its ip address is 192.168.1.12.}
                \item \texttt{Entering into boot loader}
                \item \texttt{ag7100> set serverip 192.168.1.12}
                \item \texttt{ag7100> tftp 0x80010000 u-boot-1000V0.9.bin}
				\item \texttt{ag7100> erase 0xbf000000 +0x50000}
				\item \texttt{ag7100> cp.b 0x80010000 0xbf000000 0x50000}
                \item \texttt{ag7100> set bootcmd 'fsload 80800000 image/uImage;bootm 80800000'}
                \item \texttt{ag7100> saveenv}
                \item \texttt{ag7100> reset}
                \item \texttt{Entering into boot loader again}
                \item \texttt{ag7100>bootm}
                \item \texttt{Then the device should be in tftp recovery mode. Please run the command "tftp -i 192.168.1.1 put WNR1000v2-V1.0.2.6.img" on MS-DOS of your PC.}
        \end{enumerate}
\end{itemize}

\subsection{Known issues}
        \begin{enumerate}
                \item \texttt{This version has been updated to Atheros LSDK 7.3.0.235.}
        \end{enumerate}

\section{Firmware V1.0.2.7}

\tlabel{sec:1-0-1}
\subsection{Repository}
\begin{itemize}
\item GIT Repository  itgserver/pub/scm/openwrt/ronger/openwrt.git
\begin{itemize}
    \item Branch: \texttt{sh\_sw\_one\_br}
    \item Tag: \texttt{WNR1000v2-V-1-0-2-7}
\end{itemize}
\end{itemize}

\subsection{Fixed Bugs}
\begin{itemize}
\item As below:
\begin{enumerate}
	\item APPLICATION
	\begin{itemize}
		\item 	\texttt{13852 Once the storage is plugged, then the whole disk must \
						be mounted automatically.}
		\item 	\texttt{13614 [USB storage]Disable Network Connection does not work.}
		\item 	\texttt{13889 [mercury-mms] Change Seagate EXT HDD volume label as \
						"Seagate", it still display as U:(No Volume)}
		\item 	\texttt{13920 The whole disk cannot be shared in V1.0.2.5}
		\item 	\texttt{13069 [NETGEAR-CEC -- 731]DUT can't block TCP SYN scan packets correctly}
		\item 	\texttt{13757 [mercury-mms] DUT will forward all echo/chargen packets to \
						LAN when config port forwarding}
		\item 	\texttt{13708 [USB storage]Large file can not be visited from HTTP}
		\item 	\texttt{13740 [mercury-mms] Set as "RIP\_2M" DUT will not send out any RIP packet.}
		\item 	\texttt{13895 dnsmasq has no response sometimes.}
		\item 	\texttt{13741 [mercury-mms] Ixia simulated 253 DHCP clients cannot get IP address from DUT.}
		\item 	\texttt{13955 [SIP]X-Lite conference fail.}
		\item 	\texttt{13775 [mercury-mms] UPNP Xbox live full test is fail}
	\end{itemize}
\end{enumerate}

\begin{enumerate}
	\item GUI
	\begin{itemize}
		\item 	\texttt{13873 Should follow Lancelot's format in shared folder info}
		\item 	\texttt{13578 [USB storage]"Safely Remove USB Device" button now \
						removes all the devices.}
		\item 	\texttt{13707 [USB storage]If there are sevral USB devices connect to \
						the DUT, Device Names on "USB Drive Approved Devices" page are wrong}
		\item 	\texttt{13051 [WDS]When WDS is enabled and works in Wireless Base Station \
						mode,please show router status page as NETGEAR spec.}
		\item 	\texttt{13906 [NETGEAR-1139--Address Reservation] DUT is case sensitive to \
						letter in mac address if add address reservation items. -- copy from WPN824v3}
		\item 	\texttt{13907 [QoS]We can not add MAC rule to MAC Device List}
	\end{itemize}
\end{enumerate}

\begin{enumerate}
	\item WIRELESS
	\begin{itemize}
		\item 	\texttt{13785 [mercury-mms] When 2.4G WDS is successful, both base \
						station and repeater GUI show that status is disconnection.}
		\item 	\texttt{13788 [mercury-mms] WNR1000v2 2.4G WDS cannot work with itself if \
						security is WEP}
		\item 	\texttt{13774 [mercury-mms] When WPS process is successful, the security \
						is not WPA and WPA2 mixed mode.}
		\item 	\texttt{13847 [WPS]Use WNDA3100 test "WPS\_IOT",the UI display wrong.}
		\item 	\texttt{13782 [mercury-mms] When 5G band WPS process is done,"Keep \
						Existing Wireless Settings" checkbox won't be checked}
		\item 	\texttt{13627 [wps]When use PIN to config the DUT ,both \
						the a\/n and b\/g\/n should be configed.}
	\end{itemize}
\end{enumerate}
\end{itemize}

\subsection{Steps to burn boot loader and firmware}
\begin{itemize}
\item As below:
        \begin{enumerate}
                \item \texttt{Please burn u-boot-1000V1.0.bin}
                \item \texttt{Set up a tftp server on your PC, its ip address is 192.168.1.12.}
                \item \texttt{Entering into boot loader}
                \item \texttt{ag7100> set serverip 192.168.1.12}
                \item \texttt{ag7100> tftp 0x80010000 u-boot-1000V1.0.bin}
				\item \texttt{ag7100> erase 0xbf000000 +0x50000}
				\item \texttt{ag7100> cp.b 0x80010000 0xbf000000 0x50000}
                \item \texttt{ag7100> set bootcmd 'fsload 80800000 image/uImage;bootm 80800000'}
                \item \texttt{ag7100> saveenv}
                \item \texttt{ag7100> reset}
                \item \texttt{Entering into boot loader again}
                \item \texttt{ag7100>bootm}
                \item \texttt{Then the device should be in tftp recovery mode. Please run the command "tftp -i 192.168.1.1 put WNR1000v2-V1.0.2.7.img" on MS-DOS of your PC.}
        \end{enumerate}
\end{itemize}

\subsection{Known issues}
        \begin{enumerate}
                \item \texttt{This version has been updated to Atheros LSDK 7.3.0.279.}
        \end{enumerate}

\section{Firmware V1.0.2.8}

\tlabel{sec:1-0-1}
\subsection{Repository}
\begin{itemize}
\item GIT Repository  itgserver/pub/scm/openwrt/ronger/openwrt.git
\begin{itemize}
    \item Branch: \texttt{sh\_sw\_one\_br}
    \item Tag: \texttt{WNR1000v2-V-1-0-2-8}
\end{itemize}
\end{itemize}

\subsection{Fixed Bugs}
\begin{itemize}
\item As below:
\begin{enumerate}
	\item APPLICATION
	\begin{itemize}
		\item 	\texttt{14070 [samba]error time to access samba}
		\item 	\texttt{14049 [USB Storage]Share folders in different USB devices \
						may have the same share name.}
		\item 	\texttt{14119 [samba]update\_smb file should surport the lock adding}
		\item 	\texttt{13981 [SQA Taipei-50]DUT cannot be tested by xBox testing tool.}
		\item 	\texttt{13986 [SQA Taipei-54]DUT does not hijack DNS qurey packets \
						when destination is other DNS server rather than DUT's LAN IP.}
		\item 	\texttt{13988 [SQA Taipei-56][BPA]Under Bigpond mode, if we fill in \
						"test.com" in the authentication server field,DUT will send \
						out packets asking "test.com.vic.bigpond.net.au".}
		\item 	\texttt{14006 [SQA Taipei-72]Can't use POT tools get or set the POT value.}
		\item 	\texttt{14002 [SQA Taipei-69]User on the LAN side can login DUT \
						through WAN IP address.}
		\item 	\texttt{13997 [SQA Taipei-65]The log doesn't be deleted after the DUT \
						sends out the log by e-mail successfully.}
		\item 	\texttt{13995 [SQA Taipei-63]If DUT WAN port mode is a PPTP and my address \
						is obtained through DHCP mode, DUT will not send out DHCP discover \
						packets every 5 minutes when performing following steps.}
		\item 	\texttt{13982 [SQA Taipei-51][Port triggering]The port trigger could not \
						use the same outgoing port to tirigger different port rules}
		\item 	\texttt{13990 [SQA Taipei-58]Something wrong when PPTP server field is an FQDN address}
		\item 	\texttt{13985 [SQA Taipei-53]Change DUT LAN IP the static router table is NULL.}
		\item 	\texttt{14123 [net-web] Sometimes http server needs the user to input the \
						password twice times.}
		\item 	\texttt{13972 [SQA Taipei-41]If PC within LAN sends out packets containig \
						strict route, DUT will not let this PC send packets to WAN.}
		\item 	\texttt{13984 [SQA Taipei-52][Static Route]If DUT WAN port obtains an IP \
						address, the static route for LAN side is not}
		\item 	\texttt{13910 [USB storage]When download file from USB storage, it will \
						call a lot of memory.}
		\item 	\texttt{14015 [USB storage]About wireless webpage timeout when writing \
						large file from PC to USB disk via samba}
		\item 	\texttt{14137 [qos] can not set Uplink bandwidth correctly \
						when the value is up than 80Mbps}
	\end{itemize}
\end{enumerate}

\begin{enumerate}
	\item GUI
	\begin{itemize}
		\item 	\texttt{13637 [USB storage]Select "No" to "Enable any USB Device connected \
						to the USB port", reboot the DUT, the disk not in "Approved USB Devices" \
						list will also be monuted.}
		\item 	\texttt{13992 [SQA Taipei-60]If PPTP My IP Address is NULL, the gateway IP can't input.}
		\item 	\texttt{14034 [SQA Taipei-91]11a/n should not show message about \
						"channel settings cannot be used auto" when configure 11b\/g\/n \
						wireless repeating function.}
		\item 	\texttt{14025 [SQA Taipei-82]Region of USA and Japan should not display \
						100,104,108,112,116,120,124,128,132,136,140 channels.}
		\item 	\texttt{14080 [mercury-mms] Defect ID : 108 [GUI] Can the Help \
						Frame be made narrower}
		\item 	\texttt{14078 [mercury-mms] Defect ID : 116 [USB] HTTP (via internet) \
						link should reflect current ddns domain.}
		\item 	\texttt{13930 [WDS]When 5G WDS is enabled, Repeater IP Address does not work.}
		\item 	\texttt{14046 [SQA Taipei-103]802.11b/g/n WMM function is always in enabled state.}
		\item 	\texttt{14047 [SQA Taipei-104]802.11a/n WMM function is always in disabled state.}
		\item 	\texttt{14055 GUI should follow Lancelot's wps requirement}
		\item 	\texttt{13714 [mercury-mms] DUT will not reject some multicast MAC \
						address in basic settings page.}
		\item 	\texttt{13715 [mercury-mms] If disable both 2.4G and 5G wireless radio, \
						in router status page the channel info of the 2 bands are not \
						consistent.}
		\item 	\texttt{14069 [USB NETStorage]after change shared folder info format, \
						http link can't open}
	\end{itemize}
\end{enumerate}

\begin{enumerate}
	\item WIRELESS
	\begin{itemize}
		\item 	\texttt{14039 [SQA Taipei-96]The DUT doesn't support WPS external registrar. 
						Wired external registrar cannot configure DUT through WPS-PIN method.}
		\item 	\texttt{14040 [SQA Taipei-97]This is the WPS-PIN function issue.}
		\item 	\texttt{14035 [SQA Taipei-92]Wireless repeater function is fail.}
		\item 	\texttt{14036 [SQA Taipei-93]Wireless base sation function is fail.}
		\item 	\texttt{13770 WNR1000v2 5G band WDS doesn't work}
		\item 	\texttt{14031 [SQA Taipei-88]WPA/WPA2 enterprise of WPA-PSK[TKIP] mode cannot \
						correctly work.}
		\item 	\texttt{13767 [mercury-mms] [WPS IOT]WNHDE111 WPS with PBC method failed to work \
						with DUT in unconfigured state.}
		\item 	\texttt{13419 [Static Route] add a static route for the LAN port, and it doesn't work. \
						PS: wpn824v3 doesn't work either}
		\item 	\texttt{14088 [WPS]WPS icon in Vista shows "AtherosAP"}
		\item 	\texttt{13670 Network key attribute is missing in WPS M7 and M8}
		\item 	\texttt{13672 WPS proxy function does not work}
		\item 	\texttt{13780 [mercury-mms] [WPS IOT] DUT's 2.4G band WPS is fail both AP-PIN and \
						PBC mode when use Vista64 SP1 Feature Pack with WG111v3 and Centrino4965.}
		\item 	\texttt{13773 [mercury-mms] [WPS IOT]In XP OS WNDA3100v2 AP PIN failed to work with \
						WNR1000v2 in configured state.}		
	\end{itemize}
\end{enumerate}
\end{itemize}

\subsection{Steps to burn boot loader and firmware}
\begin{itemize}
\item As below:
        \begin{enumerate}
                \item \texttt{Please burn u-boot-1000V1.1.bin}
                \item \texttt{Set up a tftp server on your PC, its ip address is 192.168.1.12.}
                \item \texttt{Entering into boot loader}
                \item \texttt{ag7100> set serverip 192.168.1.12}
                \item \texttt{ag7100> tftp 0x80010000 u-boot-1000V1.1.bin}
				\item \texttt{ag7100> erase 0xbf000000 +0x50000}
				\item \texttt{ag7100> cp.b 0x80010000 0xbf000000 0x50000}
                \item \texttt{ag7100> set bootcmd 'fsload 80800000 image/uImage;bootm 80800000'}
                \item \texttt{ag7100> saveenv}
                \item \texttt{ag7100> reset}
                \item \texttt{Entering into boot loader again}
                \item \texttt{ag7100>bootm}
                \item \texttt{Then the device should be in tftp recovery mode. Please run the command "tftp -i 192.168.1.1 put WNR1000v2-V1.0.2.8.img" on MS-DOS of your PC.}
        \end{enumerate}
\end{itemize}

\subsection{Known issues}
        \begin{enumerate}
                \item \texttt{This version has been updated to Atheros LSDK 7.3.0.279.}
        \end{enumerate}

\section{Firmware V1.0.2.9}

\tlabel{sec:1-0-1}
\subsection{Repository}
\begin{itemize}
\item GIT Repository  itgserver/pub/scm/openwrt/ronger/openwrt.git
\begin{itemize}
    \item Branch: \texttt{sh\_sw\_one\_br}
    \item Tag: \texttt{WNR1000v2-V-1-0-2-9}
\end{itemize}
\end{itemize}

\subsection{Fixed Bugs}
\begin{itemize}
\item As below:
\begin{enumerate}
	\item APPLICATION
	\begin{itemize}
		\item 	\texttt{14052 [USB Storage]There is no share folder in HTTP and HTTP(via internet)}
		\item 	\texttt{14272 [NETGEAR]Defect ID : 165 [USB] Enable http access by default as per \
						USB Spec 1.4}
		\item 	\texttt{14163 [mercury-mms] Defect ID : 144 WNR1000v2\_Scan test: Port 53 issue Test \
						fail. It cannot add port 53 in port triggering page.}
		\item 	\texttt{14168 [mercury-mms] Defect ID : 142 WNR1000v2\_Scan test: : Attached device \
						list tested fail, the page always shown 22 items.}
		\item 	\texttt{14154 [mercury-mms] Defect ID : 146 WNR1000v2: First time router connected \
						to PPPoE, the DNS cannot work.}
		\item 	\texttt{14169 [mercury-mms] Defect ID : 138 WNR1000v2\_Scan test: PPPoE stress \
						test fail}
		\item 	\texttt{14059 [mercury-mms] Defect ID : 114 [USB] Access Controls dont seem to apply \
						when accessing via http}
		\item 	\texttt{14158 [mercury-mms] Defect ID : 143 WNR1000v2\_Scan test: Big Files (About 3G \
						files) cannot download complete from FTP server.}
		\item 	\texttt{13987 [SQA Taipei-55]When DUT WAN port mode is a DHCP or PPTP mode, the DHCP discover}
		\item 	\texttt{14191 add WPS\_onoff to control WPS run}
		\item 	\texttt{14259 [NETGEAR-Lancelot]PPTP can't work}
		\item 	\texttt{14073 [mercury-mms] Defect ID : 112 [USB] Approved Devices list doesnt \
						show all approved devices}
		\item 	\texttt{14137 [qos] can not set Uplink bandwidth correctly when the value is \
						up than 80Mbps}
	\end{itemize}
\end{enumerate}

\begin{enumerate}
	\item GUI
	\begin{itemize}
		\item 	\texttt{14078 [mercury-mms] Defect ID : 116 [USB] HTTP (via internet) \
						link should reflect current ddns domain.}
		\item 	\texttt{14079 [mercury-mms] Defect ID : 105 [USB] On the Create Network Folder \
						screen, why is the Folder Text field not available to manually enter the path}
		\item 	\texttt{14083 [mercury-mms] Defect ID : 118 [USB] Folder browser should only \
						show selected drive}
		\item 	\texttt{14085 [mercury-mms] Defect ID : 122 [USB] In folder browser the 1st level \
						folders should be displayed by default}
		\item 	\texttt{14082 [mercury-mms] Defect ID : 117 [USB] Duplicate Share Names between \
						different USB drives}
		\item 	\texttt{13989 [SQA Taipei-57]DUT will stop responding to LAN side client if in \
						PPTP connection the internal IP of DUT and external IP of DUT is in the same subnet.}
		\item 	\texttt{14086 [mercury-mms] Defect ID : 123 [USB] Folder Brower, when the USB \
						device is collapsed, it should show a + to indicate it can be expanded}
		\item 	\texttt{14028 [SQA Taipei-85]11a/n lack "Wireless AP" and "Broadcast Name" \
						in the router status page.}
		\item 	\texttt{14053 [QoS]QoS MAC Device List sometimes shows error.}
		\item 	\texttt{13690 [USB storage]There is no help page for USB storage}
		\item 	\texttt{14271 [NETGEAR]Defect ID : 163 [USB] On approved devices page, the \
						Device Name from row 1 is also used on all the other rows}
		\item 	\texttt{14266 [NETGEAR]Defect ID : 162 [USB] Typo's in USB Basic Settings}
		\item 	\texttt{13906 [NETGEAR-1139--Address Reservation] DUT is case sensitive to \
						letter in mac address if add address reservation items. -- copy from WPN824v3}
		\item 	\texttt{13259 [Static Route]When the DUT use "Use Static IP Address" as WAN IP \
						address, the WAN IP changes,the static route is not flushed}
	\end{itemize}
\end{enumerate}

\begin{enumerate}
	\item WIRELESS
	\begin{itemize}
\item 	\texttt{13167 [Wireless settings][a/n][b/g/n]When I set SSID as "any", it \
				shows NULL in iwconfig}
	\end{itemize}
\end{enumerate}
\end{itemize}

\subsection{Steps to burn boot loader and firmware}
\begin{itemize}
\item As below:
        \begin{enumerate}
                \item \texttt{Please burn u-boot-1000V1.1.bin}
                \item \texttt{Set up a tftp server on your PC, its ip address is 192.168.1.12.}
                \item \texttt{Entering into boot loader}
                \item \texttt{ag7100> set serverip 192.168.1.12}
                \item \texttt{ag7100> tftp 0x80010000 u-boot-1000V1.1.bin}
				\item \texttt{ag7100> erase 0xbf000000 +0x50000}
				\item \texttt{ag7100> cp.b 0x80010000 0xbf000000 0x50000}
                \item \texttt{ag7100> set bootcmd 'fsload 80800000 image/uImage;bootm 80800000'}
                \item \texttt{ag7100> saveenv}
                \item \texttt{ag7100> reset}
                \item \texttt{Entering into boot loader again}
                \item \texttt{ag7100>bootm}
                \item \texttt{Then the device should be in tftp recovery mode. Please run the command "tftp -i 192.168.1.1 put WNR1000v2-V1.0.2.9.img" on MS-DOS of your PC.}
        \end{enumerate}
\end{itemize}

\subsection{Known issues}
        \begin{enumerate}
                \item \texttt{This version has been updated to Atheros LSDK 7.3.0.290.}
        \end{enumerate}
		
\section{Firmware V1.0.3.0}

\tlabel{sec:1-0-1}
\subsection{Repository}
\begin{itemize}
\item GIT Repository  itgserver/pub/scm/openwrt/ronger/openwrt.git
\begin{itemize}
    \item Branch: \texttt{sh\_sw\_one\_br}
    \item Tag: \texttt{WNR1000v2-V-1-0-3-0}
\end{itemize}
\end{itemize}

\subsection{Fixed Bugs}
\begin{itemize}
\item As below:
\begin{enumerate}
	\item APPLICATION
	\begin{itemize}
		\item 	\texttt{14372 [NETGEAR-Lancelot]The friendly name of IGD is not right}
		\item 	\texttt{14332 [NETGEAR-Tai]H323 ALG crash on DHCP stress test --- cp from WNR1000}
		\item 	\texttt{14263 [NETGEAR]Defect ID : 161 Cannot copy large file to USB}
	\end{itemize}
\end{enumerate}

\begin{enumerate}
	\item GUI
	\begin{itemize}
		\item 	\texttt{14338 [NETGEAR] modify the timer of restore to default}
		\item 	\texttt{14336 Add a new check box in Approved Device Page}
		\item 	\texttt{14337 Add apply button in approved device page}
		\item 	\texttt{14335 HTTP and FTP port range should be from 1024 to 65534.}
		\item 	\texttt{14308 [NETGEAR]Defect ID : 170 [USB] Router should always give a \
						default Network Share name when no Share name is defined for the Volume}
		\item 	\texttt{14083 [mercury-mms] Defect ID : 118 [USB] Folder browser should only show selected drive}
		\item 	\texttt{14310 [NETGEAR]Defect ID : 169 [USB] Inconsistently the Browse \
						button on Create\/Edit Network Folder, requires multiple attempts to show tree}
		\item 	\texttt{14373 [NETGEAR] disable DFS channel for 1000}
	\end{itemize}
\end{enumerate}


\end{itemize}

\subsection{Steps to burn boot loader and firmware}
\begin{itemize}
\item As below:
        \begin{enumerate}
                \item \texttt{Please burn u-boot-1000V1.1.bin}
                \item \texttt{Set up a tftp server on your PC, its ip address is 192.168.1.12.}
                \item \texttt{Entering into boot loader}
                \item \texttt{ag7100> set serverip 192.168.1.12}
                \item \texttt{ag7100> tftp 0x80010000 u-boot-1000V1.1.bin}
				\item \texttt{ag7100> erase 0xbf000000 +0x50000}
				\item \texttt{ag7100> cp.b 0x80010000 0xbf000000 0x50000}
                \item \texttt{ag7100> set bootcmd 'fsload 80800000 image/uImage;bootm 80800000'}
                \item \texttt{ag7100> saveenv}
                \item \texttt{ag7100> reset}
                \item \texttt{Entering into boot loader again}
                \item \texttt{ag7100>bootm}
                \item \texttt{Then the device should be in tftp recovery mode. Please run the command "tftp -i 192.168.1.1 put WNR1000v2-V1.0.3.0.img" on MS-DOS of your PC.}
        \end{enumerate}
\end{itemize}

\subsection{Known issues}
        \begin{enumerate}
                \item \texttt{This version has been updated to Atheros LSDK 7.3.0.290.}
        \end{enumerate}

\section{Firmware V1.0.3.1}

\tlabel{sec:1-0-1}
\subsection{Repository}
\begin{itemize}
\item GIT Repository  itgserver/pub/scm/openwrt/ronger/openwrt.git
\begin{itemize}
    \item Branch: \texttt{sh\_sw\_one\_br}
    \item Tag: \texttt{WNR1000v2-V-1-0-3-1}
\end{itemize}
\end{itemize}

\subsection{Fixed Bugs}
\begin{itemize}
\item As below:
\begin{enumerate}
	\item APPLICATION
	\begin{itemize}
		\item 	\texttt{[BUG 14374] [NETGEAR-Lancelot] It will spend very long time to open samba}
		\item 	\texttt{bug 11286 smart antenna setting both 2.4G and 5G to antenna group 1 in art test mode}
	\end{itemize}
\end{enumerate}

\begin{enumerate}
	\item GUI
	\begin{itemize}
		\item 	\texttt{bug 14384 [NETGEAR]firefox & Safari cannot browse samba}
		\item 	\texttt{bug[14191]:add WPS\_onoff to control WPS run}
	\end{itemize}
\end{enumerate}


\end{itemize}

\subsection{Steps to burn boot loader and firmware}
\begin{itemize}
\item As below:
        \begin{enumerate}
                \item \texttt{Please burn u-boot-1000V1.1.bin}
                \item \texttt{Set up a tftp server on your PC, its ip address is 192.168.1.12.}
                \item \texttt{Entering into boot loader}
                \item \texttt{ag7100> set serverip 192.168.1.12}
                \item \texttt{ag7100> tftp 0x80010000 u-boot-1000V1.1.bin}
				\item \texttt{ag7100> erase 0xbf000000 +0x50000}
				\item \texttt{ag7100> cp.b 0x80010000 0xbf000000 0x50000}
                \item \texttt{ag7100> set bootcmd 'fsload 80800000 image/uImage;bootm 80800000'}
                \item \texttt{ag7100> saveenv}
                \item \texttt{ag7100> reset}
                \item \texttt{Entering into boot loader again}
                \item \texttt{ag7100>bootm}
                \item \texttt{Then the device should be in tftp recovery mode. Please run the command "tftp -i 192.168.1.1 put WNR1000v2-V1.0.3.1.img" on MS-DOS of your PC.}
        \end{enumerate}
\end{itemize}

\subsection{Known issues}
        \begin{enumerate}
                \item \texttt{This version has been updated to Atheros LSDK 7.3.0.290.}
        \end{enumerate}

\section{Firmware V1.0.3.2}

\tlabel{sec:1-0-1}
\subsection{Repository}
\begin{itemize}
\item GIT Repository  itgserver/pub/scm/openwrt/ronger/openwrt.git
\begin{itemize}
    \item Branch: \texttt{sh\_sw\_one\_br}
    \item Tag: \texttt{WNR1000v2-V-1-0-3-2}
\end{itemize}
\end{itemize}

\subsection{Fixed Bugs}
\begin{itemize}
\item As below:
\begin{enumerate}
	\item APPLICATION
	\begin{itemize}
		\item 	\texttt{13557 [USB storage]According to New spec,when a user accesses a \
						Network Folder which is assigned to the All--no password the user \
						should NOT be prompted by a message asking them to enter any password}
		\item 	\texttt{14457 Change the behavior of USB Storage after deleting all the entries \
						of shared folders}
		\item 	\texttt{14421 smb:\/\/ServerName\/ShareName is failed under MAC OS X}
		\item 	\texttt{13757 [mercury-mms] DUT will forward all echo/chargen packets to LAN \
						when config port forwarding}
		\item 	\texttt{14165 [mercury-mms] Defect ID : 137 WNR1000v2_Scan test: Router reboot \
						on DHCP stress test after 4 days.}
		\item 	\texttt{14439 [NETGEAR]Defect ID : 171 PPPoE Dial on Demand mode can't connect \
						to Internet when we click 'Connect' button on connection status page.}
		\item 	\texttt{14462 [USB Storage] The DUT will be locked for about 30 seconds when first booting}
		\item 	\texttt{13845 [USB storage]Click the link of firefox and safari can not visit USB storega}
		\item 	\texttt{13978 [SQA Taipei-47]When download a second file from FTP server, user must re-login \
						otherwise the FTP command will be fail and it will response "cannot open data connection".}
		\item 	\texttt{14448 [NETGEAR - Lancelot][USB storage]Only after unplug and plug in, then all \
						partitions will be shared automatically if all shared folders are deleted.}
		\item 	\texttt{14363 [USB storage]After upload a large file to USB storage, fireware upgrade fail.}
		\item 	\texttt{14455 set eth0 MAC address as LAN MAC even MAC of WLAN 5g is less than eth0.}
		\item 	\texttt{14290 all uint64 datas in lltd packets are error in endian.}
	\end{itemize}
\end{enumerate}

\begin{enumerate}
	\item GUI
	\begin{itemize}
		\item 	\texttt{14082 [mercury-mms] Defect ID : 117 [USB] Duplicate Share Names between different \
						USB drives}
		\item 	\texttt{14430 Web page cannot be displayed is we restore a configuration file}
		\item 	\texttt{14066 [Router status]When the DUT work in WDS Repeater mode, an mode infomation \
						disappears in Wireless Port}
		\item 	\texttt{14308 [NETGEAR]Defect ID : 170 [USB] Router should always give a default Network \
						Share name when no Share name is defined for the Volume}
		\item 	\texttt{14371 [GUI] add a hidden page for setting smart antenna.}
		\item 	\texttt{14459 After enable WDS, if we try to change wireless settings, but we will get \
						an error message.}
		\item 	\texttt{14373 [NETGEAR] disable DFS channel for 1000}
		\item 	\texttt{14117 [Remote Management]The 0.0.0.0 should be displayed on the remote \
						management address if no WAN connection or no IP is obtained}
		\item 	\texttt{14442 [Wireless]Middle East + 300M + channel 44 in an mode will call the DUT crash}
		\item 	\texttt{14463 Add auto channel to GUI}
		\item 	\texttt{14432 If SSID or PSK has many special characters, the wireless cannot configure them normally.}
		\item 	\texttt{13991 [SQA Taipei-59]Change WAN port mode from static IP to PPTP mode, \
						the My IP Address of PPTP automatically saves static IP address.}
		\item 	\texttt{14372 [NETGEAR-Lancelot]The friendly name of IGD is not right}
		\item 	\texttt{13201 [Wireless settings][a/n][b/g/n]If I set WPA/WPA2-PSK PassPhrase as many special characters \
						it shows no wireless in iwconfig}
	\end{itemize}
\end{enumerate}

\begin{enumerate}
	\item WIRELESS
	\begin{itemize}
		\item 	\texttt{14443 [Wireless]If only an band is enabled, no matter what channel you set, \
						it always works in channel 40}
		\item 	\texttt{14213 [WPS]If the router is WPS Unconfigured in both an and bgn mode, after WPS process, \
				the wireless settings be generated automatically shows error.}
	\end{itemize}
\end{enumerate}

\end{itemize}

\subsection{Steps to burn boot loader and firmware}
\begin{itemize}
\item As below:
        \begin{enumerate}
                \item \texttt{Please burn u-boot-1000V1.1.bin}
                \item \texttt{Set up a tftp server on your PC, its ip address is 192.168.1.12.}
                \item \texttt{Entering into boot loader}
                \item \texttt{ag7100> set serverip 192.168.1.12}
                \item \texttt{ag7100> tftp 0x80010000 u-boot-1000V1.1.bin}
				\item \texttt{ag7100> erase 0xbf000000 +0x50000}
				\item \texttt{ag7100> cp.b 0x80010000 0xbf000000 0x50000}
                \item \texttt{ag7100> set bootcmd 'fsload 80800000 image/uImage;bootm 80800000'}
                \item \texttt{ag7100> saveenv}
                \item \texttt{ag7100> reset}
                \item \texttt{Entering into boot loader again}
                \item \texttt{ag7100>bootm}
                \item \texttt{Then the device should be in tftp recovery mode. Please run the command "tftp -i 192.168.1.1 put WNR1000v2-V1.0.3.2.img" on MS-DOS of your PC.}
        \end{enumerate}
\end{itemize}

\subsection{Known issues}
        \begin{enumerate}
                \item \texttt{This version has been updated to Atheros LSDK 7.3.0.290.}
        \end{enumerate}

\section{Firmware V1.0.3.3}

\tlabel{sec:1-0-1}
\subsection{Repository}
\begin{itemize}
\item GIT Repository  itgserver/pub/scm/openwrt/ronger/openwrt.git
\begin{itemize}
    \item Branch: \texttt{sh\_sw\_one\_br}
    \item Tag: \texttt{WNR1000v2-V-1-0-3-3}
\end{itemize}
\end{itemize}

\subsection{Fixed Bugs}
\begin{itemize}
\item As below:
\begin{enumerate}
	\item APPLICATION
	\begin{itemize}
		\item 	\texttt{13825 [USB storage]If there are many spaces in share name, \
						it only shows one when you visit the USB storage}
		\item 	\texttt{13838 [USB storage]File System ext3 shows error on "Edit Network Folder" page.}
		\item 	\texttt{14436 [USB storage]If there are 2 device connect to the DUT, it will not give \
						a default Network Share name to the second device}
	\end{itemize}
\end{enumerate}

\begin{enumerate}
	\item GUI
	\begin{itemize}
		\item 	\texttt{14430 Web page cannot be displayed is we restore a configuration file}
		\item 	\texttt{14463 Add auto channel to GUI}
		\item 	\texttt{14490 [NETGEAR]change "USB NETStorage" to "USB Storage"}
		\item 	\texttt{14505 [DNI]update language table }
		\item 	\texttt{14491 [NETGEAR]Additional ��Not Shared�� functionality}
		\item 	\texttt{14057 [router status]The repeater function of b\/g\/n is up \
						and a\/n is down when I closed the b\/g\/n and opened the a\/n \
						in "wireless repeating function" page!}
		\item 	\texttt{13906 [NETGEAR-1139--Address Reservation] DUT is case sensitive \
						to letter in mac address if add address reservation items. -- copy from WPN824v3 }
	\end{itemize}
\end{enumerate}

\begin{enumerate}
	\item WIRELESS
	\begin{itemize}
		\item 	\texttt{13785 [mercury-mms] When 2.4G WDS is successful, both base station and \
						repeater GUI show that status is disconnection.}
		\item 	\texttt{14443 [Wireless]If only an band is enabled, no matter what channel you \
						set, it always works in channel 40 }
		\item 	\texttt{14432 If SSID or PSK special characters, the wireless cannot configure them normally. }
	\end{itemize}
\end{enumerate}

\end{itemize}

\subsection{Steps to burn boot loader and firmware}
\begin{itemize}
\item As below:
        \begin{enumerate}
                \item \texttt{Please burn u-boot-1000V1.1.bin}
                \item \texttt{Set up a tftp server on your PC, its ip address is 192.168.1.12.}
                \item \texttt{Entering into boot loader}
                \item \texttt{ag7100> set serverip 192.168.1.12}
                \item \texttt{ag7100> tftp 0x80010000 u-boot-1000V1.1.bin}
				\item \texttt{ag7100> erase 0xbf000000 +0x50000}
				\item \texttt{ag7100> cp.b 0x80010000 0xbf000000 0x50000}
                \item \texttt{ag7100> set bootcmd 'fsload 80800000 image/uImage;bootm 80800000'}
                \item \texttt{ag7100> saveenv}
                \item \texttt{ag7100> reset}
                \item \texttt{Entering into boot loader again}
                \item \texttt{ag7100>bootm}
                \item \texttt{Then the device should be in tftp recovery mode. Please run the command "tftp -i 192.168.1.1 put WNR1000v2-V1.0.3.3.img" on MS-DOS of your PC.}
        \end{enumerate}
\end{itemize}

\subsection{Known issues}
        \begin{enumerate}
                \item \texttt{This version has been updated to Atheros LSDK 7.3.0.290.}
        \end{enumerate}

\section{Firmware V1.0.3.4}

\tlabel{sec:1-0-1}
\subsection{Repository}
\begin{itemize}
\item GIT Repository  itgserver/pub/scm/openwrt/ronger/openwrt.git
\begin{itemize}
    \item Branch: \texttt{sh\_sw\_one\_br}
    \item Tag: \texttt{WNR1000v2-V-1-0-3-4}
\end{itemize}
\end{itemize}

\subsection{Fixed Bugs}
\begin{itemize}
\item As below:
\begin{enumerate}
	\item APPLICATION
	\begin{itemize}
		\item 	\texttt{14369 wnr1000v2 \/proc\/uptime is not normal}
		\item 	\texttt{13259 [Static Route]When the DUT use "Use Static IP Address" \
						as WAN IP address, the WAN IP changes,the static route is not flushed}
	\end{itemize}
\end{enumerate}

\begin{enumerate}
	\item GUI
	\begin{itemize}
		\item 	\texttt{14430 Web page cannot be displayed is we restore a configuration file}
		\item 	\texttt{14384 [NETGEAR]firefox & Safari cannot browse samba}
		\item 	\texttt{13688 [USB storage]If I set share name as \
						it shows different in USB storage}
		\item 	\texttt{13824 [USB storage]If there are many spaces in share name, \
						we can not visit the share folder by click the link }
		\item 	\texttt{14173 [NETGEAR Lancelot] In the router status page, you only \
						show channel not including auto }
		\item 	\texttt{14546 [USB storage]"Edit" share folder with different access \
						will pop up share name has already existed message. }
		\item 	\texttt{[NETGEAR-Lancelot] ��Edit�� should be grayed out if there is not network folder. }
		\item   \texttt{14372 [NETGEAR-Lancelot]The friendly name of IGD is not right}
	\end{itemize}
\end{enumerate}

\begin{enumerate}
	\item WIRELESS
	\begin{itemize}
		\item 	\texttt{14511 WDS client mode can not associate with WDS base station}
	\end{itemize}
\end{enumerate}

\end{itemize}

\subsection{Steps to burn boot loader and firmware}
\begin{itemize}
\item As below:
        \begin{enumerate}
                \item \texttt{Please burn u-boot-1000V1.3.bin}
                \item \texttt{Set up a tftp server on your PC, its ip address is 192.168.1.12.}
                \item \texttt{Entering into boot loader}
                \item \texttt{ag7100> set serverip 192.168.1.12}
                \item \texttt{ag7100> tftp 0x80010000 u-boot-1000V1.3.bin}
				\item \texttt{ag7100> erase 0xbf000000 +0x50000}
				\item \texttt{ag7100> cp.b 0x80010000 0xbf000000 0x50000}
                \item \texttt{ag7100> set bootcmd 'fsload 80800000 image/uImage;bootm 80800000'}
                \item \texttt{ag7100> saveenv}
                \item \texttt{ag7100> reset}
                \item \texttt{Entering into boot loader again}
                \item \texttt{ag7100>bootm}
                \item \texttt{Then the device should be in tftp recovery mode. Please run the command "tftp -i 192.168.1.1 put WNR1000v2-V1.0.3.4.img" on MS-DOS of your PC.}
        \end{enumerate}
\end{itemize}

\subsection{Known issues}
        \begin{enumerate}
                \item \texttt{This version has been updated to Atheros LSDK 7.3.0.312.}
        \end{enumerate}

\section{Firmware V1.0.3.5}

\tlabel{sec:1-0-1}
\subsection{Repository}
\begin{itemize}
\item GIT Repository  itgserver/pub/scm/openwrt/ronger/openwrt.git
\begin{itemize}
    \item Branch: \texttt{sh\_sw\_one\_br}
    \item Tag: \texttt{WNR1000v2-V-1-0-3-5}
\end{itemize}
\end{itemize}

\subsection{Fixed Bugs}
\begin{itemize}
\item As below:
\begin{enumerate}
	\item APPLICATION
	\begin{itemize}
		\item 	\texttt{14577 Change each digit in firmware version to support up to 999.}
		\item 	\texttt{14591 Ignore ��New region�� in Smart Wizard.}
		\item 	\texttt{14612 Authentication dialog must be pop-up after time out on clicking apply button .}
	\end{itemize}
\end{enumerate}

\begin{enumerate}
	\item GUI
	\begin{itemize}
		\item 	\texttt{14525 [wds]when I use WDS in a/n mode,the security of wpa.wpa2 and so \
						on will not gray out }
		\item 	\texttt{14541 [USB storage]USB device in folder item on "Create Network Folder" and \
						"Edit Network Folder" page should allow lower case character }
		\item 	\texttt{14542 [USB storage]Please check the share name including any upper\/lower case combination \
						(for example, ��TEST" or ��test��) as a same share name}
		\item 	\texttt{14524 [WDS] UI bug in "router status" page }
		\item 	\texttt{14507 [WPS]If bgn mode is Configured and an mode is unconfigured, add a WPS client \
						with an mode, the warning message "Note: The wireless settings have been changed." \
						is missing on Success page .}
		\item 	\texttt{14567 There is one string in WLG\_wireless.htm not merged into lang.c.}
		\item 	\texttt{14568 WPS should not run when WDS is enable}
		\item   \texttt{14576 [Basic settings]Reboot the DUT after click the "Test" button on "Basic Settings" \
						page, it will pop up Test page again.}
		\item   \texttt{12073 [Basic Wireless settings][a\/n][b\/g\/n]When SSID is <\/script> , Basic wireless \
						settings page and router status page shows error}
		\item   \texttt{14580 [NETGEAR]Defect ID :177 - When 5G band with "Auto" channel select 36(p) + \
						40(s) as displayed in router status page, indeed it operates at 40(p) + 36(s). }
		\item   \texttt{12101 [Basic Wireless settings][a\/n][b\/g\/n][Guest Network][WPA-PSK]If I set WPA-PSK \
						PassPhrase as <\/script>,the webpage shows error }
		\item   \texttt{12143 [Guest Network][b\/g\/n]The webpage shows error when you input some special characters as SSID }
		\item   \texttt{12105 [Basic Wireless settings][a\/n][b\/g\/n][Guest Network][WPA\/WPA2 Enterprise]When I set \
						RADIUS server Shared Secret as special chars security shows nothing}
		\item   \texttt{14564 [Connection status]DHCP lease Expires shows error in Firefox and safari. }
		\item   \texttt{14597 Remove WDS from GUI }
	\end{itemize}
\end{enumerate}

\end{itemize}

\subsection{Steps to burn boot loader and firmware}
\begin{itemize}
\item As below:
        \begin{enumerate}
                \item \texttt{Please burn u-boot-1000V1.3.bin}
                \item \texttt{Set up a tftp server on your PC, its ip address is 192.168.1.12.}
                \item \texttt{Entering into boot loader}
                \item \texttt{ag7100> set serverip 192.168.1.12}
                \item \texttt{ag7100> tftp 0x80010000 u-boot-1000V1.3.bin}
				\item \texttt{ag7100> erase 0xbf000000 +0x50000}
				\item \texttt{ag7100> cp.b 0x80010000 0xbf000000 0x50000}
                \item \texttt{ag7100> set bootcmd 'fsload 80800000 image/uImage;bootm 80800000'}
                \item \texttt{ag7100> saveenv}
                \item \texttt{ag7100> reset}
                \item \texttt{Entering into boot loader again}
                \item \texttt{ag7100>bootm}
                \item \texttt{Then the device should be in tftp recovery mode. Please run the command "tftp -i 192.168.1.1 put WNR1000v2-V1.0.3.5.img" on MS-DOS of your PC.}
        \end{enumerate}
\end{itemize}

\subsection{Known issues}
        \begin{enumerate}
                \item \texttt{This version has been updated to Atheros LSDK 7.3.0.321.}
        \end{enumerate}		

\section{Firmware V1.0.3.6}

\tlabel{sec:1-0-1}
\subsection{Repository}
\begin{itemize}
\item GIT Repository  itgserver/pub/scm/openwrt/ronger/openwrt.git
\begin{itemize}
    \item Branch: \texttt{sh\_sw\_one\_br}
    \item Tag: \texttt{WNR1000v2-V-1-0-3-6}
\end{itemize}
\end{itemize}

\subsection{Fixed Bugs}
\begin{itemize}
\item As below:
\begin{enumerate}
	\item APPLICATION
	\begin{itemize}
		\item 	\texttt{14552 [DDNS]With error passowrd in DDNS page, it shows update \
						successful in "Show status" page.}
	\end{itemize}
\end{enumerate}

\begin{enumerate}
	\item GUI
	\begin{itemize}
		\item 	\texttt{14541 [USB storage]USB device in folder item on "Create \
						Network Folder" and "Edit Network Folder" page should allow lower case character}
		\item 	\texttt{12105 [Basic Wireless settings][a\/n][b\/g\/n][Guest Network][WPA\/WPA2 Enterprise] \
						When I set RADIUS server Shared Secret as characters security shows nothing}		
		\item 	\texttt{14605 [WPS]Configure APUT using PIN method through Vista or Intel PROSet,if SSID is special characters, \
						it shows OK in iwconfig,but UI is error}
\end{enumerate}

\begin{enumerate}
       \item WIRELESS
       \begin{itemize}
               \item   \texttt{14528 [WDS]the function of "disable wireless client association" works incorrectly}
               \item   \texttt{14609 change country region according Netgear requirement }
               \item   \texttt{14598 WPS is not up if only one radio is enabled}
               \item   \texttt{14599 wps interface is printed even that interface is disabled }
       \end{itemize}
\end{enumerate}


\end{itemize}

\subsection{Steps to burn boot loader and firmware}
\begin{itemize}
\item As below:
        \begin{enumerate}
                \item \texttt{Please burn u-boot-1000V1.4.bin}
                \item \texttt{Set up a tftp server on your PC, its ip address is 192.168.1.12.}
                \item \texttt{Entering into boot loader}
                \item \texttt{ag7100> set serverip 192.168.1.12}
                \item \texttt{ag7100> tftp 0x80010000 u-boot-1000V1.4.bin}
				\item \texttt{ag7100> erase 0xbf000000 +0x70000}
				\item \texttt{ag7100> cp.b 0x80010000 0xbf000000 0x50000}
                \item \texttt{ag7100> reset}
                \item \texttt{Entering into boot loader again}
                \item \texttt{ag7100>bootm}
                \item \texttt{Then the device should be in tftp recovery mode. Please run the command "tftp -i 192.168.1.1 put WNR1000v2-V1.0.3.6.img" on MS-DOS of your PC.}
        \end{enumerate}
\end{itemize}

\subsection{Known issues}
        \begin{enumerate}
                \item \texttt{This version has been updated to Atheros LSDK 7.3.0.321.}
        \end{enumerate}	
		
\section{Firmware V1.0.3.7}

\tlabel{sec:1-0-1}
\subsection{Repository}
\begin{itemize}
\item GIT Repository  itgserver/pub/scm/openwrt/ronger/openwrt.git
\begin{itemize}
    \item Branch: \texttt{sh\_sw\_one\_br}
    \item Tag: \texttt{WNR1000v2-V-1-0-3-7}
\end{itemize}
\end{itemize}

\subsection{Fixed Bugs}
\begin{itemize}
\item As below:

\begin{enumerate}
	\item GUI
	\begin{itemize}	
		\item   \texttt{14507 [WPS]If bgn mode is Configured and an mode is unconfigured, \
						add a WPS client with an mode, the warning message "Note: The \
						wireless settings have been changed."is missing on Success page .}
		\item   \texttt{14641 [NETGEAR-USB]change default value, when edits a 'Not Shared' \
						Network Folder}
		\item   \texttt{14601 [Wireless ACL] The device name shows error in Wireless Card \
						Entry on "Wireless Card Access Setup" page }
		\item   \texttt{14648 [NETGEAR]When the default share is deleted and ��Not Shared�� \
						is displayed, the table should still be sorted by the drive letter field}
		\item   \texttt{14720 [NETGEAR]Remove ��Auto�� from 5GH channel list and keep it in 2.4G. }
		\item   \texttt{14727 [NETGEAR-179]WPS successful page has error message when DUT is \
						in configured mode.}
		\item   \texttt{14708 [Static Route]Destination Subnet address should be in different \
						subnet with LAN or WAN interface.}
\end{enumerate}

\begin{enumerate}
	\item WIRELESS
	\begin{itemize}
		\item 	\texttt{14031 [SQA Taipei-88]WPA\/WPA2 enterprise of WPA-PSK[TKIP] mode \
						cannot correctly work}
		\item 	\texttt{[NETGEAR]Change FCC and CE region back to normal one. FCC 3 and \
							ETSI 1 because we will remove ��Auto��.}
		\item 	\texttt{[wps]When use AP PIN config the DUT ,if the PassPhrase include \
						some special characters,the page display wrong}
	\end{itemize}
\end{enumerate}

\end{itemize}

\subsection{Steps to burn boot loader and firmware}
\begin{itemize}
\item As below:
        \begin{enumerate}
                \item \texttt{Please burn u-boot-1000V1.4.bin}
                \item \texttt{Set up a tftp server on your PC, its ip address is 192.168.1.12.}
                \item \texttt{Entering into boot loader}
                \item \texttt{ag7100> set serverip 192.168.1.12}
                \item \texttt{ag7100> tftp 0x80010000 u-boot-1000V1.4.bin}
				\item \texttt{ag7100> erase 0xbf000000 +0x70000}
				\item \texttt{ag7100> cp.b 0x80010000 0xbf000000 0x50000}
                \item \texttt{ag7100> reset}
                \item \texttt{Entering into boot loader again}
                \item \texttt{ag7100>bootm}
                \item \texttt{Then the device should be in tftp recovery mode. Please run the command "tftp -i 192.168.1.1 put WNR1000v2-V1.0.3.7.img" on MS-DOS of your PC.}
        \end{enumerate}
\end{itemize}

\subsection{Known issues}
        \begin{enumerate}
                \item \texttt{This version has been updated to Atheros LSDK 7.3.0.321.}
        \end{enumerate}	


\section{Firmware V1.0.3.8}

\tlabel{sec:1-0-1}
\subsection{Repository}
\begin{itemize}
\item GIT Repository  itgserver/pub/scm/openwrt/ronger/openwrt.git
\begin{itemize}
    \item Branch: \texttt{sh\_sw\_one\_br}
    \item Tag: \texttt{WNR1000v2-V-1-0-3-8}
\end{itemize}
\end{itemize}

\subsection{Fixed Bugs}
\begin{itemize}
\item As below:
\begin{enumerate}
	\item APPLICATION
	\begin{itemize}
		\item 	\texttt{[BUG 14747] [USB Storage] Samba can not be accessed if LAN IP is changed}
	\end{itemize}
\end{enumerate}

\end{itemize}

\subsection{Steps to burn boot loader and firmware}
\begin{itemize}
\item As below:
        \begin{enumerate}
                \item \texttt{Please burn u-boot-1000V1.4.bin}
                \item \texttt{Set up a tftp server on your PC, its ip address is 192.168.1.12.}
                \item \texttt{Entering into boot loader}
                \item \texttt{ag7100> set serverip 192.168.1.12}
                \item \texttt{ag7100> tftp 0x80010000 u-boot-1000V1.4.bin}
				\item \texttt{ag7100> erase 0xbf000000 +0x70000}
				\item \texttt{ag7100> cp.b 0x80010000 0xbf000000 0x50000}
                \item \texttt{ag7100> reset}
                \item \texttt{Entering into boot loader again}
                \item \texttt{ag7100>bootm}
                \item \texttt{Then the device should be in tftp recovery mode. Please run the command "tftp -i 192.168.1.1 put WNR1000v2-V1.0.3.8.img" on MS-DOS of your PC.}
        \end{enumerate}
\end{itemize}

\subsection{Known issues}
        \begin{enumerate}
                \item \texttt{This version has been updated to Atheros LSDK 7.3.0.321.}
        \end{enumerate}			
		
\section{Firmware V1.0.3.9}

\tlabel{sec:1-0-1}
\subsection{Repository}
\begin{itemize}
\item GIT Repository  itgserver/pub/scm/openwrt/ronger/openwrt.git
\begin{itemize}
    \item Branch: \texttt{sh\_sw\_one\_br}
    \item Tag: \texttt{WNR1000v2-V-1-0-3-9}
\end{itemize}
\end{itemize}

\subsection{Fixed Bugs}
\begin{itemize}
\item As below:
\begin{enumerate}
	\item APPLICATION
	\begin{itemize}
		\item 	\texttt{14787 [NETGEAR 180]After WAN has idled time-out, clicking \
						"Check" in router upgrade page cannot trigger the WAN connection.}
		\item	\texttt{14806 [NETGEAR]Defect ID : 187 - there is no log when input \
						incorrect password from remote login.}
		\item	\texttt{14805 [NETGEAR]Defect ID : 188 - DUT will not send Email \
						immediately when access block site.}
		\item	\texttt{14803 [NETGEAR]Defect ID : 186 - DUT should have warning \
						message and don't allow user to input space key as account name.}
		\item	\texttt{14768 [WPS]DUT can't stop 1st WPS-PIN progress and restart 2nd progress.}
		\item	\texttt{14804 [NETGEAR]Defect ID : 185 - When set to time zone "GMT+5:30" \
						DUT will get a time same as set to "GMT+5:00". }
		\item	\texttt{14801 [NETGEAR]Defect ID : 182 - On NA firmware \
						the default time zone is set to "GMT". }
		\item	\texttt{14802 [NETGEAR]Defect ID : 184 - The "Daylight Savings" seems not take effect.}

		
	\end{itemize}
\end{enumerate}

\begin{enumerate}
       \item GUI
       \begin{itemize}
			   \item   \texttt{14185 [GUI] Qos page,when apply,only restart wireless when \
								WMM was changed. }
               \item   \texttt{14708 [Static Route]Destination Subnet address should be in \
								different subnet with LAN or WAN interface}
               \item   \texttt{14056 [router status]when I use 4 wrong Mac address in base \
								station mode,it will all display "disconnected",but when I use \
								1 correct and 3 wrong MAC address, it will all display "connected" \
								in "router status" page! }
               \item   \texttt{14666 [WDS] There is a UI bug in "router status" page when enable \
								"disable wireless client association" in base station and repeater mode.}
               \item   \texttt{14780 [Wireless]According to Spec, "Middle East" should show \
								after "Mexico" in the drop down list }
				\item   \texttt{14777 [Block Services ]Should not put same Service Type}
				\item	\texttt{14775 [Block Sites]"Trusted IP Address"should gray-out }
				\item	\texttt{12105 [Basic Wireless settings][a/n][b/g/n][Guest Network] \
								[WPA/WPA2 Enterprise]When I set RADIUS server Shared Secret \
								as special chars security shows nothing}
				\item	\texttt{14770 [LAN Subnet]If subnet is not changed but IP address \
								is changed, router should automatically comply these corresponding \
								rules to match the new IP}
				\item	\texttt{14767 [LAN Subnet]Change LAN IP subnet mask from 24 to 16, \
								the rules in block service list have been deleted }
				\item	\texttt{14769 [LAN Subnet]If subnet is changed and the subnet \
								range is from big to small, DMZ and Trusted IP Address on \
								block site page SHOULD be flushed}
				\item	\texttt{14791 [Wireless]WPAPassPhrase with some special characters, \
								client can not connect to the DUT.}
				\item	\texttt{14817 [NETGEAR]Defect ID : 189 - "Static Routes" edit function has issue.}
				
       \end{itemize}
\end{enumerate}

\end{itemize}

\subsection{Steps to burn boot loader and firmware}
\begin{itemize}
\item As below:
        \begin{enumerate}
                \item \texttt{Please burn u-boot-1000V1.4.bin}
                \item \texttt{Set up a tftp server on your PC, its ip address is 192.168.1.12.}
                \item \texttt{Entering into boot loader}
                \item \texttt{ag7100> set serverip 192.168.1.12}
                \item \texttt{ag7100> tftp 0x80010000 u-boot-1000V1.4.bin}
				\item \texttt{ag7100> erase 0xbf000000 +0x70000}
				\item \texttt{ag7100> cp.b 0x80010000 0xbf000000 0x50000}
                \item \texttt{ag7100> reset}
                \item \texttt{Entering into boot loader again}
                \item \texttt{ag7100>bootm}
                \item \texttt{Then the device should be in tftp recovery mode. Please run the command "tftp -i 192.168.1.1 put WNR1000v2-V1.0.3.9.img" on MS-DOS of your PC.}
        \end{enumerate}
\end{itemize}

\subsection{Known issues}
        \begin{enumerate}
                \item \texttt{This version has been updated to Atheros LSDK 7.3.0.321.}
        \end{enumerate}	

\section{Firmware V1.0.3.10}

\tlabel{sec:1-0-1}
\subsection{Repository}
\begin{itemize}
\item GIT Repository  itgserver/pub/scm/openwrt/ronger/openwrt.git
\begin{itemize}
    \item Branch: \texttt{sh\_sw\_one\_br}
    \item Tag: \texttt{WNR1000v2-V-1-0-3-10}
\end{itemize}
\end{itemize}

\subsection{Fixed Bugs}
\begin{itemize}
\item As below:
\begin{enumerate}
	\item APPLICATION
	\begin{itemize}
		\item 	\texttt{14844 [NETGEAR]a major issue which causes uhttpd crash and we cant see \
						USB basic setting content.}
		\item 	\texttt{14828 [NETGEAR]Defect ID : 190 - When DDNS update failed because the \
						authentication failed the info showed in DDNS status page is incorrect.}
		\item 	\texttt{14819 [Wireless]When radius server in WAN, the DUT can not find the radius server}
		\item 	\texttt{14881 [Wireless][Guest bgn/an ]Fragmentation does not work with Guest Network}
		\item 	\texttt{14822 [Wireless][Guest Network bgn]WPAPassPhrase with space , wireless \
						client can not connect to the Guest Network.}		
		\item 	\texttt{14860 [NETGEAR]Defect ID : 196 - Tracert cmd on LAN PC does not work normally.}
		\item 	\texttt{14859 [NETGEAR]Defect ID : 195 - If LAN PC sends UDP packet to WAN PC first, \
						the ICMP destination unreachable packets generated by WAN PC shoud be \
						forwarded to this LAN PC.}
		\item 	\texttt{14861 [NETGEAR]Defect ID : 197 - WNR1000v2 port 5060 issue}
		\item 	\texttt{14788 [NETGEAR 181]There is strange log in system logs.}
	\end{itemize}
\end{enumerate}

\begin{enumerate}
       \item GUI
       \begin{itemize}
			\item 	\texttt{14790 [Intel Viiv]WPAPassPhrase with special characters shows error \
							in router-info.htm}
			\item 	\texttt{14808 [port forwarding]when I change LAN IP to another subnet, the \
							service will be deleted automatically}
			\item 	\texttt{14811 [port forwarding]when I change LAN IP to another subnet, \
							the service rule's ip should change automatically}
			\item 	\texttt{14770 [LAN Subnet]If subnet is not changed but IP address is changed, \
							router should automatically comply these corresponding rules \
							to match the new IP}
			\item 	\texttt{14845 [NEETGEAR-USB]the tag in file\_info should change to "USB Storage"}
			\item 	\texttt{14820 [Wireless][Guest Network bgn]WEP can save Passphrase as 2-byte \
							characters, it is wrong.}
			\item 	\texttt{14791 [Wireless]WPAPassPhrase with some special characters, client \
							can not connect to the DUT.}
			\item 	\texttt{14776 [WAN/LAN conflict]There is no WAN/LAN confilict check when PPPoE \
							and no cable connects to the DUT's WAN}
			\item 	\texttt{14773 [WAN/LAN conflict]There is no WAN/LAN confilict check when Static \
							WAN IP address and no cable connects to the DUT's WAN}
			\item 	\texttt{14849 [NETGEAR-194]WAN help content on right hand side should be modified}
		\end{itemize}
\end{enumerate}

\end{itemize}

\subsection{Steps to burn boot loader and firmware}
\begin{itemize}
\item As below:
        \begin{enumerate}
                \item \texttt{Please burn u-boot-1000V1.4.bin}
                \item \texttt{Set up a tftp server on your PC, its ip address is 192.168.1.12.}
                \item \texttt{Entering into boot loader}
                \item \texttt{ag7100> set serverip 192.168.1.12}
                \item \texttt{ag7100> tftp 0x80010000 u-boot-1000V1.4.bin}
				\item \texttt{ag7100> erase 0xbf000000 +0x70000}
				\item \texttt{ag7100> cp.b 0x80010000 0xbf000000 0x50000}
                \item \texttt{ag7100> reset}
                \item \texttt{Entering into boot loader again}
                \item \texttt{ag7100>bootm}
                \item \texttt{Then the device should be in tftp recovery mode. Please run the command "tftp -i 192.168.1.1 put WNR1000v2-V1.0.3.10.img" on MS-DOS of your PC.}
        \end{enumerate}
\end{itemize}

\subsection{Known issues}
        \begin{enumerate}
                \item \texttt{This version has been updated to Atheros LSDK 7.3.0.321.}
        \end{enumerate}	
		
\end{document} 
